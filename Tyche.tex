%% Generated by Sphinx.
\def\sphinxdocclass{report}
\documentclass[letterpaper,10pt,english]{sphinxmanual}
\ifdefined\pdfpxdimen
   \let\sphinxpxdimen\pdfpxdimen\else\newdimen\sphinxpxdimen
\fi \sphinxpxdimen=.75bp\relax
\ifdefined\pdfimageresolution
    \pdfimageresolution= \numexpr \dimexpr1in\relax/\sphinxpxdimen\relax
\fi
%% let collapsible pdf bookmarks panel have high depth per default
\PassOptionsToPackage{bookmarksdepth=5}{hyperref}
%% turn off hyperref patch of \index as sphinx.xdy xindy module takes care of
%% suitable \hyperpage mark-up, working around hyperref-xindy incompatibility
\PassOptionsToPackage{hyperindex=false}{hyperref}
%% memoir class requires extra handling
\makeatletter\@ifclassloaded{memoir}
{\ifdefined\memhyperindexfalse\memhyperindexfalse\fi}{}\makeatother

\PassOptionsToPackage{warn}{textcomp}

\catcode`^^^^00a0\active\protected\def^^^^00a0{\leavevmode\nobreak\ }
\usepackage{cmap}
\usepackage{fontspec}
\defaultfontfeatures[\rmfamily,\sffamily,\ttfamily]{}
\usepackage{amsmath,amssymb,amstext}
\usepackage{polyglossia}
\setmainlanguage{english}



\setmainfont{FreeSerif}[
  Extension      = .otf,
  UprightFont    = *,
  ItalicFont     = *Italic,
  BoldFont       = *Bold,
  BoldItalicFont = *BoldItalic
]
\setsansfont{FreeSans}[
  Extension      = .otf,
  UprightFont    = *,
  ItalicFont     = *Oblique,
  BoldFont       = *Bold,
  BoldItalicFont = *BoldOblique,
]
\setmonofont{FreeMono}[
  Extension      = .otf,
  UprightFont    = *,
  ItalicFont     = *Oblique,
  BoldFont       = *Bold,
  BoldItalicFont = *BoldOblique,
]



\usepackage[Bjarne]{fncychap}
\usepackage[,numfigreset=1,mathnumfig]{sphinx}

\fvset{fontsize=\small}
\usepackage{geometry}


% Include hyperref last.
\usepackage{hyperref}
% Fix anchor placement for figures with captions.
\usepackage{hypcap}% it must be loaded after hyperref.
% Set up styles of URL: it should be placed after hyperref.
\urlstyle{same}

\addto\captionsenglish{\renewcommand{\contentsname}{Contents:}}

\usepackage{sphinxmessages}
\setcounter{tocdepth}{0}



\title{Tyche}
\date{Dec 29, 2022}
\release{1.2}
\author{National Renewable Energy Laboratory}
\newcommand{\sphinxlogo}{\vbox{}}
\renewcommand{\releasename}{Release}
\makeindex
\begin{document}

\pagestyle{empty}
\sphinxmaketitle
\pagestyle{plain}
\sphinxtableofcontents
\pagestyle{normal}
\phantomsection\label{\detokenize{index::doc}}


\sphinxAtStartPar
Risk and uncertainty are core characteristics of research and development (R\&D) programs. Attempting to do what has not been done before will sometimes end in failure, just as it will sometimes lead to extraordinary success. The challenge is to identify an optimal mix of R\&D investments in pathways that provide the highest returns while reducing the costs of failure. The goal of the R\&D Pathway and Portfolio Analysis and Evaluation project is to develop systematic, scalable pathway and portfolio analysis and evaluation methodologies and tools that provide high value to the U.S. Department of Energy (DOE) and its Office of Energy Efficiency \& Renewable Energy (EERE). This work aims to inform decision\sphinxhyphen{}making across R\&D projects and programs by assisting analysts and decision makers to identify and evaluate, quantify and monitor, manage, document, and communicate energy technology R\&D pathway and portfolio risks and benefits. The project\sphinxhyphen{}level risks typically considered are technology cost and performance (e.g., efficiency and environmental impact), while the portfolio level risks generally include market factors (e.g., competitiveness and consumer preference).

\sphinxAtStartPar
{\hyperref[\detokenize{cheat-sheet:sec-quickstart}]{\sphinxcrossref{\DUrole{std,std-ref}{Quick Start Guide}}}} covers how to set up an R\&D decision context using Tyche by creating the necessary input datasets and writing the technology model. {\hyperref[\detokenize{example-technology:sec-techmodelexample}]{\sphinxcrossref{\DUrole{std,std-ref}{Technology Model Example}}}} provides a simple example of developing a technology model, and {\hyperref[\detokenize{example-analysis:sec-analysisexample}]{\sphinxcrossref{\DUrole{std,std-ref}{Analysis Example}}}} provides an analysis example of decision support analysis. High\sphinxhyphen{}level information on the approach behind Tyche is given in {\hyperref[\detokenize{approach:sec-approach}]{\sphinxcrossref{\DUrole{std,std-ref}{Approach}}}} and details on the mathematical formulation used to represent technologies and analyze investment impact is given in {\hyperref[\detokenize{formulation:sec-formulation}]{\sphinxcrossref{\DUrole{std,std-ref}{Mathematical Formulation}}}}. {\hyperref[\detokenize{optimizers:sec-optimizers}]{\sphinxcrossref{\DUrole{std,std-ref}{Optimization}}}} gives information on built\sphinxhyphen{}in optimization algorithms. The complete Python API for the Tyche codebase and the technology models provided with Tyche is in {\hyperref[\detokenize{modules:sec-modules}]{\sphinxcrossref{\DUrole{std,std-ref}{Python API}}}}.

\sphinxstepscope


\chapter{Quick Start Guide}
\label{\detokenize{cheat-sheet:quick-start-guide}}\label{\detokenize{cheat-sheet:sec-quickstart}}\label{\detokenize{cheat-sheet::doc}}
\sphinxAtStartPar
The purpose of this guide is to allow a new user to set up their first R\&D decision context using Tyche. An R\&D decision context involves one or more technologies that are subject to various R\&D investments with the goal of changing the technology metrics and outcomes.


\section{Introduction}
\label{\detokenize{cheat-sheet:introduction}}
\sphinxAtStartPar
The following materials walk through:
\begin{enumerate}
\sphinxsetlistlabels{\arabic}{enumi}{enumii}{}{.}%
\item {} 
\sphinxAtStartPar
What the Technology Characterization and Evaluation (Tyche) tool does and why this is of value to the user;

\item {} 
\sphinxAtStartPar
How to set up the Tyche software for local use, including downloading and installing Anaconda;

\item {} 
\sphinxAtStartPar
How to develop input datasets for a decision context;

\item {} 
\sphinxAtStartPar
How to develop technology models for a decision context.

\end{enumerate}


\subsection{What does Tyche do?}
\label{\detokenize{cheat-sheet:what-does-tyche-do}}
\sphinxAtStartPar
The Tyche tool provides a consistent and systematic methodology for evaluating alternative R\&D investments in a technology or technology system and for comparing the impacts of these investments on metrics and outcomes of interest. Tyche is intended to provide analytical support for funding decision\sphinxhyphen{}makers as they consider how to meet their overall goals with various R\&D investment strategies.

\sphinxAtStartPar
Tyche’s methodology for evaluating and comparing R\&D investments:
\begin{enumerate}
\sphinxsetlistlabels{\arabic}{enumi}{enumii}{}{.}%
\item {} 
\sphinxAtStartPar
Uses techno\sphinxhyphen{}economic models of the technology(ies) of interest;

\item {} 
\sphinxAtStartPar
Incorporates expert elicitation to get quantitative, probabilistic estimates of how the technology(ies) of interest might change with R\&D;

\item {} 
\sphinxAtStartPar
Provides both ensemble simulation and multi\sphinxhyphen{}objective stochastic optimization capabilities that enable users to identify R\&D investments with the greatest potential for accomplishing decision\sphinxhyphen{}maker goals, determine the potential overall improvement in the technology, determine the most promising avenue of R\&D for a technology, and more.

\end{enumerate}

\sphinxAtStartPar
For additional details on the mathematics and approach behind Tyche, see the {\hyperref[\detokenize{formulation:sec-formulation}]{\sphinxcrossref{\DUrole{std,std-ref}{Mathematical Formulation}}}} and {\hyperref[\detokenize{approach:sec-approach}]{\sphinxcrossref{\DUrole{std,std-ref}{Approach}}}} sections.


\subsection{What is a “technology”?}
\label{\detokenize{cheat-sheet:what-is-a-technology}}
\sphinxAtStartPar
In the R\&D decision contexts represented and analyzed by Tyche, “technology” has a very broad definition. A technology converts input(s) to output(s) using capital equipment with a defined lifetime, and incurs fixed and/or variable costs in doing so. A technology may be a manufacturing process, a biorefinery, an agricultural process, a renewable energy technology component such as a silicon wafer or an inverter, a renewable energy technology unit such as a wind turbine or solar panel, a renewable power plant system such as a concentrated solar power plant, and more. Within the R\&D decision context, a technology is also subject to one or more research areas in which R\&D investments can be made to change the technology and its economic, environmental, and other metrics of interest. Multiple technologies can be modeled and compared within the same decision context, provided the same metrics are calculable for each technology. Within Tyche, a technology is represented both physically and economically using a simple but generalized techno\sphinxhyphen{}economic analysis (TEA) model. The TEA is based on a user defined technology model and accompanying datasets of technological and investment information.


\section{Getting Started}
\label{\detokenize{cheat-sheet:getting-started}}
\sphinxAtStartPar
This section provides guidance on setting up Tyche for use on your local machine. Tyche is written in Python and requires a local Python installation to run. It is recommended to use Anaconda and conda for installing Python and managing Tyche’s prerequisite packages.


\subsection{Install Anaconda}
\label{\detokenize{cheat-sheet:install-anaconda}}\begin{itemize}
\item {} 
\sphinxAtStartPar
Download the Anaconda distribution for your system (Windows or MacOS) from the \sphinxhref{https://www.anaconda.com/products/distribution}{Anaconda Distributions} website.

\item {} 
\sphinxAtStartPar
Once downloaded, follow the instructions provided with the installer.

\end{itemize}


\subsection{Download the Tyche software}
\label{\detokenize{cheat-sheet:download-the-tyche-software}}\begin{itemize}
\item {} 
\sphinxAtStartPar
The latest stable release of the Tyche software can be downloaded as a .zip file from the \sphinxhref{https://github.com/NREL/tyche/releases}{GitHub repository}.

\item {} 
\sphinxAtStartPar
Extract the files to a location convenient to you. It may be easiest to access these files if they are located on your desktop, but this is not a requirement.

\end{itemize}


\subsection{Navigate the Tyche directory structure}
\label{\detokenize{cheat-sheet:navigate-the-tyche-directory-structure}}
\sphinxAtStartPar
Once downloaded and extracted, the Tyche files will have the directory structure shown in \hyperref[\detokenize{cheat-sheet:fig-directorystruct}]{Fig.\@ \ref{\detokenize{cheat-sheet:fig-directorystruct}}}.

\begin{figure}[htbp]
\centering
\capstart

\noindent\sphinxincludegraphics[width=400\sphinxpxdimen]{{image1}.png}
\caption{Tyche repository directory structure. New technology models and data should be saved in sub\sphinxhyphen{}directories under the technology directory, indicated in blue.}\label{\detokenize{cheat-sheet:id1}}\label{\detokenize{cheat-sheet:fig-directorystruct}}\end{figure}
\begin{itemize}
\item {} 
\sphinxAtStartPar
\sphinxstyleemphasis{conda} contains the environment specification file used to set up the Tyche environment.

\item {} 
\sphinxAtStartPar
\sphinxstyleemphasis{docs} contains reStructured Text (.rst) files used to generate the Tyche documentation. These files are for internal use only and should not be modified.

\item {} \begin{description}
\sphinxlineitem{\sphinxstyleemphasis{src} and its subdirectories contain the Tyche analysis codebase.}\begin{itemize}
\item {} 
\sphinxAtStartPar
\sphinxstyleemphasis{technology} contains a subdirectory containing the input datasets (.xlsx) and analysis Jupyter notebooks (.ipynb) for each decision context, as well as the technology model files (.py) for each decision context.

\item {} 
\sphinxAtStartPar
\sphinxstyleemphasis{tyche} contains the Python files which provide all of Tyche’s functionalities. These files are for internal use only and should not be modified.

\end{itemize}

\end{description}

\end{itemize}

\sphinxAtStartPar
Users creating decision contexts should store the new input datasets, analysis Jupyter notebooks, and technology model files in the technology directory, which is indicated in blue in \hyperref[\detokenize{cheat-sheet:fig-directorystruct}]{Fig.\@ \ref{\detokenize{cheat-sheet:fig-directorystruct}}}. It is strongly recommended that users create sub\sphinxhyphen{}directories for each new decision context, to avoid confusing input datasets and models between contexts.


\subsection{Set up the Tyche environment using conda}
\label{\detokenize{cheat-sheet:set-up-the-tyche-environment-using-conda}}
\sphinxAtStartPar
Tyche’s codebase comes with an environment specification file that is used with Conda to automatically install all of Tyche’s required Python packages. It is strongly recommended that users create and use the Tyche environment, to avoid any package conflicts or compatibility issues. It is also recommended that users turn off any VPN before following the steps in this section.
\begin{itemize}
\item {} 
\sphinxAtStartPar
On Windows, open an Anaconda Prompt (recommended) or Command Prompt window; on Mac, open a System Terminal window.

\item {} 
\sphinxAtStartPar
Change the current working directory to the location of the extracted Tyche files using \sphinxcode{\sphinxupquote{cd path/to/tyche/directory}}.

\item {} 
\sphinxAtStartPar
Then enter the following commands, pressing Enter after each line:

\end{itemize}

\begin{sphinxVerbatim}[commandchars=\\\{\}]
\PYG{n}{conda} \PYG{n}{env} \PYG{n}{create} \PYG{o}{\PYGZhy{}}\PYG{o}{\PYGZhy{}}\PYG{n}{file} \PYG{n}{conda}\PYGZbs{}\PYG{n}{tyche}\PYG{o}{.}\PYG{n}{yml}
\PYG{n}{conda} \PYG{n}{activate} \PYG{n}{tyche}
\end{sphinxVerbatim}

\sphinxAtStartPar
Note that the first command may take up to 10 minutes to execute. If the environment creation was successful, you should see a message similar to the following:

\begin{sphinxVerbatim}[commandchars=\\\{\}]
\PYG{n}{done}
\PYG{c+c1}{\PYGZsh{}}
\PYG{c+c1}{\PYGZsh{} To activate this environment, use}
\PYG{c+c1}{\PYGZsh{}}
\PYG{c+c1}{\PYGZsh{}     \PYGZdl{} conda activate tyche}
\PYG{c+c1}{\PYGZsh{}}
\PYG{c+c1}{\PYGZsh{} To deactivate an active environment, use}
\PYG{c+c1}{\PYGZsh{}}
\PYG{c+c1}{\PYGZsh{}     \PYGZdl{} conda deactivate}
\PYG{n}{Retrieving} \PYG{n}{notices}\PYG{p}{:} \PYG{o}{.}\PYG{o}{.}\PYG{o}{.}\PYG{n}{working}\PYG{o}{.}\PYG{o}{.}\PYG{o}{.} \PYG{n}{done}
\end{sphinxVerbatim}
\begin{itemize}
\item {} 
\sphinxAtStartPar
If you receive an HTTPS error during environment creation, consider retrying the command with the \sphinxtitleref{–insecure} flag added.

\item {} 
\sphinxAtStartPar
See the \sphinxhref{https://docs.conda.io/projects/conda/en/latest/user-guide/tasks/manage-environments.html\#creating-an-environment-from-an-environment-yml-file}{conda documentation} for additional information on installing and troubleshooting environments.

\end{itemize}


\subsection{Access Tyche analysis functions}
\label{\detokenize{cheat-sheet:access-tyche-analysis-functions}}
\sphinxAtStartPar
Using Tyche locally is generally done via \sphinxhref{https://jupyter.org/}{Jupyter Notebook}, several examples of which are packaged with the Tyche codebase. To open one of these provided notebooks or to create your own:
\begin{itemize}
\item {} 
\sphinxAtStartPar
Open an Anaconda Prompt window.

\item {} 
\sphinxAtStartPar
Activate the Tyche environment with \sphinxcode{\sphinxupquote{conda activate tyche}}.

\item {} 
\sphinxAtStartPar
Change the current working directory to the location of the extracted Tyche files using \sphinxcode{\sphinxupquote{cd path/to/tyche/directory}}.

\item {} 
\sphinxAtStartPar
Open the Jupyter Notebook browser interface with \sphinxcode{\sphinxupquote{jupyter notebook}}.

\end{itemize}

\sphinxAtStartPar
A browser window or new tab (if a window was already open) will then open and show the files within the Tyche directory, from which existing notebooks can be opened and run or new notebooks created.


\section{Defining a Decision Context}
\label{\detokenize{cheat-sheet:defining-a-decision-context}}
\sphinxAtStartPar
After Tyche and its prerequisites are installed, the user can begin assembling the input datasets and technology models necessary for running their own decision context analyses. This section provides details on the contents of each input dataset required by Tyche and on the structure and function of the technology model (.py) file.

\sphinxAtStartPar
Tyche contains built\sphinxhyphen{}in data validation checks that, once run, will provide a list of any data inconsistencies or apparent errors as well as the names of the datasets in which the inconsistencies were found. Users are encouraged to review the information here to create a first draft of their input datasets, and then rely on the validation checks for additional troubleshooting. Users may also find it helpful to begin developing their input datasets by altering and adding to one of the decision context datasets packaged with Tyche, rather than starting from scratch.

\sphinxAtStartPar
An example technology model is developed in the {\hyperref[\detokenize{example-technology:sec-techmodelexample}]{\sphinxcrossref{\DUrole{std,std-ref}{Technology Model Example}}}} section, and an example of using Tyche for decision support analysis is provided in the {\hyperref[\detokenize{example-analysis:sec-analysisexample}]{\sphinxcrossref{\DUrole{std,std-ref}{Analysis Example}}}} section.


\subsection{Technology Data and Model}
\label{\detokenize{cheat-sheet:technology-data-and-model}}

\subsubsection{Designs Dataset}
\label{\detokenize{cheat-sheet:designs-dataset}}
\sphinxAtStartPar
A “design” is a set of data representing the state of a technology that results from a specific R\&D investment scenario. The \sphinxstyleemphasis{designs} dataset contains information for all of the technologies being evaluated within a decision context. \sphinxstyleemphasis{designs} contains multiple sets of data for each technology: each set represents the technology state that results from a single R\&D investment scenario.  Multiple R\&D investment scenarios are typically represented, each corresponding to a different level of technology advancement, which is quantified probabilistically through expert elicitation. \hyperref[\detokenize{cheat-sheet:tbl-designsdict}]{Table \ref{\detokenize{cheat-sheet:tbl-designsdict}}} provides a data dictionary for the \sphinxstyleemphasis{designs} dataset.


\begin{savenotes}\sphinxattablestart
\centering
\sphinxcapstartof{table}
\sphinxthecaptionisattop
\sphinxcaption{Data dictionary for the \sphinxstyleemphasis{designs} dataset which defines various technology states resulting from R\&D investments.}\label{\detokenize{cheat-sheet:id2}}\label{\detokenize{cheat-sheet:tbl-designsdict}}
\sphinxaftertopcaption
\begin{tabular}[t]{|*{4}{\X{1}{4}|}}
\hline
\sphinxstyletheadfamily 
\sphinxAtStartPar
Column Name
&\sphinxstyletheadfamily 
\sphinxAtStartPar
Data Type
&\sphinxstyletheadfamily 
\sphinxAtStartPar
Allowed Values
&\sphinxstyletheadfamily 
\sphinxAtStartPar
Description
\\
\hline
\sphinxAtStartPar
Technology
&
\sphinxAtStartPar
String
&
\sphinxAtStartPar
Any
&
\sphinxAtStartPar
Name of the technology.
\\
\hline
\sphinxAtStartPar
Scenario
&
\sphinxAtStartPar
String
&
\sphinxAtStartPar
Any names are allowed. There must be at least two scenarios defined.
&
\sphinxAtStartPar
R\&D investment scenario that results in this technology design.
\\
\hline
\sphinxAtStartPar
Variable
&
\sphinxAtStartPar
String
&\begin{itemize}
\item {} 
\sphinxAtStartPar
Input

\item {} 
\sphinxAtStartPar
Input efficiency

\item {} 
\sphinxAtStartPar
Input price

\item {} 
\sphinxAtStartPar
Output efficiency

\item {} 
\sphinxAtStartPar
Output price

\item {} 
\sphinxAtStartPar
Lifetime

\item {} 
\sphinxAtStartPar
Scale

\end{itemize}
&
\sphinxAtStartPar
Variable types required by technology model and related functions.
\\
\hline
\sphinxAtStartPar
Index
&
\sphinxAtStartPar
String
&
\sphinxAtStartPar
Any
&
\sphinxAtStartPar
Name of the elements within each Variable.
\\
\hline
\sphinxAtStartPar
Value
&\begin{itemize}
\item {} 
\sphinxAtStartPar
Float

\item {} 
\sphinxAtStartPar
Distribution

\item {} 
\sphinxAtStartPar
Mixture of distributions

\end{itemize}
&\begin{itemize}
\item {} 
\sphinxAtStartPar
Set of real numbers

\item {} 
\sphinxAtStartPar
\sphinxstyleemphasis{scipy.stats} distributions

\item {} 
\sphinxAtStartPar
Mixture of \sphinxstyleemphasis{scipy.stats} distributions

\end{itemize}
&
\sphinxAtStartPar
Value for the R\&D investment scenario.
Example: st.triang(1,loc=5,scale=0.1)
\\
\hline
\sphinxAtStartPar
Units
&
\sphinxAtStartPar
String
&
\sphinxAtStartPar
Any
&
\sphinxAtStartPar
User defined units for Variables. Not used by Tyche.
\\
\hline
\sphinxAtStartPar
Notes
&
\sphinxAtStartPar
String
&
\sphinxAtStartPar
Any
&
\sphinxAtStartPar
Description provided by user. Not used by Tyche.
\\
\hline
\end{tabular}
\par
\sphinxattableend\end{savenotes}

\sphinxAtStartPar
\sphinxstylestrong{Mandatory data.} The Variable column within the \sphinxstyleemphasis{designs} dataset must contain all seven values defined in \hyperref[\detokenize{cheat-sheet:tbl-designsdict}]{Table \ref{\detokenize{cheat-sheet:tbl-designsdict}}}. If there are no elements within a Variable for the technology under study, the Variable must still be included in the \sphinxstyleemphasis{designs} dataset: leaving out any of the Variables in this dataset will result in the \sphinxstyleemphasis{designs} dataset failing the data validation checks. The Value for unneeded Variables may be set to 0 or 1, and the Index for unneeded Variables set to None. This may be necessary for technologies without any inputs: for instance, a solar panel could be modeled without any Inputs, if sunlight is not explicitly being modeled. In this case, the single Index defined for the Input Variable can be None, and the calculations within the technology model .py file can be defined without using this value. The mandatory Variables and their component Indexes are defined further in \hyperref[\detokenize{cheat-sheet:tbl-designsvars}]{Table \ref{\detokenize{cheat-sheet:tbl-designsvars}}}.


\begin{savenotes}\sphinxattablestart
\centering
\sphinxcapstartof{table}
\sphinxthecaptionisattop
\sphinxcaption{Mandatory values for Variables in the \sphinxstyleemphasis{designs} dataset.}\label{\detokenize{cheat-sheet:id3}}\label{\detokenize{cheat-sheet:tbl-designsvars}}
\sphinxaftertopcaption
\begin{tabulary}{\linewidth}[t]{|T|T|T|}
\hline
\sphinxstyletheadfamily 
\sphinxAtStartPar
Variable
&\sphinxstyletheadfamily 
\sphinxAtStartPar
Description
&\sphinxstyletheadfamily 
\sphinxAtStartPar
Index Description
\\
\hline
\sphinxAtStartPar
Input
&
\sphinxAtStartPar
Ideal input amounts that do not account for inefficiencies or losses.
&
\sphinxAtStartPar
Names of inputs to the technology.
\\
\hline
\sphinxAtStartPar
Input efficiency
&
\sphinxAtStartPar
Input inefficiencies or losses, expressed as a number between 0 and 1.
&
\sphinxAtStartPar
Names of inputs to the technology: every input with an amount must also have an efficiency value, even if the efficiency is 1.
\\
\hline
\sphinxAtStartPar
Input price
&
\sphinxAtStartPar
Purchase price for the input(s)
&
\sphinxAtStartPar
Names of inputs to the technology.
\\
\hline
\sphinxAtStartPar
Output efficiency
&
\sphinxAtStartPar
Output efficiencies or losses, expressed as a number between 0 and 1.
&
\sphinxAtStartPar
Names of outputs from the technology. Every output must have an efficiency value, even if the efficiency is 1.
\\
\hline
\sphinxAtStartPar
Output price
&
\sphinxAtStartPar
Sale price for the output(s).
&
\sphinxAtStartPar
Names of outputs from the technology. Every output must have a price, even if the price is irrelevant (in which case, set the price to 0).
\\
\hline
\sphinxAtStartPar
Lifetime
&
\sphinxAtStartPar
Time that a piece of capital spends in use; time it takes for a piece of capital’s value to depreciate to zero.
&
\sphinxAtStartPar
Names of the capital components of the technology.
\\
\hline
\sphinxAtStartPar
Scale
&
\sphinxAtStartPar
Scale at which the technology operates (one value for the technology).
&
\sphinxAtStartPar
No index.
\\
\hline
\end{tabulary}
\par
\sphinxattableend\end{savenotes}


\subsubsection{Parameters Dataset}
\label{\detokenize{cheat-sheet:parameters-dataset}}
\sphinxAtStartPar
The \sphinxstyleemphasis{parameters} dataset contains any additional technology\sphinxhyphen{}related data, other than that contained in the \sphinxstyleemphasis{designs} dataset, that is required to calculate a technology’s capital cost, fixed cost, production (actual output amounts), and metrics. (These calculations are implemented within the technology model .py file, discussed in the next section.) Identically to the \sphinxstyleemphasis{designs} dataset, the \sphinxstyleemphasis{parameters} dataset contains multiple sets of data corresponding to different R\&D investment scenarios. A data dictionary for the \sphinxstyleemphasis{parameters} dataset is given in \hyperref[\detokenize{cheat-sheet:tbl-paramsdict}]{Table \ref{\detokenize{cheat-sheet:tbl-paramsdict}}}.


\begin{savenotes}\sphinxattablestart
\centering
\sphinxcapstartof{table}
\sphinxthecaptionisattop
\sphinxcaption{Data dictionary for the \sphinxstyleemphasis{parameters} dataset, which, if necessary, provides additional technology\sphinxhyphen{}related data other than that in the \sphinxstyleemphasis{designs} dataset.}\label{\detokenize{cheat-sheet:id4}}\label{\detokenize{cheat-sheet:tbl-paramsdict}}
\sphinxaftertopcaption
\begin{tabulary}{\linewidth}[t]{|T|T|T|}
\hline
\sphinxstyletheadfamily 
\sphinxAtStartPar
Column Name
&\sphinxstyletheadfamily 
\sphinxAtStartPar
Data type
&\sphinxstyletheadfamily 
\sphinxAtStartPar
Description
\\
\hline
\sphinxAtStartPar
Technology
&
\sphinxAtStartPar
String
&
\sphinxAtStartPar
Name of the technology.
\\
\hline
\sphinxAtStartPar
Scenario
&
\sphinxAtStartPar
String
&
\sphinxAtStartPar
Name of the R\&D investment scenario that resulted in the corresponding parameter values or distributions.
\\
\hline
\sphinxAtStartPar
Parameter
&
\sphinxAtStartPar
String
&
\sphinxAtStartPar
Name of the parameter.
\\
\hline
\sphinxAtStartPar
Offset
&
\sphinxAtStartPar
Integer
&
\sphinxAtStartPar
Numerical location of the parameter in the parameter vector; begins at zero.
\\
\hline
\sphinxAtStartPar
Value
&
\sphinxAtStartPar
Float; Distribution; Mixture of distributions
&
\sphinxAtStartPar
Parameter value for the R\&D investment scenario. Example: st.triang(1,loc=5,scale=0.1)
\\
\hline
\sphinxAtStartPar
Units
&
\sphinxAtStartPar
String
&
\sphinxAtStartPar
Parameter units. User defined; not used or checked during Tyche calculations.
\\
\hline
\sphinxAtStartPar
Notes
&
\sphinxAtStartPar
String
&
\sphinxAtStartPar
Any additional information defined by the user. Not used during Tyche calculations.
\\
\hline
\end{tabulary}
\par
\sphinxattableend\end{savenotes}

\sphinxAtStartPar
Including the Offset value in the \sphinxstyleemphasis{parameters} dataset creates a user reference that makes it easier to access parameter values when defining the technology model.

\sphinxAtStartPar
\sphinxstylestrong{Mandatory data.} The \sphinxstyleemphasis{parameters} dataset is required to exist and to include at least one Parameter for every Technology\sphinxhyphen{}Scenario combination. If there are no Parameters present in the technology model, then the Parameter may be None and 0 may be entered under both the Offset and Value columns.


\subsubsection{Technology model (.py file)}
\label{\detokenize{cheat-sheet:technology-model-py-file}}
\sphinxAtStartPar
The technology model is a Python file (.py) which is user defined and contains methods for calculating capital cost, fixed cost, production (the actual output amount), and any metrics of interest, using the content of the \sphinxstyleemphasis{designs} and \sphinxstyleemphasis{parameters} datasets. \hyperref[\detokenize{cheat-sheet:tbl-techmethods}]{Table \ref{\detokenize{cheat-sheet:tbl-techmethods}}} describes methods that must be included in the technology model. Additional methods can be included in the technology model, if necessary. The names of the mandatory methods in \hyperref[\detokenize{cheat-sheet:tbl-techmethods}]{Table \ref{\detokenize{cheat-sheet:tbl-techmethods}}} are user\sphinxhyphen{}defined and must match the contents of the \sphinxstyleemphasis{functions} dataset, discussed below. The method parameters listed in \hyperref[\detokenize{cheat-sheet:tbl-techmethods}]{Table \ref{\detokenize{cheat-sheet:tbl-techmethods}}} are also fixed and cannot be changed. In the case that a method does not require all of the mandatory input parameters, they can simply be left out of the method’s calculations.


\begin{savenotes}\sphinxattablestart
\centering
\sphinxcapstartof{table}
\sphinxthecaptionisattop
\sphinxcaption{Methods required within the technology model Python file. Method names are user\sphinxhyphen{}defined and should match the contents of the functions dataset. Additional methods can be defined within the technology model as necessary.}\label{\detokenize{cheat-sheet:id5}}\label{\detokenize{cheat-sheet:tbl-techmethods}}
\sphinxaftertopcaption
\begin{tabulary}{\linewidth}[t]{|T|T|T|}
\hline
\sphinxstyletheadfamily 
\sphinxAtStartPar
Recommended Method Name
&\sphinxstyletheadfamily 
\sphinxAtStartPar
Parameters (method inputs)
&\sphinxstyletheadfamily 
\sphinxAtStartPar
Returns
\\
\hline
\sphinxAtStartPar
capital\_cost
&
\sphinxAtStartPar
scale, parameter
&
\sphinxAtStartPar
Capital cost(s) for each type of capital in the technology.
\\
\hline
\sphinxAtStartPar
fixed\_cost
&
\sphinxAtStartPar
scale, parameter
&
\sphinxAtStartPar
Annual fixed cost(s) of operating the technology.
\\
\hline
\sphinxAtStartPar
production
&
\sphinxAtStartPar
scale, capital, lifetime, fixed, input, parameter
&
\sphinxAtStartPar
Calculated actual (not ideal) output amount(s).
\\
\hline
\sphinxAtStartPar
metrics
&
\sphinxAtStartPar
scale, capital, lifetime, fixed, input\_raw, input, input\_price, output\_raw, output, cost, parameter
&
\sphinxAtStartPar
Calculated technology metric value(s).
\\
\hline
\end{tabulary}
\par
\sphinxattableend\end{savenotes}

\sphinxAtStartPar
The production method can access the actual input amount, which is the ideal or raw input amount value multiplied by the input efficiency value (both defined in the \sphinxstyleemphasis{designs} dataset). In contrast, the metrics method can access both the ideal input amount (input\_raw) and the actual input amount (input).

\sphinxAtStartPar
All return values for the required methods, even if only a single value is returned, must be formatted as \sphinxhref{https://numpy.org/doc/stable/reference/generated/numpy.stack.html}{Numpy stacks}.

\sphinxAtStartPar
Part of Tyche’s analysis capabilities rely on the ability to evaluate the impact of multiple R\&D investments across research areas. In order for the R\&D investment impacts to be combined, it is recommended that the return values for the \sphinxcode{\sphinxupquote{metrics}} method be represented as changes from a baseline value that represents the current state of technology. These changes can then be summed across R\&D investments to see the overall impact.


\subsection{Investment Datasets}
\label{\detokenize{cheat-sheet:investment-datasets}}
\sphinxAtStartPar
The previous sections provided information on the input datasets required to define the technology(ies) of interest within a decision context, and on the content and structure of the technology model itself. This section provides information on the input datasets that define R\&D investment options and the research categories in which investments can be made.


\subsubsection{Tranches Dataset}
\label{\detokenize{cheat-sheet:tranches-dataset}}
\sphinxAtStartPar
A Tranche is a discrete unit of R\&D investment (dollar amount) in a specific research category. Research categories are defined for each technology within a decision context and represent narrow topic areas in which R\&D investments are expected to result in technological improvements. Tranches within the same research category are mutually exclusive: one cannot simultaneously invest \$1M and \$5M in a research category. A Scenario is a combination of Tranches that represents one option for making R\&D investments.

\sphinxAtStartPar
The \sphinxstyleemphasis{tranches} dataset defines a set of R\&D investments across the research categories that are relevant to the technology under study. Tranches are combined into investment Scenarios – the same Scenarios found in the \sphinxstyleemphasis{designs} and \sphinxstyleemphasis{parameters} datasets. The impact of each Scenario on the technology is highly uncertain and is quantified probabilistically using expert elicitation. A data dictionary for the \sphinxstyleemphasis{tranches} dataset is given in \hyperref[\detokenize{cheat-sheet:tbl-tranchesdict}]{Table \ref{\detokenize{cheat-sheet:tbl-tranchesdict}}}.


\begin{savenotes}\sphinxattablestart
\centering
\sphinxcapstartof{table}
\sphinxthecaptionisattop
\sphinxcaption{Data dictionary for the \sphinxstyleemphasis{tranches} dataset.}\label{\detokenize{cheat-sheet:id6}}\label{\detokenize{cheat-sheet:tbl-tranchesdict}}
\sphinxaftertopcaption
\begin{tabulary}{\linewidth}[t]{|T|T|T|}
\hline
\sphinxstyletheadfamily 
\sphinxAtStartPar
Column Name
&\sphinxstyletheadfamily 
\sphinxAtStartPar
Data Type
&\sphinxstyletheadfamily 
\sphinxAtStartPar
Description
\\
\hline
\sphinxAtStartPar
Category
&
\sphinxAtStartPar
String
&
\sphinxAtStartPar
Names of the R\&D categories in which investment can be made to impact the technology or technologies being studied.
\\
\hline
\sphinxAtStartPar
Tranche
&
\sphinxAtStartPar
String
&
\sphinxAtStartPar
Names of the tranches.
\\
\hline
\sphinxAtStartPar
Scenario
&
\sphinxAtStartPar
String
&
\sphinxAtStartPar
Names of the R\&D investment scenarios, which combine tranches across R\&D categories. The names in this column must correspond to the Scenarios listed in the designs and parameters datasets.
\\
\hline
\sphinxAtStartPar
Amount
&
\sphinxAtStartPar
Float; Distribution; Mixture of distributions
&
\sphinxAtStartPar
The R\&D investment amount of the Tranche. The amount may be defined as a scalar, a probability distribution, or a mix of probability distributions.
\\
\hline
\sphinxAtStartPar
Notes
&
\sphinxAtStartPar
String
&
\sphinxAtStartPar
Additional user\sphinxhyphen{}defined information. Not used by Tyche.
\\
\hline
\end{tabulary}
\par
\sphinxattableend\end{savenotes}


\subsubsection{Investment Dataset}
\label{\detokenize{cheat-sheet:investment-dataset}}
\sphinxAtStartPar
An Investment, similar to a Scenario, is a combination of Tranches that represents a particular R\&D strategy.

\sphinxAtStartPar
The \sphinxstyleemphasis{investments} dataset provides a separate way to look at making R\&D investments. Combining individual tranches allows users to explore and optimize R\&D investment amounts, but it may be the case that there are specific strategies that users wish to explore, without optimizing. In this case, the \sphinxstyleemphasis{investments} dataset is used to define specific combinations of tranches that are of interest. A data dictionary for the \sphinxstyleemphasis{investments} dataset is given in \hyperref[\detokenize{cheat-sheet:tbl-investmentsdict}]{Table \ref{\detokenize{cheat-sheet:tbl-investmentsdict}}}.


\begin{savenotes}\sphinxattablestart
\centering
\sphinxcapstartof{table}
\sphinxthecaptionisattop
\sphinxcaption{Data dictionary for the \sphinxstyleemphasis{investments} dataset.}\label{\detokenize{cheat-sheet:id7}}\label{\detokenize{cheat-sheet:tbl-investmentsdict}}
\sphinxaftertopcaption
\begin{tabulary}{\linewidth}[t]{|T|T|T|}
\hline
\sphinxstyletheadfamily 
\sphinxAtStartPar
Column Name
&\sphinxstyletheadfamily 
\sphinxAtStartPar
Data Type
&\sphinxstyletheadfamily 
\sphinxAtStartPar
Description
\\
\hline
\sphinxAtStartPar
Investment
&
\sphinxAtStartPar
String
&
\sphinxAtStartPar
Name of the R\&D investment. Distinct from the Scenarios.
\\
\hline
\sphinxAtStartPar
Category
&
\sphinxAtStartPar
String
&
\sphinxAtStartPar
Names of the R\&D categories being invested in. Within each row, the Category must match the Tranche. The set of Categories in the \sphinxstyleemphasis{investments} dataset must match the set of Categories in the \sphinxstyleemphasis{tranches} dataset.
\\
\hline
\sphinxAtStartPar
Tranche
&
\sphinxAtStartPar
String
&
\sphinxAtStartPar
Names of the tranches within the Investment. Within each row, the Tranche must match the Category. The set of Tranches in the \sphinxstyleemphasis{investments} dataset must match the set of Tranches in the \sphinxstyleemphasis{tranches} dataset.
\\
\hline
\sphinxAtStartPar
Notes
&\sphinxstartmulticolumn{2}%
\begin{varwidth}[t]{\sphinxcolwidth{2}{3}}
\sphinxAtStartPar
String     Additional user\sphinxhyphen{}defined information. Not used by Tyche.
\par
\vskip-\baselineskip\vbox{\hbox{\strut}}\end{varwidth}%
\sphinxstopmulticolumn
\\
\hline
\end{tabulary}
\par
\sphinxattableend\end{savenotes}

\sphinxAtStartPar
\sphinxstylestrong{Relationship between Categories, Tranches, Scenarios, and Investments.} Both the \sphinxstyleemphasis{designs} and \sphinxstyleemphasis{parameters} dataset contain technology data under multiple Scenarios. Each Scenario represents the technological outcomes from one or more Tranches, and each Tranche represents a unit of R\&D investment in a single Category (or research area). Scenarios and their component Tranches are defined in the \sphinxstyleemphasis{tranches} dataset. Tranches can also be combined to form Investments, as defined in the \sphinxstyleemphasis{investments} dataset.


\subsection{Additional Datasets}
\label{\detokenize{cheat-sheet:additional-datasets}}

\subsubsection{Indices Dataset}
\label{\detokenize{cheat-sheet:indices-dataset}}
\sphinxAtStartPar
The \sphinxstyleemphasis{indices} dataset contains the numerical indexes (location within a list or array) used to access content in the other datasets. \hyperref[\detokenize{cheat-sheet:tbl-indicesdict}]{Table \ref{\detokenize{cheat-sheet:tbl-indicesdict}}} describes the columns required for the indices table. Numerical locations for parameters should not be listed in this dataset.


\begin{savenotes}\sphinxattablestart
\centering
\sphinxcapstartof{table}
\sphinxthecaptionisattop
\sphinxcaption{Data dictionary for the \sphinxstyleemphasis{indices} dataset.}\label{\detokenize{cheat-sheet:id8}}\label{\detokenize{cheat-sheet:tbl-indicesdict}}
\sphinxaftertopcaption
\begin{tabular}[t]{|*{4}{\X{1}{4}|}}
\hline

\sphinxAtStartPar
Column Name
&
\sphinxAtStartPar
Data Type
&
\sphinxAtStartPar
Allowed Values
&
\sphinxAtStartPar
Description
\\
\hline
\sphinxAtStartPar
Technology
&
\sphinxAtStartPar
String
&
\sphinxAtStartPar
Any
&
\sphinxAtStartPar
Name of the technology
\\
\hline
\sphinxAtStartPar
Type
&
\sphinxAtStartPar
String
&\begin{itemize}
\item {} 
\sphinxAtStartPar
Capital

\item {} 
\sphinxAtStartPar
Input

\item {} 
\sphinxAtStartPar
Output

\item {} 
\sphinxAtStartPar
Metric

\end{itemize}
&
\sphinxAtStartPar
Names of the Types defined within the designs dataset.
\\
\hline
\sphinxAtStartPar
Index
&
\sphinxAtStartPar
String
&
\sphinxAtStartPar
Any
&
\sphinxAtStartPar
Name of the elements within each Type. For instance, names of the Input types.
\\
\hline
\sphinxAtStartPar
Offset
&
\sphinxAtStartPar
Integer
&
\sphinxAtStartPar
>= 0
&
\sphinxAtStartPar
Numerical location of the Index within each Type.
\\
\hline
\sphinxAtStartPar
Description
&
\sphinxAtStartPar
String
&
\sphinxAtStartPar
Any
&
\sphinxAtStartPar
Additional user\sphinxhyphen{}defined information, such as units. Not used during Tyche calculations.
\\
\hline
\sphinxAtStartPar
Notes
&
\sphinxAtStartPar
String
&
\sphinxAtStartPar
Any
&
\sphinxAtStartPar
Additional user\sphinxhyphen{}defined information. Not used during Tyche calculations.
\\
\hline
\end{tabular}
\par
\sphinxattableend\end{savenotes}

\sphinxAtStartPar
\sphinxstylestrong{Relationship between *indices* and other datasets}. A technology in the Tyche context is quantified using five sets of attribute values and one technology\sphinxhyphen{}level attribute value. The five sets of attribute values are Capital, Input, Output, Parameter, and Metric, and the technology\sphinxhyphen{}level attribute is Scale. Elements within each of the five sets are defined with an Index which simply names the element (for instance, Electricity might be one of the Index values within the Input set). Elements of Capital have an associated Lifetime. Elements of the Input set have an associated ideal amount (also called Input), an Input efficiency value, and an Input price. Elements of the Output set have only an Output efficiency and an Output price; the ideal output amounts are calculated from the technology model. Elements of the Metric set are named with an Index and are likewise calculated from the technology model. Elements of the Parameter set have only a value. The \sphinxstyleemphasis{indices} dataset lists the elements of the Capital, Input, Output, and Metric sets, and contains an Offset column giving the numerical location of each element within its set. The \sphinxstyleemphasis{designs} dataset contains values for each element of the Capital, Input, Output, and Metric sets as well as the technology\sphinxhyphen{}level Scale value. The \sphinxstyleemphasis{parameters} dataset names and gives values for each element of the Parameter set.

\sphinxAtStartPar
\sphinxstylestrong{Mandatory data.} All four Types must be listed in the \sphinxstyleemphasis{indices} dataset. If a particular Type is not relevant to the technology under study, it still must be included in this dataset.


\subsubsection{Functions Dataset}
\label{\detokenize{cheat-sheet:functions-dataset}}
\sphinxAtStartPar
The \sphinxstyleemphasis{functions} dataset is used internally by Tyche to locate the technology model file and identify the four required methods listed in \hyperref[\detokenize{cheat-sheet:tbl-techmethods}]{Table \ref{\detokenize{cheat-sheet:tbl-techmethods}}}. \hyperref[\detokenize{cheat-sheet:tbl-functionsdict}]{Table \ref{\detokenize{cheat-sheet:tbl-functionsdict}}} provides a data dictionary for the \sphinxstyleemphasis{functions} dataset.


\begin{savenotes}\sphinxattablestart
\centering
\sphinxcapstartof{table}
\sphinxthecaptionisattop
\sphinxcaption{Data dictionary for the \sphinxstyleemphasis{functions} dataset.}\label{\detokenize{cheat-sheet:id9}}\label{\detokenize{cheat-sheet:tbl-functionsdict}}
\sphinxaftertopcaption
\begin{tabulary}{\linewidth}[t]{|T|T|T|T|}
\hline
\sphinxstyletheadfamily 
\sphinxAtStartPar
Column Name
&\sphinxstyletheadfamily 
\sphinxAtStartPar
Data Type
&\sphinxstyletheadfamily 
\sphinxAtStartPar
Allowed Values
&\sphinxstyletheadfamily 
\sphinxAtStartPar
Description
\\
\hline
\sphinxAtStartPar
Technology
&
\sphinxAtStartPar
String
&
\sphinxAtStartPar
Any
&
\sphinxAtStartPar
Name of the technology.
\\
\hline
\sphinxAtStartPar
Style
&
\sphinxAtStartPar
String
&
\sphinxAtStartPar
numpy
&
\sphinxAtStartPar
See below for explanation.
\\
\hline
\sphinxAtStartPar
Module
&
\sphinxAtStartPar
String
&
\sphinxAtStartPar
Any
&
\sphinxAtStartPar
Filename of the technology model Python file. Do not include the file extension.
\\
\hline
\sphinxAtStartPar
Capital
&
\sphinxAtStartPar
String
&
\sphinxAtStartPar
Any
&
\sphinxAtStartPar
Name of the method within the technology model Python file that returns the calculated capital cost.
\\
\hline
\sphinxAtStartPar
Fixed
&
\sphinxAtStartPar
String
&
\sphinxAtStartPar
Any
&
\sphinxAtStartPar
Name of the method within the technology model Python file that returns the calculated fixed cost.
\\
\hline
\sphinxAtStartPar
Production
&
\sphinxAtStartPar
String
&
\sphinxAtStartPar
Any
&
\sphinxAtStartPar
Name of the method within the technology model Python file that returns the calculated output amount.
\\
\hline
\sphinxAtStartPar
Metrics
&
\sphinxAtStartPar
String
&
\sphinxAtStartPar
Any
&
\sphinxAtStartPar
Name of the method within the technology model Python file that returns the calculated technology metrics.
\\
\hline
\sphinxAtStartPar
Notes
&
\sphinxAtStartPar
String
&
\sphinxAtStartPar
Any
&
\sphinxAtStartPar
Any information that the user needs to record can go here. Not used during Tyche calculations.
\\
\hline
\end{tabulary}
\par
\sphinxattableend\end{savenotes}

\sphinxAtStartPar
The Style should remain \sphinxtitleref{numpy} for all Tyche versions 1.x. This indicates that inputs and outputs from the methods within the technology model Python file are treated as arrays rather than higher\sphinxhyphen{}dimensional (i.e., tensor) structures.

\sphinxAtStartPar
If only one technology model is used within a decision context, then the \sphinxstyleemphasis{functions} dataset will contain a single row.


\subsubsection{Results Dataset}
\label{\detokenize{cheat-sheet:results-dataset}}
\sphinxAtStartPar
The \sphinxstyleemphasis{results} dataset lists the Tyche outcomes that are of interest within a decision context, organized into categories defined by the Variable column. This dataset is used internally by Tyche for organizing and labeling results tables for easier user comprehension. A data dictionary for the \sphinxstyleemphasis{results} dataset is given in \hyperref[\detokenize{cheat-sheet:tbl-resultsdict}]{Table \ref{\detokenize{cheat-sheet:tbl-resultsdict}}}.


\begin{savenotes}\sphinxattablestart
\centering
\sphinxcapstartof{table}
\sphinxthecaptionisattop
\sphinxcaption{Data dictionary for the \sphinxstyleemphasis{results} dataset.}\label{\detokenize{cheat-sheet:id10}}\label{\detokenize{cheat-sheet:tbl-resultsdict}}
\sphinxaftertopcaption
\begin{tabular}[t]{|*{4}{\X{1}{4}|}}
\hline

\sphinxAtStartPar
Column Name
&
\sphinxAtStartPar
Data Type
&
\sphinxAtStartPar
Allowed Values
&
\sphinxAtStartPar
Description
\\
\hline
\sphinxAtStartPar
Technology
&
\sphinxAtStartPar
String
&
\sphinxAtStartPar
Any
&
\sphinxAtStartPar
Name of the technology.
\\
\hline
\sphinxAtStartPar
Variable
&
\sphinxAtStartPar
String
&\begin{itemize}
\item {} 
\sphinxAtStartPar
Cost

\item {} 
\sphinxAtStartPar
Output

\item {} 
\sphinxAtStartPar
Metric

\end{itemize}
&
\sphinxAtStartPar
Specific technology outcomes calculated by Tyche.
\\
\hline
\sphinxAtStartPar
Index
&
\sphinxAtStartPar
String
&
\sphinxAtStartPar
Any
&
\sphinxAtStartPar
Names of the elements within each Variable.
\\
\hline
\sphinxAtStartPar
Units
&
\sphinxAtStartPar
String
&
\sphinxAtStartPar
Any
&
\sphinxAtStartPar
User\sphinxhyphen{}defined units of the Index values. Not used or checked during Tyche calculations.
\\
\hline
\sphinxAtStartPar
Notes
&
\sphinxAtStartPar
String
&
\sphinxAtStartPar
Any
&
\sphinxAtStartPar
Additional information defined by the user. Not used during Tyche calculations.
\\
\hline
\end{tabular}
\par
\sphinxattableend\end{savenotes}

\sphinxAtStartPar
The Variable Cost is a technology\sphinxhyphen{}wide lifetime cost, and as such may not be relevant within all decision contexts. The Index of Cost can be simply Cost. The sets of Index values for the Output and Metric Variables should match the Output and Metric sets in both the \sphinxstyleemphasis{designs} and the \sphinxstyleemphasis{indices} datasets.

\sphinxAtStartPar
\sphinxstylestrong{Mandatory data.} Every Index within the Cost, Output and Metric sets defined elsewhere in the input datasets should be included in the \sphinxstyleemphasis{results} dataset.


\section{Uncertainty in the Input Datasets}
\label{\detokenize{cheat-sheet:uncertainty-in-the-input-datasets}}
\sphinxAtStartPar
Tyche provides two general use cases for exploring the relationship between R\&D investments and technological changes, both of which rely on expert elicitation to quantify inherent uncertainty. In the first and likely more common use case, a user knows what the R\&D investment options are for a technology or set of technologies and is interested in determining what impact these investment options have on the technology(ies) in order to decide how to allocate an R\&D budget. In other words, in this use case the user already knows the contents of the \sphinxstyleemphasis{tranches} and \sphinxstyleemphasis{investments} datasets, which are deterministic (fixed), and uses expert elicitation to fill in key values in the \sphinxstyleemphasis{designs} and \sphinxstyleemphasis{parameters} datasets with probability distributions.

\sphinxAtStartPar
In the second use case, a user knows what technological changes must be achieved with R\&D investment and is interested in determining the investment amount that will be required to achieve these changes. In this case the user already knows the contents of the \sphinxstyleemphasis{designs} and \sphinxstyleemphasis{parameters} dataset, which are deterministic, and uses expert elicitation to fill in the investment amounts in the \sphinxstyleemphasis{tranches} dataset.

\sphinxAtStartPar
It is critical to note that these use cases are \sphinxstylestrong{mutually exclusive}. Tyche cannot be used to evaluate a scenario in which desired technological changes as well as the investment amounts are both uncertain. What this means for the user is that probability distributions, or mixtures of distributions, can be used to specify values either in the \sphinxstyleemphasis{designs} and \sphinxstyleemphasis{parameters} datasets or in the \sphinxstyleemphasis{tranches} dataset, but not both. If distributions are used in all three datasets, the code will break by design.


\subsection{Defining values as probability distributions and mixtures}
\label{\detokenize{cheat-sheet:defining-values-as-probability-distributions-and-mixtures}}
\sphinxAtStartPar
An uncertain value can be defined within a dataset using any of the built\sphinxhyphen{}in distributions of the \sphinxhref{https://docs.scipy.org/doc/scipy/reference/stats.html}{scipy.stats} package. A list of available distributions is provided at the hyperlink. Uncertain values can also be defined as a weighted average or mixture of probability distributions using the Tyche \sphinxstyleemphasis{mixture} method.

\sphinxstepscope


\chapter{Technology Model Example}
\label{\detokenize{example-technology:technology-model-example}}\label{\detokenize{example-technology:sec-techmodelexample}}\label{\detokenize{example-technology::doc}}
\sphinxAtStartPar
Here is a very simple model for electrolysis of water. We just have water, electricity, a catalyst, and some lab space. We choose the fundamental unit of operation to be moles of H$_{\text{2}}$:

\sphinxAtStartPar
     H$_{\text{2}}$O → H$_{\text{2}}$ + ½ O$_{\text{2}}$

\sphinxAtStartPar
For this example, we treat the catalyst as the capital that we use to transform inputs into outputs. Our inputs are water and electricity, and our outputs are oxygen and hydrogen. Our only fixed cost is the rent on the lab space at \$1000/year. Using our past experience with electrolysis technology as well as some historical data, we estimate that we’ll be able to produce 6650 mol/year of hydrogen and at this scale, our catalyst has a lifetime of about 3 years. The metrics we’d like to calculate for our electrolysis technology are cost, greenhouse gas (GHG) emissions, and jobs created. Based on this information, the \sphinxstyleemphasis{designs} dataset for the base case electrolysis technology is as shown in \hyperref[\detokenize{example-technology:tbl-electrolysisdesigns}]{Table \ref{\detokenize{example-technology:tbl-electrolysisdesigns}}}.


\begin{savenotes}\sphinxattablestart
\centering
\sphinxcapstartof{table}
\sphinxthecaptionisattop
\sphinxcaption{\sphinxstyleemphasis{designs} dataset for the base case (without any R\&D) of the simple electrolysis example technology.}\label{\detokenize{example-technology:id1}}\label{\detokenize{example-technology:tbl-electrolysisdesigns}}
\sphinxaftertopcaption
\begin{tabulary}{\linewidth}[t]{|T|T|T|T|T|T|T|}
\hline
\sphinxstyletheadfamily 
\sphinxAtStartPar
Technology
&\sphinxstyletheadfamily 
\sphinxAtStartPar
Scenario
&\sphinxstyletheadfamily 
\sphinxAtStartPar
Variable
&\sphinxstyletheadfamily 
\sphinxAtStartPar
Index
&\sphinxstyletheadfamily 
\sphinxAtStartPar
Value
&\sphinxstyletheadfamily 
\sphinxAtStartPar
Units
&\sphinxstyletheadfamily 
\sphinxAtStartPar
Notes
\\
\hline
\sphinxAtStartPar
Simple electrolysis
&
\sphinxAtStartPar
Base Electrolysis
&
\sphinxAtStartPar
Input
&
\sphinxAtStartPar
Water
&
\sphinxAtStartPar
19.04
&
\sphinxAtStartPar
g/mole
&\\
\hline
\sphinxAtStartPar
Simple electrolysis
&
\sphinxAtStartPar
Base Electrolysis
&
\sphinxAtStartPar
Input efficiency
&
\sphinxAtStartPar
Water
&
\sphinxAtStartPar
0.95
&
\sphinxAtStartPar
1
&
\sphinxAtStartPar
Due to mass transport loss on input.
\\
\hline
\sphinxAtStartPar
Simple electrolysis
&
\sphinxAtStartPar
Base Electrolysis
&
\sphinxAtStartPar
Input
&
\sphinxAtStartPar
Electricity
&
\sphinxAtStartPar
279
&
\sphinxAtStartPar
kJ/mole
&\\
\hline
\sphinxAtStartPar
Simple electrolysis
&
\sphinxAtStartPar
Base Electrolysis
&
\sphinxAtStartPar
Input efficiency
&
\sphinxAtStartPar
Electricity
&
\sphinxAtStartPar
0.85
&
\sphinxAtStartPar
l
&
\sphinxAtStartPar
Due to ohmic losses on input.
\\
\hline
\sphinxAtStartPar
Simple electrolysis
&
\sphinxAtStartPar
Base Electrolysis
&
\sphinxAtStartPar
Output efficiency
&
\sphinxAtStartPar
Oxygen
&
\sphinxAtStartPar
0.9
&
\sphinxAtStartPar
1
&
\sphinxAtStartPar
Due to mass transport loss on output.
\\
\hline
\sphinxAtStartPar
Simple electrolysis
&
\sphinxAtStartPar
Base Electrolysis
&
\sphinxAtStartPar
Output efficiency
&
\sphinxAtStartPar
Hydrogen
&
\sphinxAtStartPar
0.9
&
\sphinxAtStartPar
1
&
\sphinxAtStartPar
Due to mass transport loss on output.
\\
\hline
\sphinxAtStartPar
Simple electrolysis
&
\sphinxAtStartPar
Base Electrolysis
&
\sphinxAtStartPar
Lifetime
&
\sphinxAtStartPar
Catalyst
&
\sphinxAtStartPar
3
&
\sphinxAtStartPar
yr
&
\sphinxAtStartPar
Effective lifetime of Al\sphinxhyphen{}Ni catalyst.
\\
\hline
\sphinxAtStartPar
Simple electrolysis
&
\sphinxAtStartPar
Base Electrolysis
&
\sphinxAtStartPar
Scale
&
\sphinxAtStartPar
n/a
&
\sphinxAtStartPar
6650
&
\sphinxAtStartPar
mole/yr
&
\sphinxAtStartPar
Rough estimate for a 50W setup.
\\
\hline
\sphinxAtStartPar
Simple electrolysis
&
\sphinxAtStartPar
Base Electrolysis
&
\sphinxAtStartPar
Input price
&
\sphinxAtStartPar
Water
&
\sphinxAtStartPar
4.80E\sphinxhyphen{}03
&
\sphinxAtStartPar
USD/mole
&\\
\hline
\sphinxAtStartPar
Simple electrolysis
&
\sphinxAtStartPar
Base Electrolysis
&
\sphinxAtStartPar
Input price
&
\sphinxAtStartPar
Electricity
&
\sphinxAtStartPar
3.33E\sphinxhyphen{}05
&
\sphinxAtStartPar
USD/kJ
&\\
\hline
\sphinxAtStartPar
Simple electrolysis
&
\sphinxAtStartPar
Base Electrolysis
&
\sphinxAtStartPar
Output price
&
\sphinxAtStartPar
Oxygen
&
\sphinxAtStartPar
3.00E\sphinxhyphen{}03
&
\sphinxAtStartPar
USD/g
&\\
\hline
\sphinxAtStartPar
Simple electrolysis
&
\sphinxAtStartPar
Base Electrolysis
&
\sphinxAtStartPar
Output price
&
\sphinxAtStartPar
Hydrogen
&
\sphinxAtStartPar
1.00E\sphinxhyphen{}02
&
\sphinxAtStartPar
USD/g
&\\
\hline
\end{tabulary}
\par
\sphinxattableend\end{savenotes}

\sphinxAtStartPar
Note that this is not the only way to model the electrolysis technology. We could choose to purchase lab space and equipment instead of renting, in which case we would have more types of capital, each with a particular lifetime. We could treat the oxygen output from our technology as waste instead of a coproduct and remove it from the model entirely. We could operate at a different scale and perhaps change our fixed or capital costs by doing so. Depending on where we operate this technology, our input and output prices will likely change. The Tyche framework offers great flexibility in representing technologies and technology systems; it is unlikely that there will only be a single correct way to model a decision context.

\sphinxAtStartPar
A key quantity that is not included in the \sphinxstyleemphasis{designs} dataset is our fixed cost, rent for the lab space. This quantity is included in the \sphinxstyleemphasis{parameters} dataset in \hyperref[\detokenize{example-technology:tbl-electrolysisparams}]{Table \ref{\detokenize{example-technology:tbl-electrolysisparams}}}, along with the necessary data to calculate our metrics of interest (cost, GHG, jobs).


\begin{savenotes}\sphinxattablestart
\centering
\sphinxcapstartof{table}
\sphinxthecaptionisattop
\sphinxcaption{\sphinxstyleemphasis{parameters} dataset for the base case (without any R\&D) of the simple electrolysis example technology.}\label{\detokenize{example-technology:id2}}\label{\detokenize{example-technology:tbl-electrolysisparams}}
\sphinxaftertopcaption
\begin{tabulary}{\linewidth}[t]{|T|T|T|T|T|T|T|}
\hline
\sphinxstyletheadfamily 
\sphinxAtStartPar
Technology
&\sphinxstyletheadfamily 
\sphinxAtStartPar
Scenario
&\sphinxstyletheadfamily 
\sphinxAtStartPar
Parameter
&\sphinxstyletheadfamily 
\sphinxAtStartPar
Offset
&\sphinxstyletheadfamily 
\sphinxAtStartPar
Value
&\sphinxstyletheadfamily 
\sphinxAtStartPar
Units
&\sphinxstyletheadfamily 
\sphinxAtStartPar
Notes
\\
\hline
\sphinxAtStartPar
Simple electrolysis
&
\sphinxAtStartPar
Base Electrolysis
&
\sphinxAtStartPar
Oxygen production
&
\sphinxAtStartPar
0
&
\sphinxAtStartPar
16
&
\sphinxAtStartPar
g
&\\
\hline
\sphinxAtStartPar
Simple electrolysis
&
\sphinxAtStartPar
Base Electrolysis
&
\sphinxAtStartPar
Hydrogen production
&
\sphinxAtStartPar
1
&
\sphinxAtStartPar
2
&
\sphinxAtStartPar
g
&\\
\hline
\sphinxAtStartPar
Simple electrolysis
&
\sphinxAtStartPar
Base Electrolysis
&
\sphinxAtStartPar
Water consumption
&
\sphinxAtStartPar
2
&
\sphinxAtStartPar
18.08
&
\sphinxAtStartPar
g
&\\
\hline
\sphinxAtStartPar
Simple electrolysis
&
\sphinxAtStartPar
Base Electrolysis
&
\sphinxAtStartPar
Electricity consumption
&
\sphinxAtStartPar
3
&
\sphinxAtStartPar
237
&
\sphinxAtStartPar
kJ
&\\
\hline
\sphinxAtStartPar
Simple electrolysis
&
\sphinxAtStartPar
Base Electrolysis
&
\sphinxAtStartPar
Jobs
&
\sphinxAtStartPar
4
&
\sphinxAtStartPar
1.50E\sphinxhyphen{}04
&
\sphinxAtStartPar
job/mole
&\\
\hline
\sphinxAtStartPar
Simple electrolysis
&
\sphinxAtStartPar
Base Electrolysis
&
\sphinxAtStartPar
Reference scale
&
\sphinxAtStartPar
5
&
\sphinxAtStartPar
6650
&
\sphinxAtStartPar
mole/yr
&\\
\hline
\sphinxAtStartPar
Simple electrolysis
&
\sphinxAtStartPar
Base Electrolysis
&
\sphinxAtStartPar
Reference capital cost for catalyst
&
\sphinxAtStartPar
6
&
\sphinxAtStartPar
0.63
&
\sphinxAtStartPar
USD
&\\
\hline
\sphinxAtStartPar
Simple electrolysis
&
\sphinxAtStartPar
Base Electrolysis
&
\sphinxAtStartPar
Reference fixed cost for rent
&
\sphinxAtStartPar
7
&
\sphinxAtStartPar
1000
&
\sphinxAtStartPar
USD/yr
&\\
\hline
\sphinxAtStartPar
Simple electrolysis
&
\sphinxAtStartPar
Base Electrolysis
&
\sphinxAtStartPar
GHG factor for water
&
\sphinxAtStartPar
8
&
\sphinxAtStartPar
0.00108
&
\sphinxAtStartPar
gCO2e/g
&
\sphinxAtStartPar
based on 244,956 gallons = 1 Mg CO2e
\\
\hline
\sphinxAtStartPar
Simple electrolysis
&
\sphinxAtStartPar
Base Electrolysis
&
\sphinxAtStartPar
GHG factor for electricity
&
\sphinxAtStartPar
9
&
\sphinxAtStartPar
0.138
&
\sphinxAtStartPar
gCO2e/kJ
&
\sphinxAtStartPar
based on 1 kWh = 0.5 kg CO2e
\\
\hline
\end{tabulary}
\par
\sphinxattableend\end{savenotes}

\sphinxAtStartPar
Within our R\&D decision context, we’re interested in increasing the input and output efficiencies of this process so we can produce hydrogen as cheaply as possible. Experts could assess how much R\&D to increase the various efficiencies \(\eta\) would cost. They could also suggest different catalysts, adding alkali, or replacing the process with PEM.

\sphinxAtStartPar
The \sphinxcode{\sphinxupquote{indices}} table (see \hyperref[\detokenize{example-technology:tbl-indices}]{Table \ref{\detokenize{example-technology:tbl-indices}}}) simply describes the various
indices available for the variables. The \sphinxcode{\sphinxupquote{Offset}} column specifies the
memory location in the argument for the production and metric functions.


\begin{savenotes}\sphinxattablestart
\centering
\sphinxcapstartof{table}
\sphinxthecaptionisattop
\sphinxcaption{Example of the \sphinxstyleliteralintitle{\sphinxupquote{indices}} table.}\label{\detokenize{example-technology:id3}}\label{\detokenize{example-technology:tbl-indices}}
\sphinxaftertopcaption
\begin{tabulary}{\linewidth}[t]{|T|T|T|T|T|T|}
\hline
\sphinxstyletheadfamily 
\sphinxAtStartPar
Technology
&\sphinxstyletheadfamily 
\sphinxAtStartPar
Type
&\sphinxstyletheadfamily 
\sphinxAtStartPar
Index
&\sphinxstyletheadfamily 
\sphinxAtStartPar
Offset
&\sphinxstyletheadfamily 
\sphinxAtStartPar
Description
&\sphinxstyletheadfamily 
\sphinxAtStartPar
Notes
\\
\hline
\sphinxAtStartPar
Simple electrolysis
&
\sphinxAtStartPar
Capital
&
\sphinxAtStartPar
Catalyst
&
\sphinxAtStartPar
0
&
\sphinxAtStartPar
Catalyst
&\\
\hline
\sphinxAtStartPar
Simple electrolysis
&
\sphinxAtStartPar
Fixed
&
\sphinxAtStartPar
Rent
&
\sphinxAtStartPar
0
&
\sphinxAtStartPar
Rent
&\\
\hline
\sphinxAtStartPar
Simple electrolysis
&
\sphinxAtStartPar
Input
&
\sphinxAtStartPar
Water
&
\sphinxAtStartPar
0
&
\sphinxAtStartPar
Water
&\\
\hline
\sphinxAtStartPar
Simple electrolysis
&
\sphinxAtStartPar
Input
&
\sphinxAtStartPar
Electricity
&
\sphinxAtStartPar
1
&
\sphinxAtStartPar
Electricity
&\\
\hline
\sphinxAtStartPar
Simple electrolysis
&
\sphinxAtStartPar
Output
&
\sphinxAtStartPar
Oxygen
&
\sphinxAtStartPar
0
&
\sphinxAtStartPar
Oxygen
&\\
\hline
\sphinxAtStartPar
Simple electrolysis
&
\sphinxAtStartPar
Output
&
\sphinxAtStartPar
Hydrogen
&
\sphinxAtStartPar
1
&
\sphinxAtStartPar
Hydrogen
&\\
\hline
\sphinxAtStartPar
Simple electrolysis
&
\sphinxAtStartPar
Metric
&
\sphinxAtStartPar
Cost
&
\sphinxAtStartPar
0
&
\sphinxAtStartPar
Cost
&\\
\hline
\sphinxAtStartPar
Simple electrolysis
&
\sphinxAtStartPar
Metric
&
\sphinxAtStartPar
Jobs
&
\sphinxAtStartPar
1
&
\sphinxAtStartPar
Jobs
&\\
\hline
\sphinxAtStartPar
Simple electrolysis
&
\sphinxAtStartPar
Metric
&
\sphinxAtStartPar
GHG
&
\sphinxAtStartPar
2
&
\sphinxAtStartPar
GHGs
&\\
\hline
\end{tabulary}
\par
\sphinxattableend\end{savenotes}


\section{Production function (à la Leontief)}
\label{\detokenize{example-technology:production-function-a-la-leontief}}
\sphinxAtStartPar
\(P_\mathrm{oxygen} = \left( 16.00~\mathrm{g} \right) \cdot \min \left\{ \frac{I^*_\mathrm{water}}{18.08~\mathrm{g}}, \frac{I^*_\mathrm{electricity}}{237~\mathrm{kJ}} \right\}\)

\sphinxAtStartPar
\(P_\mathrm{hydrogen} = \left( 2.00~\mathrm{g} \right) \cdot \min \left\{ \frac{I^*_\mathrm{water}}{18.08~\mathrm{g}}, \frac{I^*_\mathrm{electricity}}{237~\mathrm{kJ}} \right\}\)


\section{Metric functions}
\label{\detokenize{example-technology:metric-functions}}
\sphinxAtStartPar
\(M_\mathrm{cost} = K / O_\mathrm{hydrogen}\)

\sphinxAtStartPar
\(M_\mathrm{GHG} = \left( \left( 0.00108~\mathrm{gCO2e/gH20} \right) I_\mathrm{water} + \left( 0.138~\mathrm{gCO2e/kJ} \right) I_\mathrm{electricity} \right) / O_\mathrm{hydrogen}\)

\sphinxAtStartPar
\(M_\mathrm{jobs} = \left( 0.00015~\mathrm{job/mole} \right) / O_\mathrm{hydrogen}\)


\section{Performance of current design.}
\label{\detokenize{example-technology:performance-of-current-design}}
\sphinxAtStartPar
\(K = 0.18~\mathrm{USD/mole}\) (i.e., not profitable since it is
positive)

\sphinxAtStartPar
\(O_\mathrm{oxygen} = 14~\mathrm{g/mole}\)

\sphinxAtStartPar
\(O_\mathrm{hydrogen} = 1.8~\mathrm{g/mole}\)

\sphinxAtStartPar
\(\mu_\mathrm{cost} = 0.102~\mathrm{USD/gH2}\)

\sphinxAtStartPar
\(\mu_\mathrm{GHG} = 21.4~\mathrm{gCO2e/gH2}\)

\sphinxAtStartPar
\(\mu_\mathrm{jobs} = 0.000083~\mathrm{job/gH2}\)


\section{Technology Model}
\label{\detokenize{example-technology:technology-model}}
\sphinxAtStartPar
Each technology design requires a Python file with a capital cost, a fixed cost, a production, and a metrics function. \hyperref[\detokenize{example-technology:lst-electrolysis}]{Listing \ref{\detokenize{example-technology:lst-electrolysis}}} shows these functions for the simple electrolysis example.
\sphinxSetupCaptionForVerbatim{Example technology\sphinxhyphen{}defining functions.}
\def\sphinxLiteralBlockLabel{\label{\detokenize{example-technology:lst-electrolysis}}}
\begin{sphinxVerbatim}[commandchars=\\\{\}]
\PYG{c+c1}{\PYGZsh{} Simple electrolysis.}


\PYG{c+c1}{\PYGZsh{} All of the computations must be vectorized, so use `numpy`.}
\PYG{k+kn}{import} \PYG{n+nn}{numpy} \PYG{k}{as} \PYG{n+nn}{np}


\PYG{c+c1}{\PYGZsh{} Capital\PYGZhy{}cost function.}
\PYG{k}{def} \PYG{n+nf}{capital\PYGZus{}cost}\PYG{p}{(}
  \PYG{n}{scale}\PYG{p}{,}
  \PYG{n}{parameter}
\PYG{p}{)}\PYG{p}{:}

  \PYG{c+c1}{\PYGZsh{} Scale the reference values.}
  \PYG{k}{return} \PYG{n}{np}\PYG{o}{.}\PYG{n}{stack}\PYG{p}{(}\PYG{p}{[}\PYG{n}{np}\PYG{o}{.}\PYG{n}{multiply}\PYG{p}{(}
    \PYG{n}{parameter}\PYG{p}{[}\PYG{l+m+mi}{6}\PYG{p}{]}\PYG{p}{,} \PYG{n}{np}\PYG{o}{.}\PYG{n}{divide}\PYG{p}{(}\PYG{n}{scale}\PYG{p}{,} \PYG{n}{parameter}\PYG{p}{[}\PYG{l+m+mi}{5}\PYG{p}{]}\PYG{p}{)}
  \PYG{p}{)}\PYG{p}{]}\PYG{p}{)}


\PYG{c+c1}{\PYGZsh{} Fixed\PYGZhy{}cost function.}
\PYG{k}{def} \PYG{n+nf}{fixed\PYGZus{}cost}\PYG{p}{(}
  \PYG{n}{scale}\PYG{p}{,}
  \PYG{n}{parameter}
\PYG{p}{)}\PYG{p}{:}

  \PYG{c+c1}{\PYGZsh{} Scale the reference values.}
  \PYG{k}{return} \PYG{n}{np}\PYG{o}{.}\PYG{n}{stack}\PYG{p}{(}\PYG{p}{[}\PYG{n}{np}\PYG{o}{.}\PYG{n}{multiply}\PYG{p}{(}
    \PYG{n}{parameter}\PYG{p}{[}\PYG{l+m+mi}{7}\PYG{p}{]}\PYG{p}{,}
    \PYG{n}{np}\PYG{o}{.}\PYG{n}{divide}\PYG{p}{(}\PYG{n}{scale}\PYG{p}{,} \PYG{n}{parameter}\PYG{p}{[}\PYG{l+m+mi}{5}\PYG{p}{]}\PYG{p}{)}
  \PYG{p}{)}\PYG{p}{]}\PYG{p}{)}


\PYG{c+c1}{\PYGZsh{} Production function.}
\PYG{k}{def} \PYG{n+nf}{production}\PYG{p}{(}
  \PYG{n}{capital}\PYG{p}{,}
  \PYG{n}{fixed}\PYG{p}{,}
  \PYG{n+nb}{input}\PYG{p}{,}
  \PYG{n}{parameter}
\PYG{p}{)}\PYG{p}{:}

  \PYG{c+c1}{\PYGZsh{} Moles of input.}
  \PYG{n}{water}       \PYG{o}{=} \PYG{n}{np}\PYG{o}{.}\PYG{n}{divide}\PYG{p}{(}\PYG{n+nb}{input}\PYG{p}{[}\PYG{l+m+mi}{0}\PYG{p}{]}\PYG{p}{,} \PYG{n}{parameter}\PYG{p}{[}\PYG{l+m+mi}{2}\PYG{p}{]}\PYG{p}{)}
  \PYG{n}{electricity} \PYG{o}{=} \PYG{n}{np}\PYG{o}{.}\PYG{n}{divide}\PYG{p}{(}\PYG{n+nb}{input}\PYG{p}{[}\PYG{l+m+mi}{1}\PYG{p}{]}\PYG{p}{,} \PYG{n}{parameter}\PYG{p}{[}\PYG{l+m+mi}{3}\PYG{p}{]}\PYG{p}{)}

  \PYG{c+c1}{\PYGZsh{} Moles of output.}
  \PYG{n}{output} \PYG{o}{=} \PYG{n}{np}\PYG{o}{.}\PYG{n}{minimum}\PYG{p}{(}\PYG{n}{water}\PYG{p}{,} \PYG{n}{electricity}\PYG{p}{)}

  \PYG{c+c1}{\PYGZsh{} Grams of output.}
  \PYG{n}{oxygen}   \PYG{o}{=} \PYG{n}{np}\PYG{o}{.}\PYG{n}{multiply}\PYG{p}{(}\PYG{n}{output}\PYG{p}{,} \PYG{n}{parameter}\PYG{p}{[}\PYG{l+m+mi}{0}\PYG{p}{]}\PYG{p}{)}
  \PYG{n}{hydrogen} \PYG{o}{=} \PYG{n}{np}\PYG{o}{.}\PYG{n}{multiply}\PYG{p}{(}\PYG{n}{output}\PYG{p}{,} \PYG{n}{parameter}\PYG{p}{[}\PYG{l+m+mi}{1}\PYG{p}{]}\PYG{p}{)}

  \PYG{c+c1}{\PYGZsh{} Package results.}
  \PYG{k}{return} \PYG{n}{np}\PYG{o}{.}\PYG{n}{stack}\PYG{p}{(}\PYG{p}{[}\PYG{n}{oxygen}\PYG{p}{,} \PYG{n}{hydrogen}\PYG{p}{]}\PYG{p}{)}


\PYG{c+c1}{\PYGZsh{} Metrics function.}
\PYG{k}{def} \PYG{n+nf}{metrics}\PYG{p}{(}
  \PYG{n}{capital}\PYG{p}{,}
  \PYG{n}{fixed}\PYG{p}{,}
  \PYG{n}{input\PYGZus{}raw}\PYG{p}{,}
  \PYG{n+nb}{input}\PYG{p}{,}
  \PYG{n}{img}\PYG{o}{/}\PYG{n}{output\PYGZus{}raw}\PYG{p}{,}
  \PYG{n}{output}\PYG{p}{,}
  \PYG{n}{cost}\PYG{p}{,}
  \PYG{n}{parameter}
\PYG{p}{)}\PYG{p}{:}

  \PYG{c+c1}{\PYGZsh{} Hydrogen output.}
  \PYG{n}{hydrogen} \PYG{o}{=} \PYG{n}{output}\PYG{p}{[}\PYG{l+m+mi}{1}\PYG{p}{]}

  \PYG{c+c1}{\PYGZsh{} Cost of hydrogen.}
  \PYG{n}{cost1} \PYG{o}{=} \PYG{n}{np}\PYG{o}{.}\PYG{n}{divide}\PYG{p}{(}\PYG{n}{cost}\PYG{p}{,} \PYG{n}{hydrogen}\PYG{p}{)}

  \PYG{c+c1}{\PYGZsh{} Jobs normalized to hydrogen.}
  \PYG{n}{jobs} \PYG{o}{=} \PYG{n}{np}\PYG{o}{.}\PYG{n}{divide}\PYG{p}{(}\PYG{n}{parameter}\PYG{p}{[}\PYG{l+m+mi}{4}\PYG{p}{]}\PYG{p}{,} \PYG{n}{hydrogen}\PYG{p}{)}

  \PYG{c+c1}{\PYGZsh{} GHGs associated with water and electricity.}
  \PYG{n}{water}       \PYG{o}{=} \PYG{n}{np}\PYG{o}{.}\PYG{n}{multiply}\PYG{p}{(}\PYG{n}{input\PYGZus{}raw}\PYG{p}{[}\PYG{l+m+mi}{0}\PYG{p}{]}\PYG{p}{,} \PYG{n}{parameter}\PYG{p}{[}\PYG{l+m+mi}{8}\PYG{p}{]}\PYG{p}{)}
  \PYG{n}{electricity} \PYG{o}{=} \PYG{n}{np}\PYG{o}{.}\PYG{n}{multiply}\PYG{p}{(}\PYG{n}{input\PYGZus{}raw}\PYG{p}{[}\PYG{l+m+mi}{1}\PYG{p}{]}\PYG{p}{,} \PYG{n}{parameter}\PYG{p}{[}\PYG{l+m+mi}{9}\PYG{p}{]}\PYG{p}{)}
  \PYG{n}{co2e} \PYG{o}{=} \PYG{n}{np}\PYG{o}{.}\PYG{n}{divide}\PYG{p}{(}\PYG{n}{np}\PYG{o}{.}\PYG{n}{add}\PYG{p}{(}\PYG{n}{water}\PYG{p}{,} \PYG{n}{electricity}\PYG{p}{)}\PYG{p}{,} \PYG{n}{hydrogen}\PYG{p}{)}

  \PYG{c+c1}{\PYGZsh{} Package results.}
  \PYG{k}{return} \PYG{n}{np}\PYG{o}{.}\PYG{n}{stack}\PYG{p}{(}\PYG{p}{[}\PYG{n}{cost1}\PYG{p}{,} \PYG{n}{jobs}\PYG{p}{,} \PYG{n}{co2e}\PYG{p}{]}\PYG{p}{)}
\end{sphinxVerbatim}

\sphinxstepscope


\chapter{Analysis Example}
\label{\detokenize{example-analysis:analysis-example}}\label{\detokenize{example-analysis:sec-analysisexample}}\label{\detokenize{example-analysis::doc}}
\sphinxAtStartPar
Multiple Objectives for Residential PV.


\section{Import packages.}
\label{\detokenize{example-analysis:import-packages}}
\begin{sphinxVerbatim}[commandchars=\\\{\}]
\PYG{k+kn}{import} \PYG{n+nn}{os}
\PYG{k+kn}{import} \PYG{n+nn}{sys}
\PYG{n}{sys}\PYG{o}{.}\PYG{n}{path}\PYG{o}{.}\PYG{n}{insert}\PYG{p}{(}\PYG{l+m+mi}{0}\PYG{p}{,} \PYG{n}{os}\PYG{o}{.}\PYG{n}{path}\PYG{o}{.}\PYG{n}{abspath}\PYG{p}{(}\PYG{l+s+s2}{\PYGZdq{}}\PYG{l+s+s2}{../src}\PYG{l+s+s2}{\PYGZdq{}}\PYG{p}{)}\PYG{p}{)}
\end{sphinxVerbatim}

\begin{sphinxVerbatim}[commandchars=\\\{\}]
\PYG{k+kn}{import} \PYG{n+nn}{numpy}             \PYG{k}{as} \PYG{n+nn}{np}
\PYG{k+kn}{import} \PYG{n+nn}{matplotlib}\PYG{n+nn}{.}\PYG{n+nn}{pyplot} \PYG{k}{as} \PYG{n+nn}{pl}
\PYG{k+kn}{import} \PYG{n+nn}{pandas}            \PYG{k}{as} \PYG{n+nn}{pd}
\PYG{k+kn}{import} \PYG{n+nn}{seaborn}           \PYG{k}{as} \PYG{n+nn}{sb}
\PYG{k+kn}{import} \PYG{n+nn}{tyche}             \PYG{k}{as} \PYG{n+nn}{ty}

\PYG{k+kn}{from} \PYG{n+nn}{copy}            \PYG{k+kn}{import} \PYG{n}{deepcopy}
\PYG{k+kn}{from} \PYG{n+nn}{IPython}\PYG{n+nn}{.}\PYG{n+nn}{display} \PYG{k+kn}{import} \PYG{n}{Image}
\end{sphinxVerbatim}


\section{Load data.}
\label{\detokenize{example-analysis:load-data}}
\sphinxAtStartPar
The data should be stored in a set of comma\sphinxhyphen{}separated value files in a sub\sphinxhyphen{}directory of the technology folder, as shown in \hyperref[\detokenize{cheat-sheet:fig-directorystruct}]{the directory structure diagram (Fig.\@ \ref{\detokenize{cheat-sheet:fig-directorystruct}})}

\begin{sphinxVerbatim}[commandchars=\\\{\}]
\PYG{n}{designs} \PYG{o}{=} \PYG{n}{ty}\PYG{o}{.}\PYG{n}{Designs}\PYG{p}{(}\PYG{l+s+s2}{\PYGZdq{}}\PYG{l+s+s2}{data/pv\PYGZus{}residential\PYGZus{}simple}\PYG{l+s+s2}{\PYGZdq{}}\PYG{p}{)}
\end{sphinxVerbatim}

\begin{sphinxVerbatim}[commandchars=\\\{\}]
\PYG{n}{investments} \PYG{o}{=} \PYG{n}{ty}\PYG{o}{.}\PYG{n}{Investments}\PYG{p}{(}\PYG{l+s+s2}{\PYGZdq{}}\PYG{l+s+s2}{data/pv\PYGZus{}residential\PYGZus{}simple}\PYG{l+s+s2}{\PYGZdq{}}\PYG{p}{)}
\end{sphinxVerbatim}

\sphinxAtStartPar
Compile the production and metric functions for each technology in the dataset.

\begin{sphinxVerbatim}[commandchars=\\\{\}]
\PYG{n}{designs}\PYG{o}{.}\PYG{n}{compile}\PYG{p}{(}\PYG{p}{)}
\end{sphinxVerbatim}


\section{Examine the data.}
\label{\detokenize{example-analysis:examine-the-data}}
\sphinxAtStartPar
The \sphinxcode{\sphinxupquote{functions}} table specifies where the Python code for each technology resides.

\begin{sphinxVerbatim}[commandchars=\\\{\}]
\PYG{n}{designs}\PYG{o}{.}\PYG{n}{functions}
\end{sphinxVerbatim}



\sphinxAtStartPar
Right now, only the style \sphinxcode{\sphinxupquote{numpy}} is supported.

\sphinxAtStartPar
The \sphinxcode{\sphinxupquote{indices}} table defines the subscripts for variables.

\begin{sphinxVerbatim}[commandchars=\\\{\}]
\PYG{n}{designs}\PYG{o}{.}\PYG{n}{indices}
\end{sphinxVerbatim}



\sphinxAtStartPar
The \sphinxcode{\sphinxupquote{designs}} table contains the cost, input, efficiency, and price data for a scenario.

\begin{sphinxVerbatim}[commandchars=\\\{\}]
\PYG{n}{designs}\PYG{o}{.}\PYG{n}{designs}
\end{sphinxVerbatim}



\sphinxAtStartPar
The \sphinxcode{\sphinxupquote{parameters}} table contains additional techno\sphinxhyphen{}economic parameters for each technology.

\begin{sphinxVerbatim}[commandchars=\\\{\}]
\PYG{n}{designs}\PYG{o}{.}\PYG{n}{parameters}
\end{sphinxVerbatim}



\sphinxAtStartPar
The \sphinxcode{\sphinxupquote{results}} table specifies the units of measure for results of computations.

\begin{sphinxVerbatim}[commandchars=\\\{\}]
\PYG{n}{designs}\PYG{o}{.}\PYG{n}{results}
\end{sphinxVerbatim}



\sphinxAtStartPar
The \sphinxcode{\sphinxupquote{tranches}} table specifies multually exclusive possibilities for investments: only one \sphinxcode{\sphinxupquote{Tranch}} may be selected for each \sphinxcode{\sphinxupquote{Category}}.

\begin{sphinxVerbatim}[commandchars=\\\{\}]
\PYG{n}{investments}\PYG{o}{.}\PYG{n}{tranches}
\end{sphinxVerbatim}



\sphinxAtStartPar
The \sphinxcode{\sphinxupquote{investments}} table bundles a consistent set of tranches (one per category) into an overall investment.

\begin{sphinxVerbatim}[commandchars=\\\{\}]
\PYG{n}{investments}\PYG{o}{.}\PYG{n}{investments}
\end{sphinxVerbatim}




\section{Evaluate the scenarios in the dataset.}
\label{\detokenize{example-analysis:evaluate-the-scenarios-in-the-dataset}}
\begin{sphinxVerbatim}[commandchars=\\\{\}]
\PYG{n}{scenario\PYGZus{}results} \PYG{o}{=} \PYG{n}{designs}\PYG{o}{.}\PYG{n}{evaluate\PYGZus{}scenarios}\PYG{p}{(}\PYG{n}{sample\PYGZus{}count}\PYG{o}{=}\PYG{l+m+mi}{50}\PYG{p}{)}
\end{sphinxVerbatim}

\begin{sphinxVerbatim}[commandchars=\\\{\}]
\PYG{n}{scenario\PYGZus{}results}\PYG{o}{.}\PYG{n}{xs}\PYG{p}{(}\PYG{l+m+mi}{1}\PYG{p}{,} \PYG{n}{level}\PYG{o}{=}\PYG{l+s+s2}{\PYGZdq{}}\PYG{l+s+s2}{Sample}\PYG{l+s+s2}{\PYGZdq{}}\PYG{p}{,} \PYG{n}{drop\PYGZus{}level}\PYG{o}{=}\PYG{k+kc}{False}\PYG{p}{)}
\end{sphinxVerbatim}




\subsection{Save results.}
\label{\detokenize{example-analysis:save-results}}
\begin{sphinxVerbatim}[commandchars=\\\{\}]
\PYG{n}{scenario\PYGZus{}results}\PYG{o}{.}\PYG{n}{to\PYGZus{}csv}\PYG{p}{(}\PYG{l+s+s2}{\PYGZdq{}}\PYG{l+s+s2}{output/pv\PYGZus{}residential\PYGZus{}simple/example\PYGZhy{}scenario.csv}\PYG{l+s+s2}{\PYGZdq{}}\PYG{p}{)}
\end{sphinxVerbatim}


\subsection{Plot GHG metric.}
\label{\detokenize{example-analysis:plot-ghg-metric}}
\begin{sphinxVerbatim}[commandchars=\\\{\}]
\PYG{n}{g} \PYG{o}{=} \PYG{n}{sb}\PYG{o}{.}\PYG{n}{boxplot}\PYG{p}{(}
    \PYG{n}{x}\PYG{o}{=}\PYG{l+s+s2}{\PYGZdq{}}\PYG{l+s+s2}{Scenario}\PYG{l+s+s2}{\PYGZdq{}}\PYG{p}{,}
    \PYG{n}{y}\PYG{o}{=}\PYG{l+s+s2}{\PYGZdq{}}\PYG{l+s+s2}{Value}\PYG{l+s+s2}{\PYGZdq{}}\PYG{p}{,}
    \PYG{n}{data}\PYG{o}{=}\PYG{n}{scenario\PYGZus{}results}\PYG{o}{.}\PYG{n}{xs}\PYG{p}{(}
        \PYG{p}{[}\PYG{l+s+s2}{\PYGZdq{}}\PYG{l+s+s2}{Metric}\PYG{l+s+s2}{\PYGZdq{}}\PYG{p}{,} \PYG{l+s+s2}{\PYGZdq{}}\PYG{l+s+s2}{GHG}\PYG{l+s+s2}{\PYGZdq{}}\PYG{p}{]}\PYG{p}{,}
        \PYG{n}{level}\PYG{o}{=}\PYG{p}{[}\PYG{l+s+s2}{\PYGZdq{}}\PYG{l+s+s2}{Variable}\PYG{l+s+s2}{\PYGZdq{}}\PYG{p}{,} \PYG{l+s+s2}{\PYGZdq{}}\PYG{l+s+s2}{Index}\PYG{l+s+s2}{\PYGZdq{}}\PYG{p}{]}
    \PYG{p}{)}\PYG{o}{.}\PYG{n}{reset\PYGZus{}index}\PYG{p}{(}\PYG{p}{)}\PYG{p}{[}\PYG{p}{[}\PYG{l+s+s2}{\PYGZdq{}}\PYG{l+s+s2}{Scenario}\PYG{l+s+s2}{\PYGZdq{}}\PYG{p}{,} \PYG{l+s+s2}{\PYGZdq{}}\PYG{l+s+s2}{Value}\PYG{l+s+s2}{\PYGZdq{}}\PYG{p}{]}\PYG{p}{]}\PYG{p}{,}
    \PYG{n}{order}\PYG{o}{=}\PYG{p}{[}
        \PYG{l+s+s2}{\PYGZdq{}}\PYG{l+s+s2}{2015 Actual}\PYG{l+s+s2}{\PYGZdq{}}              \PYG{p}{,}
        \PYG{l+s+s2}{\PYGZdq{}}\PYG{l+s+s2}{Module Slow Progress}\PYG{l+s+s2}{\PYGZdq{}}      \PYG{p}{,}
        \PYG{l+s+s2}{\PYGZdq{}}\PYG{l+s+s2}{Module Moderate Progress}\PYG{l+s+s2}{\PYGZdq{}}  \PYG{p}{,}
        \PYG{l+s+s2}{\PYGZdq{}}\PYG{l+s+s2}{Module Fast Progress}\PYG{l+s+s2}{\PYGZdq{}}      \PYG{p}{,}
        \PYG{l+s+s2}{\PYGZdq{}}\PYG{l+s+s2}{Inverter Slow Progress}\PYG{l+s+s2}{\PYGZdq{}}    \PYG{p}{,}
        \PYG{l+s+s2}{\PYGZdq{}}\PYG{l+s+s2}{Inverter Moderate Progress}\PYG{l+s+s2}{\PYGZdq{}}\PYG{p}{,}
        \PYG{l+s+s2}{\PYGZdq{}}\PYG{l+s+s2}{Inverter Fast Progress}\PYG{l+s+s2}{\PYGZdq{}}    \PYG{p}{,}
        \PYG{l+s+s2}{\PYGZdq{}}\PYG{l+s+s2}{BoS Slow Progress}\PYG{l+s+s2}{\PYGZdq{}}         \PYG{p}{,}
        \PYG{l+s+s2}{\PYGZdq{}}\PYG{l+s+s2}{BoS Moderate Progress}\PYG{l+s+s2}{\PYGZdq{}}     \PYG{p}{,}
        \PYG{l+s+s2}{\PYGZdq{}}\PYG{l+s+s2}{BoS Fast Progress}\PYG{l+s+s2}{\PYGZdq{}}         \PYG{p}{,}
    \PYG{p}{]}
\PYG{p}{)}
\PYG{n}{g}\PYG{o}{.}\PYG{n}{set}\PYG{p}{(}\PYG{n}{ylabel}\PYG{o}{=}\PYG{l+s+s2}{\PYGZdq{}}\PYG{l+s+s2}{GHG Reduction [gCO2e / system]}\PYG{l+s+s2}{\PYGZdq{}}\PYG{p}{)}
\PYG{n}{g}\PYG{o}{.}\PYG{n}{set\PYGZus{}xticklabels}\PYG{p}{(}\PYG{n}{g}\PYG{o}{.}\PYG{n}{get\PYGZus{}xticklabels}\PYG{p}{(}\PYG{p}{)}\PYG{p}{,} \PYG{n}{rotation}\PYG{o}{=}\PYG{l+m+mi}{30}\PYG{p}{)}\PYG{p}{;}
\end{sphinxVerbatim}

\noindent\sphinxincludegraphics{{output_35_0}.png}


\subsection{Plot LCOE metric.}
\label{\detokenize{example-analysis:plot-lcoe-metric}}
\begin{sphinxVerbatim}[commandchars=\\\{\}]
\PYG{n}{g} \PYG{o}{=} \PYG{n}{sb}\PYG{o}{.}\PYG{n}{boxplot}\PYG{p}{(}
    \PYG{n}{x}\PYG{o}{=}\PYG{l+s+s2}{\PYGZdq{}}\PYG{l+s+s2}{Scenario}\PYG{l+s+s2}{\PYGZdq{}}\PYG{p}{,}
    \PYG{n}{y}\PYG{o}{=}\PYG{l+s+s2}{\PYGZdq{}}\PYG{l+s+s2}{Value}\PYG{l+s+s2}{\PYGZdq{}}\PYG{p}{,}
    \PYG{n}{data}\PYG{o}{=}\PYG{n}{scenario\PYGZus{}results}\PYG{o}{.}\PYG{n}{xs}\PYG{p}{(}
        \PYG{p}{[}\PYG{l+s+s2}{\PYGZdq{}}\PYG{l+s+s2}{Metric}\PYG{l+s+s2}{\PYGZdq{}}\PYG{p}{,} \PYG{l+s+s2}{\PYGZdq{}}\PYG{l+s+s2}{LCOE}\PYG{l+s+s2}{\PYGZdq{}}\PYG{p}{]}\PYG{p}{,}
        \PYG{n}{level}\PYG{o}{=}\PYG{p}{[}\PYG{l+s+s2}{\PYGZdq{}}\PYG{l+s+s2}{Variable}\PYG{l+s+s2}{\PYGZdq{}}\PYG{p}{,} \PYG{l+s+s2}{\PYGZdq{}}\PYG{l+s+s2}{Index}\PYG{l+s+s2}{\PYGZdq{}}\PYG{p}{]}
    \PYG{p}{)}\PYG{o}{.}\PYG{n}{reset\PYGZus{}index}\PYG{p}{(}\PYG{p}{)}\PYG{p}{[}\PYG{p}{[}\PYG{l+s+s2}{\PYGZdq{}}\PYG{l+s+s2}{Scenario}\PYG{l+s+s2}{\PYGZdq{}}\PYG{p}{,} \PYG{l+s+s2}{\PYGZdq{}}\PYG{l+s+s2}{Value}\PYG{l+s+s2}{\PYGZdq{}}\PYG{p}{]}\PYG{p}{]}\PYG{p}{,}
    \PYG{n}{order}\PYG{o}{=}\PYG{p}{[}
        \PYG{l+s+s2}{\PYGZdq{}}\PYG{l+s+s2}{2015 Actual}\PYG{l+s+s2}{\PYGZdq{}}              \PYG{p}{,}
        \PYG{l+s+s2}{\PYGZdq{}}\PYG{l+s+s2}{Module Slow Progress}\PYG{l+s+s2}{\PYGZdq{}}      \PYG{p}{,}
        \PYG{l+s+s2}{\PYGZdq{}}\PYG{l+s+s2}{Module Moderate Progress}\PYG{l+s+s2}{\PYGZdq{}}  \PYG{p}{,}
        \PYG{l+s+s2}{\PYGZdq{}}\PYG{l+s+s2}{Module Fast Progress}\PYG{l+s+s2}{\PYGZdq{}}      \PYG{p}{,}
        \PYG{l+s+s2}{\PYGZdq{}}\PYG{l+s+s2}{Inverter Slow Progress}\PYG{l+s+s2}{\PYGZdq{}}    \PYG{p}{,}
        \PYG{l+s+s2}{\PYGZdq{}}\PYG{l+s+s2}{Inverter Moderate Progress}\PYG{l+s+s2}{\PYGZdq{}}\PYG{p}{,}
        \PYG{l+s+s2}{\PYGZdq{}}\PYG{l+s+s2}{Inverter Fast Progress}\PYG{l+s+s2}{\PYGZdq{}}    \PYG{p}{,}
        \PYG{l+s+s2}{\PYGZdq{}}\PYG{l+s+s2}{BoS Slow Progress}\PYG{l+s+s2}{\PYGZdq{}}         \PYG{p}{,}
        \PYG{l+s+s2}{\PYGZdq{}}\PYG{l+s+s2}{BoS Moderate Progress}\PYG{l+s+s2}{\PYGZdq{}}     \PYG{p}{,}
        \PYG{l+s+s2}{\PYGZdq{}}\PYG{l+s+s2}{BoS Fast Progress}\PYG{l+s+s2}{\PYGZdq{}}         \PYG{p}{,}
    \PYG{p}{]}
\PYG{p}{)}
\PYG{n}{g}\PYG{o}{.}\PYG{n}{set}\PYG{p}{(}\PYG{n}{ylabel}\PYG{o}{=}\PYG{l+s+s2}{\PYGZdq{}}\PYG{l+s+s2}{LCOE Reduction [USD / kWh]}\PYG{l+s+s2}{\PYGZdq{}}\PYG{p}{)}
\PYG{n}{g}\PYG{o}{.}\PYG{n}{set\PYGZus{}xticklabels}\PYG{p}{(}\PYG{n}{g}\PYG{o}{.}\PYG{n}{get\PYGZus{}xticklabels}\PYG{p}{(}\PYG{p}{)}\PYG{p}{,} \PYG{n}{rotation}\PYG{o}{=}\PYG{l+m+mi}{30}\PYG{p}{)}\PYG{p}{;}
\end{sphinxVerbatim}

\noindent\sphinxincludegraphics{{output_37_0}.png}


\subsection{Plot labor metric.}
\label{\detokenize{example-analysis:plot-labor-metric}}
\begin{sphinxVerbatim}[commandchars=\\\{\}]
\PYG{n}{g} \PYG{o}{=} \PYG{n}{sb}\PYG{o}{.}\PYG{n}{boxplot}\PYG{p}{(}
    \PYG{n}{x}\PYG{o}{=}\PYG{l+s+s2}{\PYGZdq{}}\PYG{l+s+s2}{Scenario}\PYG{l+s+s2}{\PYGZdq{}}\PYG{p}{,}
    \PYG{n}{y}\PYG{o}{=}\PYG{l+s+s2}{\PYGZdq{}}\PYG{l+s+s2}{Value}\PYG{l+s+s2}{\PYGZdq{}}\PYG{p}{,}
    \PYG{n}{data}\PYG{o}{=}\PYG{n}{scenario\PYGZus{}results}\PYG{o}{.}\PYG{n}{xs}\PYG{p}{(}
        \PYG{p}{[}\PYG{l+s+s2}{\PYGZdq{}}\PYG{l+s+s2}{Metric}\PYG{l+s+s2}{\PYGZdq{}}\PYG{p}{,} \PYG{l+s+s2}{\PYGZdq{}}\PYG{l+s+s2}{Labor}\PYG{l+s+s2}{\PYGZdq{}}\PYG{p}{]}\PYG{p}{,}
        \PYG{n}{level}\PYG{o}{=}\PYG{p}{[}\PYG{l+s+s2}{\PYGZdq{}}\PYG{l+s+s2}{Variable}\PYG{l+s+s2}{\PYGZdq{}}\PYG{p}{,} \PYG{l+s+s2}{\PYGZdq{}}\PYG{l+s+s2}{Index}\PYG{l+s+s2}{\PYGZdq{}}\PYG{p}{]}
    \PYG{p}{)}\PYG{o}{.}\PYG{n}{reset\PYGZus{}index}\PYG{p}{(}\PYG{p}{)}\PYG{p}{[}\PYG{p}{[}\PYG{l+s+s2}{\PYGZdq{}}\PYG{l+s+s2}{Scenario}\PYG{l+s+s2}{\PYGZdq{}}\PYG{p}{,} \PYG{l+s+s2}{\PYGZdq{}}\PYG{l+s+s2}{Value}\PYG{l+s+s2}{\PYGZdq{}}\PYG{p}{]}\PYG{p}{]}\PYG{p}{,}
    \PYG{n}{order}\PYG{o}{=}\PYG{p}{[}
        \PYG{l+s+s2}{\PYGZdq{}}\PYG{l+s+s2}{2015 Actual}\PYG{l+s+s2}{\PYGZdq{}}              \PYG{p}{,}
        \PYG{l+s+s2}{\PYGZdq{}}\PYG{l+s+s2}{Module Slow Progress}\PYG{l+s+s2}{\PYGZdq{}}      \PYG{p}{,}
        \PYG{l+s+s2}{\PYGZdq{}}\PYG{l+s+s2}{Module Moderate Progress}\PYG{l+s+s2}{\PYGZdq{}}  \PYG{p}{,}
        \PYG{l+s+s2}{\PYGZdq{}}\PYG{l+s+s2}{Module Fast Progress}\PYG{l+s+s2}{\PYGZdq{}}      \PYG{p}{,}
        \PYG{l+s+s2}{\PYGZdq{}}\PYG{l+s+s2}{Inverter Slow Progress}\PYG{l+s+s2}{\PYGZdq{}}    \PYG{p}{,}
        \PYG{l+s+s2}{\PYGZdq{}}\PYG{l+s+s2}{Inverter Moderate Progress}\PYG{l+s+s2}{\PYGZdq{}}\PYG{p}{,}
        \PYG{l+s+s2}{\PYGZdq{}}\PYG{l+s+s2}{Inverter Fast Progress}\PYG{l+s+s2}{\PYGZdq{}}    \PYG{p}{,}
        \PYG{l+s+s2}{\PYGZdq{}}\PYG{l+s+s2}{BoS Slow Progress}\PYG{l+s+s2}{\PYGZdq{}}         \PYG{p}{,}
        \PYG{l+s+s2}{\PYGZdq{}}\PYG{l+s+s2}{BoS Moderate Progress}\PYG{l+s+s2}{\PYGZdq{}}     \PYG{p}{,}
        \PYG{l+s+s2}{\PYGZdq{}}\PYG{l+s+s2}{BoS Fast Progress}\PYG{l+s+s2}{\PYGZdq{}}         \PYG{p}{,}
    \PYG{p}{]}
\PYG{p}{)}
\PYG{n}{g}\PYG{o}{.}\PYG{n}{set}\PYG{p}{(}\PYG{n}{ylabel}\PYG{o}{=}\PYG{l+s+s2}{\PYGZdq{}}\PYG{l+s+s2}{Labor Increase [USD / system]}\PYG{l+s+s2}{\PYGZdq{}}\PYG{p}{)}
\PYG{n}{g}\PYG{o}{.}\PYG{n}{set\PYGZus{}xticklabels}\PYG{p}{(}\PYG{n}{g}\PYG{o}{.}\PYG{n}{get\PYGZus{}xticklabels}\PYG{p}{(}\PYG{p}{)}\PYG{p}{,} \PYG{n}{rotation}\PYG{o}{=}\PYG{l+m+mi}{15}\PYG{p}{)}\PYG{p}{;}
\end{sphinxVerbatim}

\noindent\sphinxincludegraphics{{output_39_0}.png}


\section{Evaluate the investments in the dataset.}
\label{\detokenize{example-analysis:evaluate-the-investments-in-the-dataset}}
\begin{sphinxVerbatim}[commandchars=\\\{\}]
\PYG{n}{investment\PYGZus{}results} \PYG{o}{=} \PYG{n}{investments}\PYG{o}{.}\PYG{n}{evaluate\PYGZus{}investments}\PYG{p}{(}\PYG{n}{designs}\PYG{p}{,} \PYG{n}{sample\PYGZus{}count}\PYG{o}{=}\PYG{l+m+mi}{50}\PYG{p}{)}
\end{sphinxVerbatim}


\subsection{Costs of investments.}
\label{\detokenize{example-analysis:costs-of-investments}}
\begin{sphinxVerbatim}[commandchars=\\\{\}]
\PYG{n}{investment\PYGZus{}results}\PYG{o}{.}\PYG{n}{amounts}
\end{sphinxVerbatim}




\subsection{Benefits of investments.}
\label{\detokenize{example-analysis:benefits-of-investments}}
\begin{sphinxVerbatim}[commandchars=\\\{\}]
\PYG{n}{investment\PYGZus{}results}\PYG{o}{.}\PYG{n}{metrics}\PYG{o}{.}\PYG{n}{xs}\PYG{p}{(}\PYG{l+m+mi}{1}\PYG{p}{,} \PYG{n}{level}\PYG{o}{=}\PYG{l+s+s2}{\PYGZdq{}}\PYG{l+s+s2}{Sample}\PYG{l+s+s2}{\PYGZdq{}}\PYG{p}{,} \PYG{n}{drop\PYGZus{}level}\PYG{o}{=}\PYG{k+kc}{False}\PYG{p}{)}
\end{sphinxVerbatim}



\begin{sphinxVerbatim}[commandchars=\\\{\}]
\PYG{n}{investment\PYGZus{}results}\PYG{o}{.}\PYG{n}{summary}\PYG{o}{.}\PYG{n}{xs}\PYG{p}{(}\PYG{l+m+mi}{1}\PYG{p}{,} \PYG{n}{level}\PYG{o}{=}\PYG{l+s+s2}{\PYGZdq{}}\PYG{l+s+s2}{Sample}\PYG{l+s+s2}{\PYGZdq{}}\PYG{p}{,} \PYG{n}{drop\PYGZus{}level}\PYG{o}{=}\PYG{k+kc}{False}\PYG{p}{)}
\end{sphinxVerbatim}




\subsection{Save results.}
\label{\detokenize{example-analysis:id1}}
\begin{sphinxVerbatim}[commandchars=\\\{\}]
\PYG{n}{investment\PYGZus{}results}\PYG{o}{.}\PYG{n}{amounts}\PYG{o}{.}\PYG{n}{to\PYGZus{}csv}\PYG{p}{(}\PYG{l+s+s2}{\PYGZdq{}}\PYG{l+s+s2}{output/pv\PYGZus{}residential\PYGZus{}simple/example\PYGZhy{}investment\PYGZhy{}amounts.csv}\PYG{l+s+s2}{\PYGZdq{}}\PYG{p}{)}
\end{sphinxVerbatim}

\begin{sphinxVerbatim}[commandchars=\\\{\}]
\PYG{n}{investment\PYGZus{}results}\PYG{o}{.}\PYG{n}{metrics}\PYG{o}{.}\PYG{n}{to\PYGZus{}csv}\PYG{p}{(}\PYG{l+s+s2}{\PYGZdq{}}\PYG{l+s+s2}{output/pv\PYGZus{}residential\PYGZus{}simple/example\PYGZhy{}investment\PYGZhy{}metrics.csv}\PYG{l+s+s2}{\PYGZdq{}}\PYG{p}{)}
\end{sphinxVerbatim}


\subsection{Plot GHG metric.}
\label{\detokenize{example-analysis:id2}}
\begin{sphinxVerbatim}[commandchars=\\\{\}]
\PYG{n}{g} \PYG{o}{=} \PYG{n}{sb}\PYG{o}{.}\PYG{n}{boxplot}\PYG{p}{(}
    \PYG{n}{x}\PYG{o}{=}\PYG{l+s+s2}{\PYGZdq{}}\PYG{l+s+s2}{Investment}\PYG{l+s+s2}{\PYGZdq{}}\PYG{p}{,}
    \PYG{n}{y}\PYG{o}{=}\PYG{l+s+s2}{\PYGZdq{}}\PYG{l+s+s2}{Value}\PYG{l+s+s2}{\PYGZdq{}}\PYG{p}{,}
    \PYG{n}{data}\PYG{o}{=}\PYG{n}{investment\PYGZus{}results}\PYG{o}{.}\PYG{n}{metrics}\PYG{o}{.}\PYG{n}{xs}\PYG{p}{(}
        \PYG{l+s+s2}{\PYGZdq{}}\PYG{l+s+s2}{GHG}\PYG{l+s+s2}{\PYGZdq{}}\PYG{p}{,}
        \PYG{n}{level}\PYG{o}{=}\PYG{l+s+s2}{\PYGZdq{}}\PYG{l+s+s2}{Index}\PYG{l+s+s2}{\PYGZdq{}}
    \PYG{p}{)}\PYG{o}{.}\PYG{n}{reset\PYGZus{}index}\PYG{p}{(}\PYG{p}{)}\PYG{p}{[}\PYG{p}{[}\PYG{l+s+s2}{\PYGZdq{}}\PYG{l+s+s2}{Investment}\PYG{l+s+s2}{\PYGZdq{}}\PYG{p}{,} \PYG{l+s+s2}{\PYGZdq{}}\PYG{l+s+s2}{Value}\PYG{l+s+s2}{\PYGZdq{}}\PYG{p}{]}\PYG{p}{]}\PYG{p}{,}
    \PYG{n}{order}\PYG{o}{=}\PYG{p}{[}
        \PYG{l+s+s2}{\PYGZdq{}}\PYG{l+s+s2}{Low R\PYGZam{}D}\PYG{l+s+s2}{\PYGZdq{}}   \PYG{p}{,}
        \PYG{l+s+s2}{\PYGZdq{}}\PYG{l+s+s2}{Medium R\PYGZam{}D}\PYG{l+s+s2}{\PYGZdq{}}\PYG{p}{,}
        \PYG{l+s+s2}{\PYGZdq{}}\PYG{l+s+s2}{High R\PYGZam{}D}\PYG{l+s+s2}{\PYGZdq{}}  \PYG{p}{,}
    \PYG{p}{]}
\PYG{p}{)}
\PYG{n}{g}\PYG{o}{.}\PYG{n}{set}\PYG{p}{(}\PYG{n}{ylabel}\PYG{o}{=}\PYG{l+s+s2}{\PYGZdq{}}\PYG{l+s+s2}{GHG Reduction [gCO2e / system]}\PYG{l+s+s2}{\PYGZdq{}}\PYG{p}{)}
\PYG{n}{g}\PYG{o}{.}\PYG{n}{set\PYGZus{}xticklabels}\PYG{p}{(}\PYG{n}{g}\PYG{o}{.}\PYG{n}{get\PYGZus{}xticklabels}\PYG{p}{(}\PYG{p}{)}\PYG{p}{,} \PYG{n}{rotation}\PYG{o}{=}\PYG{l+m+mi}{15}\PYG{p}{)}\PYG{p}{;}
\end{sphinxVerbatim}

\noindent\sphinxincludegraphics{{output_51_0}.png}


\subsection{Plot LCOE metric.}
\label{\detokenize{example-analysis:id3}}
\begin{sphinxVerbatim}[commandchars=\\\{\}]
\PYG{n}{g} \PYG{o}{=} \PYG{n}{sb}\PYG{o}{.}\PYG{n}{boxplot}\PYG{p}{(}
    \PYG{n}{x}\PYG{o}{=}\PYG{l+s+s2}{\PYGZdq{}}\PYG{l+s+s2}{Investment}\PYG{l+s+s2}{\PYGZdq{}}\PYG{p}{,}
    \PYG{n}{y}\PYG{o}{=}\PYG{l+s+s2}{\PYGZdq{}}\PYG{l+s+s2}{Value}\PYG{l+s+s2}{\PYGZdq{}}\PYG{p}{,}
    \PYG{n}{data}\PYG{o}{=}\PYG{n}{investment\PYGZus{}results}\PYG{o}{.}\PYG{n}{metrics}\PYG{o}{.}\PYG{n}{xs}\PYG{p}{(}
        \PYG{l+s+s2}{\PYGZdq{}}\PYG{l+s+s2}{LCOE}\PYG{l+s+s2}{\PYGZdq{}}\PYG{p}{,}
        \PYG{n}{level}\PYG{o}{=}\PYG{l+s+s2}{\PYGZdq{}}\PYG{l+s+s2}{Index}\PYG{l+s+s2}{\PYGZdq{}}
    \PYG{p}{)}\PYG{o}{.}\PYG{n}{reset\PYGZus{}index}\PYG{p}{(}\PYG{p}{)}\PYG{p}{[}\PYG{p}{[}\PYG{l+s+s2}{\PYGZdq{}}\PYG{l+s+s2}{Investment}\PYG{l+s+s2}{\PYGZdq{}}\PYG{p}{,} \PYG{l+s+s2}{\PYGZdq{}}\PYG{l+s+s2}{Value}\PYG{l+s+s2}{\PYGZdq{}}\PYG{p}{]}\PYG{p}{]}\PYG{p}{,}
    \PYG{n}{order}\PYG{o}{=}\PYG{p}{[}
        \PYG{l+s+s2}{\PYGZdq{}}\PYG{l+s+s2}{Low R\PYGZam{}D}\PYG{l+s+s2}{\PYGZdq{}}   \PYG{p}{,}
        \PYG{l+s+s2}{\PYGZdq{}}\PYG{l+s+s2}{Medium R\PYGZam{}D}\PYG{l+s+s2}{\PYGZdq{}}\PYG{p}{,}
        \PYG{l+s+s2}{\PYGZdq{}}\PYG{l+s+s2}{High R\PYGZam{}D}\PYG{l+s+s2}{\PYGZdq{}}  \PYG{p}{,}
    \PYG{p}{]}
\PYG{p}{)}
\PYG{n}{g}\PYG{o}{.}\PYG{n}{set}\PYG{p}{(}\PYG{n}{ylabel}\PYG{o}{=}\PYG{l+s+s2}{\PYGZdq{}}\PYG{l+s+s2}{LCOE Reduction [USD / kWh]}\PYG{l+s+s2}{\PYGZdq{}}\PYG{p}{)}
\PYG{n}{g}\PYG{o}{.}\PYG{n}{set\PYGZus{}xticklabels}\PYG{p}{(}\PYG{n}{g}\PYG{o}{.}\PYG{n}{get\PYGZus{}xticklabels}\PYG{p}{(}\PYG{p}{)}\PYG{p}{,} \PYG{n}{rotation}\PYG{o}{=}\PYG{l+m+mi}{15}\PYG{p}{)}\PYG{p}{;}
\end{sphinxVerbatim}

\noindent\sphinxincludegraphics{{output_53_0}.png}


\subsection{Plot labor metric.}
\label{\detokenize{example-analysis:id4}}
\begin{sphinxVerbatim}[commandchars=\\\{\}]
\PYG{n}{g} \PYG{o}{=} \PYG{n}{sb}\PYG{o}{.}\PYG{n}{boxplot}\PYG{p}{(}
    \PYG{n}{x}\PYG{o}{=}\PYG{l+s+s2}{\PYGZdq{}}\PYG{l+s+s2}{Investment}\PYG{l+s+s2}{\PYGZdq{}}\PYG{p}{,}
    \PYG{n}{y}\PYG{o}{=}\PYG{l+s+s2}{\PYGZdq{}}\PYG{l+s+s2}{Value}\PYG{l+s+s2}{\PYGZdq{}}\PYG{p}{,}
    \PYG{n}{data}\PYG{o}{=}\PYG{n}{investment\PYGZus{}results}\PYG{o}{.}\PYG{n}{metrics}\PYG{o}{.}\PYG{n}{xs}\PYG{p}{(}
        \PYG{l+s+s2}{\PYGZdq{}}\PYG{l+s+s2}{Labor}\PYG{l+s+s2}{\PYGZdq{}}\PYG{p}{,}
        \PYG{n}{level}\PYG{o}{=}\PYG{l+s+s2}{\PYGZdq{}}\PYG{l+s+s2}{Index}\PYG{l+s+s2}{\PYGZdq{}}
    \PYG{p}{)}\PYG{o}{.}\PYG{n}{reset\PYGZus{}index}\PYG{p}{(}\PYG{p}{)}\PYG{p}{[}\PYG{p}{[}\PYG{l+s+s2}{\PYGZdq{}}\PYG{l+s+s2}{Investment}\PYG{l+s+s2}{\PYGZdq{}}\PYG{p}{,} \PYG{l+s+s2}{\PYGZdq{}}\PYG{l+s+s2}{Value}\PYG{l+s+s2}{\PYGZdq{}}\PYG{p}{]}\PYG{p}{]}\PYG{p}{,}
    \PYG{n}{order}\PYG{o}{=}\PYG{p}{[}
        \PYG{l+s+s2}{\PYGZdq{}}\PYG{l+s+s2}{Low R\PYGZam{}D}\PYG{l+s+s2}{\PYGZdq{}}   \PYG{p}{,}
        \PYG{l+s+s2}{\PYGZdq{}}\PYG{l+s+s2}{Medium R\PYGZam{}D}\PYG{l+s+s2}{\PYGZdq{}}\PYG{p}{,}
        \PYG{l+s+s2}{\PYGZdq{}}\PYG{l+s+s2}{High R\PYGZam{}D}\PYG{l+s+s2}{\PYGZdq{}}  \PYG{p}{,}
    \PYG{p}{]}
\PYG{p}{)}
\PYG{n}{g}\PYG{o}{.}\PYG{n}{set}\PYG{p}{(}\PYG{n}{ylabel}\PYG{o}{=}\PYG{l+s+s2}{\PYGZdq{}}\PYG{l+s+s2}{Labor Increase [USD / system]}\PYG{l+s+s2}{\PYGZdq{}}\PYG{p}{)}
\PYG{n}{g}\PYG{o}{.}\PYG{n}{set\PYGZus{}xticklabels}\PYG{p}{(}\PYG{n}{g}\PYG{o}{.}\PYG{n}{get\PYGZus{}xticklabels}\PYG{p}{(}\PYG{p}{)}\PYG{p}{,} \PYG{n}{rotation}\PYG{o}{=}\PYG{l+m+mi}{15}\PYG{p}{)}\PYG{p}{;}
\end{sphinxVerbatim}

\noindent\sphinxincludegraphics{{output_55_0}.png}


\section{Multi\sphinxhyphen{}objective decision analysis.}
\label{\detokenize{example-analysis:multi-objective-decision-analysis}}

\subsection{Compute costs and metrics for tranches.}
\label{\detokenize{example-analysis:compute-costs-and-metrics-for-tranches}}
\sphinxAtStartPar
Tranches are atomic units for building investment portfolios. Evaluate
all of the tranches, so we can assemble them into investments
(portfolios).

\begin{sphinxVerbatim}[commandchars=\\\{\}]
\PYG{n}{tranche\PYGZus{}results} \PYG{o}{=} \PYG{n}{investments}\PYG{o}{.}\PYG{n}{evaluate\PYGZus{}tranches}\PYG{p}{(}\PYG{n}{designs}\PYG{p}{,} \PYG{n}{sample\PYGZus{}count}\PYG{o}{=}\PYG{l+m+mi}{50}\PYG{p}{)}
\end{sphinxVerbatim}

\sphinxAtStartPar
Display the cost of each tranche.

\begin{sphinxVerbatim}[commandchars=\\\{\}]
\PYG{n}{tranche\PYGZus{}results}\PYG{o}{.}\PYG{n}{amounts}
\end{sphinxVerbatim}



\sphinxAtStartPar
Display the metrics for each tranche.

\begin{sphinxVerbatim}[commandchars=\\\{\}]
\PYG{n}{tranche\PYGZus{}results}\PYG{o}{.}\PYG{n}{summary}
\end{sphinxVerbatim}



\sphinxAtStartPar
Save the results.

\begin{sphinxVerbatim}[commandchars=\\\{\}]
\PYG{n}{tranche\PYGZus{}results}\PYG{o}{.}\PYG{n}{amounts}\PYG{o}{.}\PYG{n}{to\PYGZus{}csv}\PYG{p}{(}\PYG{l+s+s2}{\PYGZdq{}}\PYG{l+s+s2}{output/pv\PYGZus{}residential\PYGZus{}simple/example\PYGZhy{}tranche\PYGZhy{}amounts.csv}\PYG{l+s+s2}{\PYGZdq{}}\PYG{p}{)}
\PYG{n}{tranche\PYGZus{}results}\PYG{o}{.}\PYG{n}{summary}\PYG{o}{.}\PYG{n}{to\PYGZus{}csv}\PYG{p}{(}\PYG{l+s+s2}{\PYGZdq{}}\PYG{l+s+s2}{output/pv\PYGZus{}residential\PYGZus{}simple/example\PYGZhy{}tranche\PYGZhy{}summary.csv}\PYG{l+s+s2}{\PYGZdq{}}\PYG{p}{)}
\end{sphinxVerbatim}


\subsection{Fit a response surface to the results.}
\label{\detokenize{example-analysis:fit-a-response-surface-to-the-results}}
\sphinxAtStartPar
The response surface interpolates between the discrete set of cases
provided in the expert elicitation. This allows us to study funding
levels intermediate between those scenarios.

\begin{sphinxVerbatim}[commandchars=\\\{\}]
\PYG{n}{evaluator} \PYG{o}{=} \PYG{n}{ty}\PYG{o}{.}\PYG{n}{Evaluator}\PYG{p}{(}\PYG{n}{investments}\PYG{o}{.}\PYG{n}{tranches}\PYG{p}{,} \PYG{n}{tranche\PYGZus{}results}\PYG{o}{.}\PYG{n}{summary}\PYG{p}{)}
\end{sphinxVerbatim}

\sphinxAtStartPar
Here are the categories of investment and the maximum amount that could
be invested in each:

\begin{sphinxVerbatim}[commandchars=\\\{\}]
\PYG{n}{evaluator}\PYG{o}{.}\PYG{n}{max\PYGZus{}amount}
\end{sphinxVerbatim}



\sphinxAtStartPar
Here are the metrics and their units of measure:

\begin{sphinxVerbatim}[commandchars=\\\{\}]
\PYG{n}{evaluator}\PYG{o}{.}\PYG{n}{units}
\end{sphinxVerbatim}




\subsubsection{Example interpolation.}
\label{\detokenize{example-analysis:example-interpolation}}
\sphinxAtStartPar
Let’s evaluate the case where each category is invested in at half of
its maximum amount.

\begin{sphinxVerbatim}[commandchars=\\\{\}]
\PYG{n}{example\PYGZus{}investments} \PYG{o}{=} \PYG{n}{evaluator}\PYG{o}{.}\PYG{n}{max\PYGZus{}amount} \PYG{o}{/} \PYG{l+m+mi}{2}
\PYG{n}{example\PYGZus{}investments}
\end{sphinxVerbatim}



\begin{sphinxVerbatim}[commandchars=\\\{\}]
\PYG{n}{evaluator}\PYG{o}{.}\PYG{n}{evaluate}\PYG{p}{(}\PYG{n}{example\PYGZus{}investments}\PYG{p}{)}
\end{sphinxVerbatim}

\begin{sphinxVerbatim}[commandchars=\\\{\}]
\PYG{n}{Category}    \PYG{n}{Index}  \PYG{n}{Sample}
\PYG{n}{BoS} \PYG{n}{R}\PYG{o}{\PYGZam{}}\PYG{n}{D}     \PYG{n}{GHG}    \PYG{l+m+mi}{1}         \PYG{o}{\PYGZhy{}}\PYG{l+m+mf}{0.0010586097518157094}
                   \PYG{l+m+mi}{2}          \PYG{l+m+mf}{7.493162517135921e\PYGZhy{}05}
                   \PYG{l+m+mi}{3}           \PYG{l+m+mf}{0.001253893601450784}
                   \PYG{l+m+mi}{4}           \PYG{o}{\PYGZhy{}}\PYG{l+m+mf}{0.00398626797827717}
                   \PYG{l+m+mi}{5}          \PYG{o}{\PYGZhy{}}\PYG{l+m+mf}{0.005572343870333896}
                                      \PYG{o}{.}\PYG{o}{.}\PYG{o}{.}
\PYG{n}{Module} \PYG{n}{R}\PYG{o}{\PYGZam{}}\PYG{n}{D}  \PYG{n}{Labor}  \PYG{l+m+mi}{46}          \PYG{l+m+mf}{0.014371009324918305}
                   \PYG{l+m+mi}{47}          \PYG{l+m+mf}{0.011128728287076228}
                   \PYG{l+m+mi}{48}         \PYG{l+m+mf}{0.0039832773605894545}
                   \PYG{l+m+mi}{49}          \PYG{l+m+mf}{0.006026680267950724}
                   \PYG{l+m+mi}{50}          \PYG{l+m+mf}{0.028844695933457842}
\PYG{n}{Name}\PYG{p}{:} \PYG{n}{Value}\PYG{p}{,} \PYG{n}{Length}\PYG{p}{:} \PYG{l+m+mi}{450}\PYG{p}{,} \PYG{n}{dtype}\PYG{p}{:} \PYG{n+nb}{object}
\end{sphinxVerbatim}

\sphinxAtStartPar
Let’s evaluate the mean instead of outputing the whole distribution.

\begin{sphinxVerbatim}[commandchars=\\\{\}]
\PYG{n}{evaluator}\PYG{o}{.}\PYG{n}{evaluate\PYGZus{}statistic}\PYG{p}{(}\PYG{n}{example\PYGZus{}investments}\PYG{p}{,} \PYG{n}{np}\PYG{o}{.}\PYG{n}{mean}\PYG{p}{)}
\end{sphinxVerbatim}

\begin{sphinxVerbatim}[commandchars=\\\{\}]
\PYG{n}{Index}
\PYG{n}{GHG}       \PYG{l+m+mf}{30.156830}
\PYG{n}{LCOE}       \PYG{l+m+mf}{0.038160}
\PYG{n}{Labor}   \PYG{o}{\PYGZhy{}}\PYG{l+m+mf}{246.843027}
\PYG{n}{Name}\PYG{p}{:} \PYG{n}{Value}\PYG{p}{,} \PYG{n}{dtype}\PYG{p}{:} \PYG{n}{float64}
\end{sphinxVerbatim}

\sphinxAtStartPar
Here is the standard deviation:

\begin{sphinxVerbatim}[commandchars=\\\{\}]
\PYG{n}{evaluator}\PYG{o}{.}\PYG{n}{evaluate\PYGZus{}statistic}\PYG{p}{(}\PYG{n}{example\PYGZus{}investments}\PYG{p}{,} \PYG{n}{np}\PYG{o}{.}\PYG{n}{std}\PYG{p}{)}
\end{sphinxVerbatim}

\begin{sphinxVerbatim}[commandchars=\\\{\}]
\PYG{n}{Index}
\PYG{n}{GHG}       \PYG{l+m+mf}{1.410956}
\PYG{n}{LCOE}      \PYG{l+m+mf}{0.000850}
\PYG{n}{Labor}    \PYG{l+m+mf}{16.070395}
\PYG{n}{Name}\PYG{p}{:} \PYG{n}{Value}\PYG{p}{,} \PYG{n}{dtype}\PYG{p}{:} \PYG{n}{float64}
\end{sphinxVerbatim}

\sphinxAtStartPar
A risk\sphinxhyphen{}averse decision maker might be interested in the 10\% percentile:

\begin{sphinxVerbatim}[commandchars=\\\{\}]
\PYG{n}{evaluator}\PYG{o}{.}\PYG{n}{evaluate\PYGZus{}statistic}\PYG{p}{(}\PYG{n}{example\PYGZus{}investments}\PYG{p}{,} \PYG{k}{lambda} \PYG{n}{x}\PYG{p}{:} \PYG{n}{np}\PYG{o}{.}\PYG{n}{quantile}\PYG{p}{(}\PYG{n}{x}\PYG{p}{,} \PYG{l+m+mf}{0.1}\PYG{p}{)}\PYG{p}{)}
\end{sphinxVerbatim}

\begin{sphinxVerbatim}[commandchars=\\\{\}]
\PYG{n}{Index}
\PYG{n}{GHG}       \PYG{l+m+mf}{28.573627}
\PYG{n}{LCOE}       \PYG{l+m+mf}{0.037140}
\PYG{n}{Labor}   \PYG{o}{\PYGZhy{}}\PYG{l+m+mf}{268.059699}
\PYG{n}{Name}\PYG{p}{:} \PYG{n}{Value}\PYG{p}{,} \PYG{n}{dtype}\PYG{p}{:} \PYG{n}{float64}
\end{sphinxVerbatim}


\subsection{ε\sphinxhyphen{}Constraint multiobjective optimization}
\label{\detokenize{example-analysis:constraint-multiobjective-optimization}}
\begin{sphinxVerbatim}[commandchars=\\\{\}]
\PYG{n}{optimizer} \PYG{o}{=} \PYG{n}{ty}\PYG{o}{.}\PYG{n}{EpsilonConstraintOptimizer}\PYG{p}{(}\PYG{n}{evaluator}\PYG{p}{)}
\end{sphinxVerbatim}

\sphinxAtStartPar
In order to meaningfully map the decision space, we need to know the
maximum values for each of the metrics.

\begin{sphinxVerbatim}[commandchars=\\\{\}]
\PYG{n}{metric\PYGZus{}max} \PYG{o}{=} \PYG{n}{optimizer}\PYG{o}{.}\PYG{n}{max\PYGZus{}metrics}\PYG{p}{(}\PYG{p}{)}
\PYG{n}{metric\PYGZus{}max}
\end{sphinxVerbatim}

\begin{sphinxVerbatim}[commandchars=\\\{\}]
\PYG{n}{GHG}      \PYG{l+m+mf}{49.429976}
\PYG{n}{LCOE}      \PYG{l+m+mf}{0.062818}
\PYG{n}{Labor}     \PYG{l+m+mf}{0.049555}
\PYG{n}{Name}\PYG{p}{:} \PYG{n}{Value}\PYG{p}{,} \PYG{n}{dtype}\PYG{p}{:} \PYG{n}{float64}
\end{sphinxVerbatim}


\subsubsection{Example optimization.}
\label{\detokenize{example-analysis:example-optimization}}
\sphinxAtStartPar
Limit spending to \$3M.

\begin{sphinxVerbatim}[commandchars=\\\{\}]
\PYG{n}{investment\PYGZus{}max} \PYG{o}{=} \PYG{l+m+mf}{3e6}
\end{sphinxVerbatim}

\sphinxAtStartPar
Require that the GHG reduction be at least 40 gCO2e/system and that the
Labor wages not decrease.

\begin{sphinxVerbatim}[commandchars=\\\{\}]
\PYG{n}{metric\PYGZus{}min} \PYG{o}{=} \PYG{n}{pd}\PYG{o}{.}\PYG{n}{Series}\PYG{p}{(}\PYG{p}{[}\PYG{l+m+mi}{40}\PYG{p}{,} \PYG{l+m+mi}{0}\PYG{p}{]}\PYG{p}{,} \PYG{n}{name} \PYG{o}{=} \PYG{l+s+s2}{\PYGZdq{}}\PYG{l+s+s2}{Value}\PYG{l+s+s2}{\PYGZdq{}}\PYG{p}{,} \PYG{n}{index} \PYG{o}{=} \PYG{p}{[}\PYG{l+s+s2}{\PYGZdq{}}\PYG{l+s+s2}{GHG}\PYG{l+s+s2}{\PYGZdq{}}\PYG{p}{,} \PYG{l+s+s2}{\PYGZdq{}}\PYG{l+s+s2}{Labor}\PYG{l+s+s2}{\PYGZdq{}}\PYG{p}{]}\PYG{p}{)}
\PYG{n}{metric\PYGZus{}min}
\end{sphinxVerbatim}

\begin{sphinxVerbatim}[commandchars=\\\{\}]
\PYG{n}{GHG}      \PYG{l+m+mi}{40}
\PYG{n}{Labor}     \PYG{l+m+mi}{0}
\PYG{n}{Name}\PYG{p}{:} \PYG{n}{Value}\PYG{p}{,} \PYG{n}{dtype}\PYG{p}{:} \PYG{n}{int64}
\end{sphinxVerbatim}

\sphinxAtStartPar
Compute the ε\sphinxhyphen{}constrained maximum for the LCOE.

\begin{sphinxVerbatim}[commandchars=\\\{\}]
\PYG{n}{optimum} \PYG{o}{=} \PYG{n}{optimizer}\PYG{o}{.}\PYG{n}{maximize}\PYG{p}{(}
    \PYG{l+s+s2}{\PYGZdq{}}\PYG{l+s+s2}{LCOE}\PYG{l+s+s2}{\PYGZdq{}}                       \PYG{p}{,}
    \PYG{n}{total\PYGZus{}amount} \PYG{o}{=} \PYG{n}{investment\PYGZus{}max}\PYG{p}{,}
    \PYG{n}{min\PYGZus{}metric}   \PYG{o}{=} \PYG{n}{metric\PYGZus{}min}    \PYG{p}{,}
    \PYG{n}{statistic}    \PYG{o}{=} \PYG{n}{np}\PYG{o}{.}\PYG{n}{mean}       \PYG{p}{,}
\PYG{p}{)}
\PYG{n}{optimum}\PYG{o}{.}\PYG{n}{exit\PYGZus{}message}
\end{sphinxVerbatim}

\begin{sphinxVerbatim}[commandchars=\\\{\}]
\PYG{l+s+s1}{\PYGZsq{}}\PYG{l+s+s1}{Optimization terminated successfully.}\PYG{l+s+s1}{\PYGZsq{}}
\end{sphinxVerbatim}

\sphinxAtStartPar
Here are the optimal spending levels:

\begin{sphinxVerbatim}[commandchars=\\\{\}]
\PYG{n}{np}\PYG{o}{.}\PYG{n}{round}\PYG{p}{(}\PYG{n}{optimum}\PYG{o}{.}\PYG{n}{amounts}\PYG{p}{)}
\end{sphinxVerbatim}

\begin{sphinxVerbatim}[commandchars=\\\{\}]
\PYG{n}{Category}
\PYG{n}{BoS} \PYG{n}{R}\PYG{o}{\PYGZam{}}\PYG{n}{D}               \PYG{l+m+mf}{0.0}
\PYG{n}{Inverter} \PYG{n}{R}\PYG{o}{\PYGZam{}}\PYG{n}{D}          \PYG{l+m+mf}{0.0}
\PYG{n}{Module} \PYG{n}{R}\PYG{o}{\PYGZam{}}\PYG{n}{D}      \PYG{l+m+mf}{3000000.0}
\PYG{n}{Name}\PYG{p}{:} \PYG{n}{Amount}\PYG{p}{,} \PYG{n}{dtype}\PYG{p}{:} \PYG{n}{float64}
\end{sphinxVerbatim}

\sphinxAtStartPar
Here are the three metrics at that optimum:

\begin{sphinxVerbatim}[commandchars=\\\{\}]
\PYG{n}{optimum}\PYG{o}{.}\PYG{n}{metrics}
\end{sphinxVerbatim}

\begin{sphinxVerbatim}[commandchars=\\\{\}]
\PYG{n}{Index}
\PYG{n}{GHG}      \PYG{l+m+mf}{41.627691}
\PYG{n}{LCOE}      \PYG{l+m+mf}{0.037566}
\PYG{n}{Labor}     \PYG{l+m+mf}{0.028691}
\PYG{n}{Name}\PYG{p}{:} \PYG{n}{Value}\PYG{p}{,} \PYG{n}{dtype}\PYG{p}{:} \PYG{n}{float64}
\end{sphinxVerbatim}

\sphinxAtStartPar
\sphinxstyleemphasis{Thus, by putting all of the investment into Module R\&D, we can expected
to achieve a mean 3.75 ¢/kWh reduction in LCOE under the GHG and Labor
constraints.}

\sphinxAtStartPar
It turns out that there is no solution for these constraints if we
evaluate the 10th percentile of the metrics, for a risk\sphinxhyphen{}averse decision
maker.

\begin{sphinxVerbatim}[commandchars=\\\{\}]
\PYG{n}{optimum} \PYG{o}{=} \PYG{n}{optimizer}\PYG{o}{.}\PYG{n}{maximize}\PYG{p}{(}
    \PYG{l+s+s2}{\PYGZdq{}}\PYG{l+s+s2}{LCOE}\PYG{l+s+s2}{\PYGZdq{}}                       \PYG{p}{,}
    \PYG{n}{total\PYGZus{}amount} \PYG{o}{=} \PYG{n}{investment\PYGZus{}max}\PYG{p}{,}
    \PYG{n}{min\PYGZus{}metric}   \PYG{o}{=} \PYG{n}{metric\PYGZus{}min}    \PYG{p}{,}
    \PYG{n}{statistic}    \PYG{o}{=} \PYG{k}{lambda} \PYG{n}{x}\PYG{p}{:} \PYG{n}{np}\PYG{o}{.}\PYG{n}{quantile}\PYG{p}{(}\PYG{n}{x}\PYG{p}{,} \PYG{l+m+mf}{0.1}\PYG{p}{)}\PYG{p}{,}
\PYG{p}{)}
\PYG{n}{optimum}\PYG{o}{.}\PYG{n}{exit\PYGZus{}message}
\end{sphinxVerbatim}

\begin{sphinxVerbatim}[commandchars=\\\{\}]
\PYG{l+s+s1}{\PYGZsq{}}\PYG{l+s+s1}{Iteration limit exceeded}\PYG{l+s+s1}{\PYGZsq{}}
\end{sphinxVerbatim}

\sphinxAtStartPar
Let’s try again, but with a less stringent set of constraints, only
constraining GHG somewhat but not Labor at all.

\begin{sphinxVerbatim}[commandchars=\\\{\}]
\PYG{n}{optimum} \PYG{o}{=} \PYG{n}{optimizer}\PYG{o}{.}\PYG{n}{maximize}\PYG{p}{(}
    \PYG{l+s+s2}{\PYGZdq{}}\PYG{l+s+s2}{LCOE}\PYG{l+s+s2}{\PYGZdq{}}                                                         \PYG{p}{,}
    \PYG{n}{total\PYGZus{}amount} \PYG{o}{=} \PYG{n}{investment\PYGZus{}max}                                  \PYG{p}{,}
    \PYG{n}{min\PYGZus{}metric}   \PYG{o}{=} \PYG{n}{pd}\PYG{o}{.}\PYG{n}{Series}\PYG{p}{(}\PYG{p}{[}\PYG{l+m+mi}{30}\PYG{p}{]}\PYG{p}{,} \PYG{n}{name} \PYG{o}{=} \PYG{l+s+s2}{\PYGZdq{}}\PYG{l+s+s2}{Value}\PYG{l+s+s2}{\PYGZdq{}}\PYG{p}{,} \PYG{n}{index} \PYG{o}{=} \PYG{p}{[}\PYG{l+s+s2}{\PYGZdq{}}\PYG{l+s+s2}{GHG}\PYG{l+s+s2}{\PYGZdq{}}\PYG{p}{]}\PYG{p}{)}\PYG{p}{,}
    \PYG{n}{statistic}    \PYG{o}{=} \PYG{k}{lambda} \PYG{n}{x}\PYG{p}{:} \PYG{n}{np}\PYG{o}{.}\PYG{n}{quantile}\PYG{p}{(}\PYG{n}{x}\PYG{p}{,} \PYG{l+m+mf}{0.1}\PYG{p}{)}                   \PYG{p}{,}
\PYG{p}{)}
\PYG{n}{optimum}\PYG{o}{.}\PYG{n}{exit\PYGZus{}message}
\end{sphinxVerbatim}

\begin{sphinxVerbatim}[commandchars=\\\{\}]
\PYG{l+s+s1}{\PYGZsq{}}\PYG{l+s+s1}{Optimization terminated successfully.}\PYG{l+s+s1}{\PYGZsq{}}
\end{sphinxVerbatim}

\begin{sphinxVerbatim}[commandchars=\\\{\}]
\PYG{n}{np}\PYG{o}{.}\PYG{n}{round}\PYG{p}{(}\PYG{n}{optimum}\PYG{o}{.}\PYG{n}{amounts}\PYG{p}{)}
\end{sphinxVerbatim}

\begin{sphinxVerbatim}[commandchars=\\\{\}]
\PYG{n}{Category}
\PYG{n}{BoS} \PYG{n}{R}\PYG{o}{\PYGZam{}}\PYG{n}{D}               \PYG{l+m+mf}{0.0}
\PYG{n}{Inverter} \PYG{n}{R}\PYG{o}{\PYGZam{}}\PYG{n}{D}          \PYG{l+m+mf}{0.0}
\PYG{n}{Module} \PYG{n}{R}\PYG{o}{\PYGZam{}}\PYG{n}{D}      \PYG{l+m+mf}{3000000.0}
\PYG{n}{Name}\PYG{p}{:} \PYG{n}{Amount}\PYG{p}{,} \PYG{n}{dtype}\PYG{p}{:} \PYG{n}{float64}
\end{sphinxVerbatim}

\begin{sphinxVerbatim}[commandchars=\\\{\}]
\PYG{n}{optimum}\PYG{o}{.}\PYG{n}{metrics}
\end{sphinxVerbatim}

\begin{sphinxVerbatim}[commandchars=\\\{\}]
\PYG{n}{Index}
\PYG{n}{GHG}      \PYG{l+m+mf}{39.046988}
\PYG{n}{LCOE}      \PYG{l+m+mf}{0.036463}
\PYG{n}{Labor}    \PYG{o}{\PYGZhy{}}\PYG{l+m+mf}{0.019725}
\PYG{n}{Name}\PYG{p}{:} \PYG{n}{Value}\PYG{p}{,} \PYG{n}{dtype}\PYG{p}{:} \PYG{n}{float64}
\end{sphinxVerbatim}


\subsection{Pareto surfaces.}
\label{\detokenize{example-analysis:pareto-surfaces}}

\subsubsection{Metrics constrained by total investment.}
\label{\detokenize{example-analysis:metrics-constrained-by-total-investment}}
\begin{sphinxVerbatim}[commandchars=\\\{\}]
\PYG{n}{pareto\PYGZus{}amounts} \PYG{o}{=} \PYG{k+kc}{None}
\PYG{k}{for} \PYG{n}{investment\PYGZus{}max} \PYG{o+ow}{in} \PYG{n}{np}\PYG{o}{.}\PYG{n}{arange}\PYG{p}{(}\PYG{l+m+mf}{1e6}\PYG{p}{,} \PYG{l+m+mf}{9e6}\PYG{p}{,} \PYG{l+m+mf}{0.5e6}\PYG{p}{)}\PYG{p}{:}
    \PYG{n}{metrics} \PYG{o}{=} \PYG{n}{optimizer}\PYG{o}{.}\PYG{n}{max\PYGZus{}metrics}\PYG{p}{(}\PYG{n}{total\PYGZus{}amount} \PYG{o}{=} \PYG{n}{investment\PYGZus{}max}\PYG{p}{)}
    \PYG{n}{pareto\PYGZus{}amounts} \PYG{o}{=} \PYG{n}{pd}\PYG{o}{.}\PYG{n}{DataFrame}\PYG{p}{(}
        \PYG{p}{[}\PYG{n}{metrics}\PYG{o}{.}\PYG{n}{values}\PYG{p}{]}                                         \PYG{p}{,}
        \PYG{n}{columns} \PYG{o}{=} \PYG{n}{metrics}\PYG{o}{.}\PYG{n}{index}\PYG{o}{.}\PYG{n}{values}                           \PYG{p}{,}
        \PYG{n}{index}   \PYG{o}{=} \PYG{n}{pd}\PYG{o}{.}\PYG{n}{Index}\PYG{p}{(}\PYG{p}{[}\PYG{n}{investment\PYGZus{}max} \PYG{o}{/} \PYG{l+m+mf}{1e6}\PYG{p}{]}\PYG{p}{,} \PYG{n}{name} \PYG{o}{=} \PYG{l+s+s2}{\PYGZdq{}}\PYG{l+s+s2}{Investment [M\PYGZdl{}]}\PYG{l+s+s2}{\PYGZdq{}}\PYG{p}{)}\PYG{p}{,}
    \PYG{p}{)}\PYG{o}{.}\PYG{n}{append}\PYG{p}{(}\PYG{n}{pareto\PYGZus{}amounts}\PYG{p}{)}
\PYG{n}{pareto\PYGZus{}amounts}
\end{sphinxVerbatim}



\begin{sphinxVerbatim}[commandchars=\\\{\}]
\PYG{n}{sb}\PYG{o}{.}\PYG{n}{relplot}\PYG{p}{(}
    \PYG{n}{x}         \PYG{o}{=} \PYG{l+s+s2}{\PYGZdq{}}\PYG{l+s+s2}{Investment [M\PYGZdl{}]}\PYG{l+s+s2}{\PYGZdq{}}\PYG{p}{,}
    \PYG{n}{y}         \PYG{o}{=} \PYG{l+s+s2}{\PYGZdq{}}\PYG{l+s+s2}{Value}\PYG{l+s+s2}{\PYGZdq{}}          \PYG{p}{,}
    \PYG{n}{col}       \PYG{o}{=} \PYG{l+s+s2}{\PYGZdq{}}\PYG{l+s+s2}{Metric}\PYG{l+s+s2}{\PYGZdq{}}         \PYG{p}{,}
    \PYG{n}{kind}      \PYG{o}{=} \PYG{l+s+s2}{\PYGZdq{}}\PYG{l+s+s2}{line}\PYG{l+s+s2}{\PYGZdq{}}           \PYG{p}{,}
    \PYG{n}{facet\PYGZus{}kws} \PYG{o}{=} \PYG{p}{\PYGZob{}}\PYG{l+s+s1}{\PYGZsq{}}\PYG{l+s+s1}{sharey}\PYG{l+s+s1}{\PYGZsq{}}\PYG{p}{:} \PYG{k+kc}{False}\PYG{p}{\PYGZcb{}}\PYG{p}{,}
    \PYG{n}{data}      \PYG{o}{=} \PYG{n}{pareto\PYGZus{}amounts}\PYG{o}{.}\PYG{n}{reset\PYGZus{}index}\PYG{p}{(}\PYG{p}{)}\PYG{o}{.}\PYG{n}{melt}\PYG{p}{(}\PYG{n}{id\PYGZus{}vars} \PYG{o}{=} \PYG{l+s+s2}{\PYGZdq{}}\PYG{l+s+s2}{Investment [M\PYGZdl{}]}\PYG{l+s+s2}{\PYGZdq{}}\PYG{p}{,} \PYG{n}{var\PYGZus{}name} \PYG{o}{=} \PYG{l+s+s2}{\PYGZdq{}}\PYG{l+s+s2}{Metric}\PYG{l+s+s2}{\PYGZdq{}}\PYG{p}{,} \PYG{n}{value\PYGZus{}name} \PYG{o}{=} \PYG{l+s+s2}{\PYGZdq{}}\PYG{l+s+s2}{Value}\PYG{l+s+s2}{\PYGZdq{}}\PYG{p}{)}
\PYG{p}{)}
\end{sphinxVerbatim}

\begin{sphinxVerbatim}[commandchars=\\\{\}]
\PYG{o}{\PYGZlt{}}\PYG{n}{seaborn}\PYG{o}{.}\PYG{n}{axisgrid}\PYG{o}{.}\PYG{n}{FacetGrid} \PYG{n}{at} \PYG{l+m+mh}{0x7f9da11752b0}\PYG{o}{\PYGZgt{}}
\end{sphinxVerbatim}

\noindent\sphinxincludegraphics{{output_108_1}.png}

\sphinxAtStartPar
\sphinxstyleemphasis{We see that the LCOE metric saturates more slowly than the GHG and
Labor ones.}


\subsubsection{GHG vs LCOE, constrained by total investment.}
\label{\detokenize{example-analysis:ghg-vs-lcoe-constrained-by-total-investment}}
\begin{sphinxVerbatim}[commandchars=\\\{\}]
\PYG{n}{investment\PYGZus{}max} \PYG{o}{=} \PYG{l+m+mi}{3}
\PYG{n}{pareto\PYGZus{}ghg\PYGZus{}lcoe} \PYG{o}{=} \PYG{k+kc}{None}
\PYG{k}{for} \PYG{n}{lcoe\PYGZus{}min} \PYG{o+ow}{in} \PYG{l+m+mf}{0.95} \PYG{o}{*} \PYG{n}{np}\PYG{o}{.}\PYG{n}{arange}\PYG{p}{(}\PYG{l+m+mf}{0.5}\PYG{p}{,} \PYG{l+m+mf}{0.9}\PYG{p}{,} \PYG{l+m+mf}{0.05}\PYG{p}{)} \PYG{o}{*} \PYG{n}{pareto\PYGZus{}amounts}\PYG{o}{.}\PYG{n}{loc}\PYG{p}{[}\PYG{n}{investment\PYGZus{}max}\PYG{p}{,} \PYG{l+s+s2}{\PYGZdq{}}\PYG{l+s+s2}{LCOE}\PYG{l+s+s2}{\PYGZdq{}}\PYG{p}{]}\PYG{p}{:}
    \PYG{n}{optimum} \PYG{o}{=} \PYG{n}{optimizer}\PYG{o}{.}\PYG{n}{maximize}\PYG{p}{(}
        \PYG{l+s+s2}{\PYGZdq{}}\PYG{l+s+s2}{GHG}\PYG{l+s+s2}{\PYGZdq{}}\PYG{p}{,}
        \PYG{n}{max\PYGZus{}amount}   \PYG{o}{=} \PYG{n}{pd}\PYG{o}{.}\PYG{n}{Series}\PYG{p}{(}\PYG{p}{[}\PYG{l+m+mf}{0.9e6}\PYG{p}{,} \PYG{l+m+mf}{3.0e6}\PYG{p}{,} \PYG{l+m+mf}{1.0e6}\PYG{p}{]}\PYG{p}{,} \PYG{n}{name} \PYG{o}{=} \PYG{l+s+s2}{\PYGZdq{}}\PYG{l+s+s2}{Amount}\PYG{l+s+s2}{\PYGZdq{}}\PYG{p}{,} \PYG{n}{index} \PYG{o}{=} \PYG{p}{[}\PYG{l+s+s2}{\PYGZdq{}}\PYG{l+s+s2}{BoS R\PYGZam{}D}\PYG{l+s+s2}{\PYGZdq{}}\PYG{p}{,} \PYG{l+s+s2}{\PYGZdq{}}\PYG{l+s+s2}{Inverter R\PYGZam{}D}\PYG{l+s+s2}{\PYGZdq{}}\PYG{p}{,} \PYG{l+s+s2}{\PYGZdq{}}\PYG{l+s+s2}{Module R\PYGZam{}D}\PYG{l+s+s2}{\PYGZdq{}}\PYG{p}{]}\PYG{p}{)}\PYG{p}{,}
        \PYG{n}{total\PYGZus{}amount} \PYG{o}{=} \PYG{n}{investment\PYGZus{}max} \PYG{o}{*} \PYG{l+m+mf}{1e6}                                 \PYG{p}{,}
        \PYG{n}{min\PYGZus{}metric}   \PYG{o}{=} \PYG{n}{pd}\PYG{o}{.}\PYG{n}{Series}\PYG{p}{(}\PYG{p}{[}\PYG{n}{lcoe\PYGZus{}min}\PYG{p}{]}\PYG{p}{,} \PYG{n}{name} \PYG{o}{=} \PYG{l+s+s2}{\PYGZdq{}}\PYG{l+s+s2}{Value}\PYG{l+s+s2}{\PYGZdq{}}\PYG{p}{,} \PYG{n}{index} \PYG{o}{=} \PYG{p}{[}\PYG{l+s+s2}{\PYGZdq{}}\PYG{l+s+s2}{LCOE}\PYG{l+s+s2}{\PYGZdq{}}\PYG{p}{]}\PYG{p}{)}\PYG{p}{,}
    \PYG{p}{)}
    \PYG{n}{pareto\PYGZus{}ghg\PYGZus{}lcoe} \PYG{o}{=} \PYG{n}{pd}\PYG{o}{.}\PYG{n}{DataFrame}\PYG{p}{(}
        \PYG{p}{[}\PYG{p}{[}\PYG{n}{investment\PYGZus{}max}\PYG{p}{,} \PYG{n}{lcoe\PYGZus{}min}\PYG{p}{,} \PYG{n}{optimum}\PYG{o}{.}\PYG{n}{metrics}\PYG{p}{[}\PYG{l+s+s2}{\PYGZdq{}}\PYG{l+s+s2}{LCOE}\PYG{l+s+s2}{\PYGZdq{}}\PYG{p}{]}\PYG{p}{,} \PYG{n}{optimum}\PYG{o}{.}\PYG{n}{metrics}\PYG{p}{[}\PYG{l+s+s2}{\PYGZdq{}}\PYG{l+s+s2}{GHG}\PYG{l+s+s2}{\PYGZdq{}}\PYG{p}{]}\PYG{p}{,} \PYG{n}{optimum}\PYG{o}{.}\PYG{n}{exit\PYGZus{}message}\PYG{p}{]}\PYG{p}{]}\PYG{p}{,}
        \PYG{n}{columns} \PYG{o}{=} \PYG{p}{[}\PYG{l+s+s2}{\PYGZdq{}}\PYG{l+s+s2}{Investment [M\PYGZdl{}]}\PYG{l+s+s2}{\PYGZdq{}}\PYG{p}{,} \PYG{l+s+s2}{\PYGZdq{}}\PYG{l+s+s2}{LCOE (min)}\PYG{l+s+s2}{\PYGZdq{}}\PYG{p}{,} \PYG{l+s+s2}{\PYGZdq{}}\PYG{l+s+s2}{LCOE}\PYG{l+s+s2}{\PYGZdq{}}\PYG{p}{,} \PYG{l+s+s2}{\PYGZdq{}}\PYG{l+s+s2}{GHG}\PYG{l+s+s2}{\PYGZdq{}}\PYG{p}{,} \PYG{l+s+s2}{\PYGZdq{}}\PYG{l+s+s2}{Result}\PYG{l+s+s2}{\PYGZdq{}}\PYG{p}{]}                               \PYG{p}{,}
    \PYG{p}{)}\PYG{o}{.}\PYG{n}{append}\PYG{p}{(}\PYG{n}{pareto\PYGZus{}ghg\PYGZus{}lcoe}\PYG{p}{)}
\PYG{n}{pareto\PYGZus{}ghg\PYGZus{}lcoe} \PYG{o}{=} \PYG{n}{pareto\PYGZus{}ghg\PYGZus{}lcoe}\PYG{o}{.}\PYG{n}{set\PYGZus{}index}\PYG{p}{(}\PYG{p}{[}\PYG{l+s+s2}{\PYGZdq{}}\PYG{l+s+s2}{Investment [M\PYGZdl{}]}\PYG{l+s+s2}{\PYGZdq{}}\PYG{p}{,} \PYG{l+s+s2}{\PYGZdq{}}\PYG{l+s+s2}{LCOE (min)}\PYG{l+s+s2}{\PYGZdq{}}\PYG{p}{]}\PYG{p}{)}
\PYG{n}{pareto\PYGZus{}ghg\PYGZus{}lcoe}
\end{sphinxVerbatim}



\begin{sphinxVerbatim}[commandchars=\\\{\}]
\PYG{n}{sb}\PYG{o}{.}\PYG{n}{relplot}\PYG{p}{(}
    \PYG{n}{x} \PYG{o}{=} \PYG{l+s+s2}{\PYGZdq{}}\PYG{l+s+s2}{LCOE}\PYG{l+s+s2}{\PYGZdq{}}\PYG{p}{,}
    \PYG{n}{y} \PYG{o}{=} \PYG{l+s+s2}{\PYGZdq{}}\PYG{l+s+s2}{GHG}\PYG{l+s+s2}{\PYGZdq{}}\PYG{p}{,}
    \PYG{n}{kind} \PYG{o}{=} \PYG{l+s+s2}{\PYGZdq{}}\PYG{l+s+s2}{scatter}\PYG{l+s+s2}{\PYGZdq{}}\PYG{p}{,}
    \PYG{n}{data} \PYG{o}{=} \PYG{n}{pareto\PYGZus{}ghg\PYGZus{}lcoe}\PYG{c+c1}{\PYGZsh{}[pareto\PYGZus{}ghg\PYGZus{}lcoe.Result == \PYGZdq{}Optimization terminated successfully.\PYGZdq{}]}
\PYG{p}{)}
\end{sphinxVerbatim}

\begin{sphinxVerbatim}[commandchars=\\\{\}]
\PYG{o}{\PYGZlt{}}\PYG{n}{seaborn}\PYG{o}{.}\PYG{n}{axisgrid}\PYG{o}{.}\PYG{n}{FacetGrid} \PYG{n}{at} \PYG{l+m+mh}{0x7f9da13ae630}\PYG{o}{\PYGZgt{}}
\end{sphinxVerbatim}

\noindent\sphinxincludegraphics{{output_112_1}.png}

\sphinxAtStartPar
\sphinxstyleemphasis{The three types of investment are too decoupled to make an interesting
pareto frontier, and we also need a better solver if we want to push to
lower right.}


\section{Run the interactive explorer for the decision space.}
\label{\detokenize{example-analysis:run-the-interactive-explorer-for-the-decision-space}}
\sphinxAtStartPar
Make sure the the \sphinxcode{\sphinxupquote{tk}} package is installed on your machine. Here is
the Anaconda link: \sphinxurl{https://anaconda.org/anaconda/tk}.

\begin{sphinxVerbatim}[commandchars=\\\{\}]
\PYG{n}{w} \PYG{o}{=} \PYG{n}{ty}\PYG{o}{.}\PYG{n}{DecisionWindow}\PYG{p}{(}\PYG{n}{evaluator}\PYG{p}{)}
\PYG{n}{w}\PYG{o}{.}\PYG{n}{mainloop}\PYG{p}{(}\PYG{p}{)}
\end{sphinxVerbatim}

\sphinxAtStartPar
A new window should open that looks like the image below. Moving the
sliders will cause a recomputation of the boxplots.

\begin{sphinxVerbatim}[commandchars=\\\{\}]
\PYG{n}{Image}\PYG{p}{(}\PYG{l+s+s2}{\PYGZdq{}}\PYG{l+s+s2}{pv\PYGZus{}residential\PYGZus{}simple\PYGZus{}gui.png}\PYG{l+s+s2}{\PYGZdq{}}\PYG{p}{)}
\end{sphinxVerbatim}

\noindent\sphinxincludegraphics{{output_118_0}.png}

\sphinxstepscope


\chapter{Approach}
\label{\detokenize{approach:approach}}\label{\detokenize{approach:sec-approach}}\label{\detokenize{approach::doc}}
\sphinxAtStartPar
Our production\sphinxhyphen{}function approach to R\&D portfolio evaluation is
mathematically formulated as a stochastic multi\sphinxhyphen{}objective
decision\sphinxhyphen{}optimization problem and is implemented in the Python
programming language. The framework abstracts the technology\sphinxhyphen{}independent
aspects of the problem into a generic computational schema and enables
the modeler to specify the technology\sphinxhyphen{}dependent aspects in a set of data
tables and Python functions. This approach not only minimizes the labor
needed to add new technologies, but it also enforces uniformity of
financial, mass\sphinxhyphen{}balance, and other assumptions in the analysis.

\sphinxAtStartPar
The framework is scalable, supporting rapid computation on laptop
computers and large\sphinxhyphen{}ensemble studies on high\sphinxhyphen{}performance computers (HPC).
The use of vectorized operations for the stochastic calculations and of
response\sphinxhyphen{}surface fits for the portfolio evaluations minimizes the
computational resources needed for complex multi\sphinxhyphen{}objective
optimizations. The software handles parameterized studies such as
tornado plots, Monte\sphinxhyphen{}Carlo sensitivity analyses, and a generalization of
epsilon\sphinxhyphen{}constraint optimization.

\sphinxAtStartPar
All values in the data tables may be probability distributions,
specified by Python expressions using a large library of standard
distributions, or the values may be simple numbers. Expert opinion is
encoded through these distributions. The opinions may be combined prior
to simulation or subsequent to it.

\sphinxAtStartPar
Four example technologies have been implemented as examples illustrating the
framework’s use: biorefineries, electrolysis, residential photovoltaics
(PV), and utility\sphinxhyphen{}scale PV. A desktop user interface allows exploration
of the cost\sphinxhyphen{}benefit trade\sphinxhyphen{}offs in portfolio decision problems.

\sphinxAtStartPar
Below we detail the mathematical formulation and its implementation as a
Python module with user\sphinxhyphen{}specified data tables and technology functions.

\sphinxAtStartPar
We also provide a sample analysis that exercises the framework’s main
features.

\sphinxstepscope


\chapter{Mathematical Formulation}
\label{\detokenize{formulation:mathematical-formulation}}\label{\detokenize{formulation:sec-formulation}}\label{\detokenize{formulation::doc}}
\sphinxAtStartPar
We separate the financial and conversion\sphinxhyphen{}efficiency aspects of a
production process, which are generic across all technologies, from the
physical and technical aspects, which are necessarily specific to the
particular process. The motivation for this is that the financial and
waste computations can be done uniformly for any technology (even for
disparate ones such as PV cells and biofuels) and that different experts
may be required to assess the cost, waste, and techno\sphinxhyphen{}physical aspects
of technological progress. \hyperref[\detokenize{formulation:tbl-sets}]{Table \ref{\detokenize{formulation:tbl-sets}}} defines the indices that are used
for the variables that are defined in \hyperref[\detokenize{formulation:tbl-variables}]{Table \ref{\detokenize{formulation:tbl-variables}}}.


\begin{savenotes}\sphinxattablestart
\centering
\sphinxcapstartof{table}
\sphinxthecaptionisattop
\sphinxcaption{Definitions for set indices used for variable subscripts.}\label{\detokenize{formulation:id1}}\label{\detokenize{formulation:tbl-sets}}
\sphinxaftertopcaption
\begin{tabulary}{\linewidth}[t]{|T|T|T|}
\hline
\sphinxstyletheadfamily 
\sphinxAtStartPar
Set
&\sphinxstyletheadfamily 
\sphinxAtStartPar
Description
&\sphinxstyletheadfamily 
\sphinxAtStartPar
Examples
\\
\hline
\sphinxAtStartPar
\(c \in \mathcal{C}\)
&
\sphinxAtStartPar
capital
&
\sphinxAtStartPar
equipment
\\
\hline
\sphinxAtStartPar
\(f \in \mathcal{F}\)
&
\sphinxAtStartPar
fixed cost
&
\sphinxAtStartPar
rent, insurance
\\
\hline
\sphinxAtStartPar
\(i \in \mathcal{I}\)
&
\sphinxAtStartPar
input
&
\sphinxAtStartPar
feedstock, labor
\\
\hline
\sphinxAtStartPar
\(o \in \mathcal{O}\)
&
\sphinxAtStartPar
output
&
\sphinxAtStartPar
product, co\sphinxhyphen{}product, waste
\\
\hline
\sphinxAtStartPar
\(m \in \mathcal{M}\)
&
\sphinxAtStartPar
metric
&
\sphinxAtStartPar
cost, jobs, carbon footprint, efficiency, lifetime
\\
\hline
\sphinxAtStartPar
\(p \in \mathcal{P}\)
&
\sphinxAtStartPar
technical parameter
&
\sphinxAtStartPar
temperature, pressure
\\
\hline
\sphinxAtStartPar
\(\nu \in N\)
&
\sphinxAtStartPar
technology type
&
\sphinxAtStartPar
electrolysis, PV cell
\\
\hline
\sphinxAtStartPar
\(\theta \in \Theta\)
&
\sphinxAtStartPar
scenario
&
\sphinxAtStartPar
the result of a particular investment
\\
\hline
\sphinxAtStartPar
\(\chi \in X\)
&
\sphinxAtStartPar
investment category
&
\sphinxAtStartPar
investment alternatives
\\
\hline
\sphinxAtStartPar
\(\phi \in \Phi_\chi\)
&
\sphinxAtStartPar
investment
&
\sphinxAtStartPar
a particular investment
\\
\hline
\sphinxAtStartPar
\(\omega \in \Omega\)
&
\sphinxAtStartPar
portfolio
&
\sphinxAtStartPar
a basket of investments
\\
\hline
\end{tabulary}
\par
\sphinxattableend\end{savenotes}


\begin{savenotes}\sphinxatlongtablestart\begin{longtable}[c]{|l|l|l|l|}
\sphinxthelongtablecaptionisattop
\caption{Definitions for variables.\strut}\label{\detokenize{formulation:id2}}\label{\detokenize{formulation:tbl-variables}}\\*[\sphinxlongtablecapskipadjust]
\hline
\sphinxstyletheadfamily 
\sphinxAtStartPar
Variable
&\sphinxstyletheadfamily 
\sphinxAtStartPar
Type
&\sphinxstyletheadfamily 
\sphinxAtStartPar
Description
&\sphinxstyletheadfamily 
\sphinxAtStartPar
Units
\\
\hline
\endfirsthead

\multicolumn{4}{c}%
{\makebox[0pt]{\sphinxtablecontinued{\tablename\ \thetable{} \textendash{} continued from previous page}}}\\
\hline
\sphinxstyletheadfamily 
\sphinxAtStartPar
Variable
&\sphinxstyletheadfamily 
\sphinxAtStartPar
Type
&\sphinxstyletheadfamily 
\sphinxAtStartPar
Description
&\sphinxstyletheadfamily 
\sphinxAtStartPar
Units
\\
\hline
\endhead

\hline
\multicolumn{4}{r}{\makebox[0pt][r]{\sphinxtablecontinued{continues on next page}}}\\
\endfoot

\endlastfoot

\sphinxAtStartPar
\(K\)
&
\sphinxAtStartPar
calculated
&
\sphinxAtStartPar
unit cost
&
\sphinxAtStartPar
USD/unit
\\
\hline
\sphinxAtStartPar
\(C_c\)
&
\sphinxAtStartPar
function
&
\sphinxAtStartPar
capital cost
&
\sphinxAtStartPar
USD
\\
\hline
\sphinxAtStartPar
\(\tau_c\)
&
\sphinxAtStartPar
cost
&
\sphinxAtStartPar
lifetime of capital
&
\sphinxAtStartPar
year
\\
\hline
\sphinxAtStartPar
\(S\)
&
\sphinxAtStartPar
cost
&
\sphinxAtStartPar
scale of operation
&
\sphinxAtStartPar
unit/year
\\
\hline
\sphinxAtStartPar
\(F_f\)
&
\sphinxAtStartPar
function
&
\sphinxAtStartPar
fixed cost
&
\sphinxAtStartPar
USD/year
\\
\hline
\sphinxAtStartPar
\(I_i\)
&
\sphinxAtStartPar
input
&
\sphinxAtStartPar
input quantity
&
\sphinxAtStartPar
input/unit
\\
\hline
\sphinxAtStartPar
\(I^*_i\)
&
\sphinxAtStartPar
calculated
&
\sphinxAtStartPar
ideal input quantity
&
\sphinxAtStartPar
input/unit
\\
\hline
\sphinxAtStartPar
\(\eta_i\)
&
\sphinxAtStartPar
waste
&
\sphinxAtStartPar
input efficiency
&
\sphinxAtStartPar
input/input
\\
\hline
\sphinxAtStartPar
\(p_i\)
&
\sphinxAtStartPar
cost
&
\sphinxAtStartPar
input price
&
\sphinxAtStartPar
USD/input
\\
\hline
\sphinxAtStartPar
\(O_o\)
&
\sphinxAtStartPar
calculated
&
\sphinxAtStartPar
output quantity
&
\sphinxAtStartPar
output/unit
\\
\hline
\sphinxAtStartPar
\(O^*_o\)
&
\sphinxAtStartPar
calculated
&
\sphinxAtStartPar
ideal output quantity
&
\sphinxAtStartPar
output/unit
\\
\hline
\sphinxAtStartPar
\(\eta^\prime_o\)
&
\sphinxAtStartPar
waste
&
\sphinxAtStartPar
output efficiency
&
\sphinxAtStartPar
output/output
\\
\hline
\sphinxAtStartPar
\(p^\prime_o\)
&
\sphinxAtStartPar
cost
&
\sphinxAtStartPar
output price (+/\sphinxhyphen{})
&
\sphinxAtStartPar
USD/output
\\
\hline
\sphinxAtStartPar
\(\mu_m\)
&
\sphinxAtStartPar
calculated
&
\sphinxAtStartPar
metric
&
\sphinxAtStartPar
metric/unit
\\
\hline
\sphinxAtStartPar
\(P_o\)
&
\sphinxAtStartPar
function
&
\sphinxAtStartPar
production function
&
\sphinxAtStartPar
output/unit
\\
\hline
\sphinxAtStartPar
\(M_m\)
&
\sphinxAtStartPar
function
&
\sphinxAtStartPar
metric function
&
\sphinxAtStartPar
metric/unit
\\
\hline
\sphinxAtStartPar
\(\alpha_p\)
&
\sphinxAtStartPar
parameter
&
\sphinxAtStartPar
technical parameter
&
\sphinxAtStartPar
(mixed)
\\
\hline
\sphinxAtStartPar
\(\xi_\theta\)
&
\sphinxAtStartPar
variable
&
\sphinxAtStartPar
scenario inputs
&
\sphinxAtStartPar
(mixed)
\\
\hline
\sphinxAtStartPar
\(\zeta_\theta\)
&
\sphinxAtStartPar
variable
&
\sphinxAtStartPar
scenario outputs
&
\sphinxAtStartPar
(mixed)
\\
\hline
\sphinxAtStartPar
\(\psi\)
&
\sphinxAtStartPar
function
&
\sphinxAtStartPar
scenario evaluation
&
\sphinxAtStartPar
(mixed)
\\
\hline
\sphinxAtStartPar
\(\sigma_\phi\)
&
\sphinxAtStartPar
function
&
\sphinxAtStartPar
scenario probability
&
\sphinxAtStartPar
1
\\
\hline
\sphinxAtStartPar
\(q_\phi\)
&
\sphinxAtStartPar
variable
&
\sphinxAtStartPar
investment cost
&
\sphinxAtStartPar
USD
\\
\hline
\sphinxAtStartPar
\(\mathbf{\zeta}_\phi\)
&
\sphinxAtStartPar
random variable
&
\sphinxAtStartPar
investment outcome
&
\sphinxAtStartPar
(mixed)
\\
\hline
\sphinxAtStartPar
\(\mathbf{Z}(\omega)\)
&
\sphinxAtStartPar
random variable
&
\sphinxAtStartPar
portfolio outcome
&
\sphinxAtStartPar
(mixed)
\\
\hline
\sphinxAtStartPar
\(Q(\omega)\)
&
\sphinxAtStartPar
calculated
&
\sphinxAtStartPar
portfolio cost
&
\sphinxAtStartPar
USD
\\
\hline
\sphinxAtStartPar
\(Q^\mathrm{min}\)
&
\sphinxAtStartPar
parameter
&
\sphinxAtStartPar
minimum portfolio cost
&
\sphinxAtStartPar
USD
\\
\hline
\sphinxAtStartPar
\(Q^\mathrm{max}\)
&
\sphinxAtStartPar
parameter
&
\sphinxAtStartPar
maximum portfolio cost
&
\sphinxAtStartPar
USD
\\
\hline
\sphinxAtStartPar
\(q^\mathrm{min}_\phi\)
&
\sphinxAtStartPar
parameter
&
\sphinxAtStartPar
minimum category cost
&
\sphinxAtStartPar
USD
\\
\hline
\sphinxAtStartPar
\(q^\mathrm{max}_\phi\)
&
\sphinxAtStartPar
parameter
&
\sphinxAtStartPar
maximum category cost
&
\sphinxAtStartPar
USD
\\
\hline
\sphinxAtStartPar
\(Z^\mathrm{min}\)
&
\sphinxAtStartPar
parameter
&
\sphinxAtStartPar
minimum output/metric
&
\sphinxAtStartPar
(mixed)
\\
\hline
\sphinxAtStartPar
\(Z^\mathrm{max}\)
&
\sphinxAtStartPar
parameter
&
\sphinxAtStartPar
maximum output/metric
&
\sphinxAtStartPar
(mixed)
\\
\hline
\sphinxAtStartPar
\(\mathbb{F}\), \(\mathbb{G}\)
&
\sphinxAtStartPar
operator
&
\sphinxAtStartPar
evaluate probabilities
&
\sphinxAtStartPar
(mixed)
\\
\hline
\end{longtable}\sphinxatlongtableend\end{savenotes}


\section{Cost}
\label{\detokenize{formulation:cost}}
\sphinxAtStartPar
The cost characterizations (capital and fixed costs) are represented as
functions of the scale of operations and of the technical parameters in
the design:
\begin{itemize}
\item {} 
\sphinxAtStartPar
Capital cost: \(C_c(S, \alpha_p)\).

\item {} 
\sphinxAtStartPar
Fixed cost: \(F_f(S, \alpha_p)\).

\end{itemize}

\sphinxAtStartPar
The per\sphinxhyphen{}unit cost is computed using a simple levelization formula:

\sphinxAtStartPar
\(K = \left( \sum_c C_c / \tau_c + \sum_f F_f \right) / S + \sum_i p_i \cdot I_i - \sum_o p^\prime_o \cdot O_o\)


\section{Waste}
\label{\detokenize{formulation:waste}}
\sphinxAtStartPar
The waste relative to the idealized production process is captured by
the \(\eta\) parameters. Expert elicitation might estimate how the
\(\eta\)s would change in response to R\&D investment.
\begin{itemize}
\item {} 
\sphinxAtStartPar
Waste of input: \(I^*_i = \eta_i I_i\).

\item {} 
\sphinxAtStartPar
Waste of output: \(O_o = \eta^\prime_o O^*_o\).

\end{itemize}


\section{Production}
\label{\detokenize{formulation:production}}
\sphinxAtStartPar
The production function idealizes production by ignoring waste, but
accounting for physical and technical processes (e.g., stoichiometry).
This requires a technical model or a tabulation/fit of the results of
technical modeling.

\sphinxAtStartPar
\(O^*_o = P_o(S, C_c, \tau_c, F_f, I^*_i, \alpha_p)\)


\section{Metrics}
\label{\detokenize{formulation:metrics}}
\sphinxAtStartPar
Metrics such as efficiency, lifetime, or carbon footprint are also
compute based on the physical and technical characteristics of the
process. This requires a technical model or a tabulation/fit of the
results of technical modeling. We use the convention that higher values
are worse and lower values are better.

\sphinxAtStartPar
\(\mu_m = M_m(S, C_c, \tau_c, F_f, I_i, I^*_i, O^*_o, O_o, K, \alpha_p)\)


\section{Scenarios}
\label{\detokenize{formulation:scenarios}}
\sphinxAtStartPar
A \sphinxstyleemphasis{scenario} represents a state of affairs for a technology \(\nu\).
If we denote the scenario as \(\theta\), we have the tuple of input
variables

\sphinxAtStartPar
\(\xi_\theta = \left(S, C_c, \tau_c, F_f, I_i, \eta_i, \eta^\prime_o, \alpha_p, p_i, p^\prime_o\middle) \right|_\theta\)

\sphinxAtStartPar
and the tuple of output variables

\sphinxAtStartPar
\(\zeta_\theta = \left(K, I^*_i, O^*_o, O_o, \mu_m\middle) \right|_\theta\)

\sphinxAtStartPar
and their relationship

\sphinxAtStartPar
\(\zeta_\theta = \psi_\nu\left(\xi_\theta\middle) \right|_{\nu = \nu(\theta)}\)

\sphinxAtStartPar
given the tuple of functions

\sphinxAtStartPar
\(\psi_\nu = \left(P_o, M_m\middle) \right|_\nu\)

\sphinxAtStartPar
for the technology of the scenario.


\section{Investments}
\label{\detokenize{formulation:investments}}
\sphinxAtStartPar
An \sphinxstyleemphasis{investment} \(\phi\) assigns a probability distribution to
scenarios:

\sphinxAtStartPar
\(\sigma_\phi(\theta) = P\left(\theta \middle| \phi\right)\).

\sphinxAtStartPar
such that

\sphinxAtStartPar
\(\int d\theta \sigma_\phi(\theta) = 1\) or
\(\sum_\theta \sigma_\phi(\theta) = 1\),

\sphinxAtStartPar
depending upon whether one is performing the computations discretely or
continuously. Expectations and other measures on probability
distributions can be computed from the \(\sigma_\phi(\theta)\). We
treat the outcome \(\mathbf{\zeta}_\phi\) as a random variable for
the outcomes \(\zeta_\theta\) according to the distribution
\(\sigma_\phi(\theta)\).

\sphinxAtStartPar
Because investment options may be mutually exclusive, as is the case for
investing in the same R\&D at different funding levels, we say
\(\Phi_\chi\) is the set of mutually exclusive investments (i.e.,
only one can occur simultaneously) in investment category \(\chi\):
investments in different categories \(\chi\) can be combined
arbitrarily, but just one investment from each \(\Phi_\chi\) may be
chosen.

\sphinxAtStartPar
Thus the universe of all portfolios is
\(\Omega = \prod_\chi \Phi_\chi\), so a particular portfolio
\(\omega \in \Omega\) has components
\(\phi = \omega_\chi \in \Phi_\chi\). The overall outcome of a
portfolio is a random variable:

\sphinxAtStartPar
\(\mathbf{Z}(\omega) = \sum_\chi \mathbf{\zeta}_\phi \mid_{\phi = \omega_\chi}\)

\sphinxAtStartPar
The cost of an investment in one of the constituents \(\phi\) is
\(q_\phi\), so the cost of a porfolio is:

\sphinxAtStartPar
\(Q(\omega) = \sum_\chi q_\phi \mid_{\phi = \omega_\chi}\)


\section{Decision problem}
\label{\detokenize{formulation:decision-problem}}
\sphinxAtStartPar
The multi\sphinxhyphen{}objective decision problem is

\sphinxAtStartPar
\(\min_{\omega \in \Omega} \  \mathbb{F} \  \mathbf{Z}(\omega)\)

\sphinxAtStartPar
such that

\sphinxAtStartPar
\(Q^\mathrm{min} \leq Q(\omega) \leq Q^\mathrm{max}\) ,

\sphinxAtStartPar
\(q^\mathrm{min}_\phi \leq q_{\phi=\omega_\chi} \leq q^\mathrm{max}_\phi\)
,

\sphinxAtStartPar
\(Z^\mathrm{min} \leq \mathbb{G} \  \mathbf{Z}(\omega) \leq Z^\mathrm{max}\)
,

\sphinxAtStartPar
where \(\mathbb{F}\) and \(\mathbb{G}\) are the expectation
operator \(\mathbb{E}\), the value\sphinxhyphen{}at\sphinxhyphen{}risk, or another operator on
probability spaces. Recall that \(\mathbf{Z}\) is a vector with
components for cost \(K\) and each metric \(\mu_m\), so this is
a multi\sphinxhyphen{}objective problem.

\sphinxAtStartPar
The two\sphinxhyphen{}stage decision problem is a special case of the general problem
outlined here: Each scenario \(\theta\) can be considers as a
composite of one or more stages.


\section{Experts}
\label{\detokenize{formulation:experts}}
\sphinxAtStartPar
Each expert elicitation takes the form of an assessment of the
probability and range (e.g., 10th to 90th percentile) of change in the
cost or waste parameters or the production or metric functions. In
essence, the expert elicitation defines \(\sigma_\phi(\theta)\) for
each potential scenario \(\theta\) of each investment \(\phi\).

\sphinxstepscope


\chapter{Optimization}
\label{\detokenize{optimizers:optimization}}\label{\detokenize{optimizers:sec-optimizers}}\label{\detokenize{optimizers::doc}}

\section{Nonlinear Programming Formulation}
\label{\detokenize{optimizers:nonlinear-programming-formulation}}
\sphinxAtStartPar
Three methods in the \sphinxcode{\sphinxupquote{EpsilonConstraintOptimizer}} class, \sphinxcode{\sphinxupquote{opt\_slsqp}}, \sphinxcode{\sphinxupquote{opt\_shgo}} and \sphinxcode{\sphinxupquote{opt\_diffev}}, are
wrappers for optimization algorithm calls. (The {\hyperref[\detokenize{tyche:sec-epsconstraint}]{\sphinxcrossref{\DUrole{std,std-ref}{tyche.EpsilonConstraints}}}} section provides full parameter and return value information for these methods.) The optimization methods define the optimization problem according to each algorithm’s requirements, call the algorithm, and provide either optimized results in a standard format for postprocessing, or an error messages if the optimization did not complete successfully. The SLSQP algorithm, which is not a global optimizer, is provided to assess problem feasibility and provide reasonable upper and lower bounds on metrics being optimized. Because the technology models within an R\&D decision context may be arbitrarily complex, two global optimization algorithms were also implemented. The global algorithms were chosen according to the following criteria.
\begin{itemize}
\item {} 
\sphinxAtStartPar
Ability to perform constrained optimization with inequality constraints, for instance on metric values or investment amounts.

\item {} 
\sphinxAtStartPar
Ability to optimize without specified Jacobian or Hessian functions (derivative\sphinxhyphen{}less optimization).

\item {} 
\sphinxAtStartPar
Ability to specify bounds on individual decision variables, which allows constraints on single research areas.

\item {} 
\sphinxAtStartPar
Ability to work on a variety of potentially non\sphinxhyphen{}convex and otherwise complex problems.

\end{itemize}


\subsection{Algorithm Testing}
\label{\detokenize{optimizers:algorithm-testing}}
\sphinxAtStartPar
As a point of comparison between algorithms, the {\hyperref[\detokenize{technology:sec-simplerespv}]{\sphinxcrossref{\DUrole{std,std-ref}{Simple Residential Photovoltaics}}}} decision context was optimized for minimum levelized cost of electricity (LCOE) subject to an investment constraint and a metric constraint (GHG). The solutions are given in \hyperref[\detokenize{optimizers:tbl-algoperformance}]{Table \ref{\detokenize{optimizers:tbl-algoperformance}}}. The solve times listed are in addition to the time required to set up the problem and solve for the optimum metric values; this procedure currently uses the SLSQP algorithm by default. This setup time is between 10 and 15 seconds.


\begin{savenotes}\sphinxattablestart
\centering
\sphinxcapstartof{table}
\sphinxthecaptionisattop
\sphinxcaption{Minimizing LCOE subject to a total investment amount of \$3 MM USD and GHG being at least 40.}\label{\detokenize{optimizers:id1}}\label{\detokenize{optimizers:tbl-algoperformance}}
\sphinxaftertopcaption
\begin{tabulary}{\linewidth}[t]{|T|T|T|T|}
\hline
\sphinxstyletheadfamily 
\sphinxAtStartPar
Algorithm
&\sphinxstyletheadfamily 
\sphinxAtStartPar
Objective Function Value
&\sphinxstyletheadfamily 
\sphinxAtStartPar
GHG Constraint Value
&\sphinxstyletheadfamily 
\sphinxAtStartPar
Solve Time (s)
\\
\hline
\sphinxAtStartPar
Differential evolution
&
\sphinxAtStartPar
0.037567
&
\sphinxAtStartPar
41.699885
&
\sphinxAtStartPar
145
\\
\hline
\sphinxAtStartPar
Differential evolution
&
\sphinxAtStartPar
0.037547
&
\sphinxAtStartPar
41.632867
&
\sphinxAtStartPar
589
\\
\hline
\sphinxAtStartPar
SLSQP
&
\sphinxAtStartPar
0.037712
&
\sphinxAtStartPar
41.969348
&
\sphinxAtStartPar
\textasciitilde{} 2
\\
\hline
\sphinxAtStartPar
SHGO
&
\sphinxAtStartPar
None found
&
\sphinxAtStartPar
None found
&
\sphinxAtStartPar
None found
\\
\hline
\end{tabulary}
\par
\sphinxattableend\end{savenotes}

\sphinxAtStartPar
Additional details for each solution are given below under the section for the corresponding algorithm.


\section{Sequential Least Squares Programming (SLSQP)}
\label{\detokenize{optimizers:sequential-least-squares-programming-slsqp}}
\sphinxAtStartPar
The Sequential Least Squares Programming algorithm uses a gradient search method to locate a possibly local optimum. {[}6{]} A complete list of parameters and options for the \sphinxcode{\sphinxupquote{fmin\_slsqp}} algorithm is available in the  \sphinxhref{https://docs.scipy.org/doc/scipy/reference/generated/scipy.optimize.fmin\_slsqp.html}{scipy.optimize.fmin\_slsqp} documentation.

\sphinxAtStartPar
Constraints for \sphinxcode{\sphinxupquote{fmin\_slsqp}} are defined either as a single function that takes as input a vector of decision variable values and returns an array containing the value of all constraints in the problem simultaneously. Both equality and inequality constraints can be defined, although they must be as separate functions and are provided to the \sphinxcode{\sphinxupquote{fmin\_slsqp}} algorithm under separate arguments.


\subsection{SLSQP Solution to Simple Residential Photovoltaics}
\label{\detokenize{optimizers:slsqp-solution-to-simple-residential-photovoltaics}}
\sphinxAtStartPar
Solve time: 1.5 s


\begin{savenotes}\sphinxattablestart
\centering
\sphinxcapstartof{table}
\sphinxthecaptionisattop
\sphinxcaption{Optimal decision variables found by the SLSQP algorithm.}\label{\detokenize{optimizers:id2}}\label{\detokenize{optimizers:tbl-slsqpvars}}
\sphinxaftertopcaption
\begin{tabulary}{\linewidth}[t]{|T|T|}
\hline
\sphinxstyletheadfamily 
\sphinxAtStartPar
Decision Variable
&\sphinxstyletheadfamily 
\sphinxAtStartPar
Optimized Value
\\
\hline
\sphinxAtStartPar
BoS R\&D
&
\sphinxAtStartPar
1.25 E\sphinxhyphen{}04
\\
\hline
\sphinxAtStartPar
Inverter R\&D
&
\sphinxAtStartPar
3.64 E\sphinxhyphen{}08
\\
\hline
\sphinxAtStartPar
Module R\&D
&
\sphinxAtStartPar
3.00 E+06
\\
\hline
\end{tabulary}
\par
\sphinxattableend\end{savenotes}


\begin{savenotes}\sphinxattablestart
\centering
\sphinxcapstartof{table}
\sphinxthecaptionisattop
\sphinxcaption{Optimal system metrics found by the SLSQP algorithm.}\label{\detokenize{optimizers:id3}}\label{\detokenize{optimizers:tbl-slsqpmetrics}}
\sphinxaftertopcaption
\begin{tabulary}{\linewidth}[t]{|T|T|}
\hline
\sphinxstyletheadfamily 
\sphinxAtStartPar
System Metric
&\sphinxstyletheadfamily 
\sphinxAtStartPar
Optimized Value
\\
\hline
\sphinxAtStartPar
GHG
&
\sphinxAtStartPar
41.97
\\
\hline
\sphinxAtStartPar
LCOE
&
\sphinxAtStartPar
0.038
\\
\hline
\sphinxAtStartPar
Labor
&
\sphinxAtStartPar
0.032
\\
\hline
\end{tabulary}
\par
\sphinxattableend\end{savenotes}


\section{Differential Evolution}
\label{\detokenize{optimizers:differential-evolution}}
\sphinxAtStartPar
Differential evolution is one type of evolutionary algorithm that iteratively improves on an initial population, or set of potential solutions. {[}5{]} Differential evolution is well\sphinxhyphen{}suited to searching large solution spaces with multiple local minima, but does not guarantee convergence to the global minimum. Moreover, users may need to adjust the default solving parameters and options in order to obtain a solution and cut down on solve time. A complete list of parameters and options for the \sphinxtitleref{differential\_evolution} algorithm is available in the \sphinxhref{https://docs.scipy.org/doc/scipy/reference/generated/scipy.optimize.differential\_evolution.html}{scipy.optimize.differential\_evolution} documentation.

\sphinxAtStartPar
Constraints for \sphinxtitleref{differential\_evolution} are defined by passing the same multi\sphinxhyphen{}valued function defined in \sphinxtitleref{opt\_slsqp} to the \sphinxhref{https://docs.scipy.org/doc/scipy/reference/generated/scipy.optimize.NonlinearConstraint.html}{NonlinearConstraint} method.


\subsection{Differential Evolution Solutions to Simple Residential Photovoltaics}
\label{\detokenize{optimizers:differential-evolution-solutions-to-simple-residential-photovoltaics}}
\sphinxAtStartPar
Differential evolution stochastically populates the initial set of potential solutions, and so the optimal solution and solve time may vary with the random seed used.


\subsubsection{Solution 1}
\label{\detokenize{optimizers:solution-1}}
\sphinxAtStartPar
Starting with a random seed of 2, the solution time was 145 seconds.


\begin{savenotes}\sphinxattablestart
\centering
\sphinxcapstartof{table}
\sphinxthecaptionisattop
\sphinxcaption{Optimal decision variables found by the differential evolution algorithm with a seed of 2.}\label{\detokenize{optimizers:id4}}\label{\detokenize{optimizers:tbl-diffevvars1}}
\sphinxaftertopcaption
\begin{tabulary}{\linewidth}[t]{|T|T|}
\hline
\sphinxstyletheadfamily 
\sphinxAtStartPar
Decision Variable
&\sphinxstyletheadfamily 
\sphinxAtStartPar
Optimized Value
\\
\hline
\sphinxAtStartPar
BoS R\&D
&
\sphinxAtStartPar
9.62 E+02
\\
\hline
\sphinxAtStartPar
Inverter R\&D
&
\sphinxAtStartPar
5.33 E+02
\\
\hline
\sphinxAtStartPar
Module R\&D
&
\sphinxAtStartPar
2.99 E+06
\\
\hline
\end{tabulary}
\par
\sphinxattableend\end{savenotes}


\begin{savenotes}\sphinxattablestart
\centering
\sphinxcapstartof{table}
\sphinxthecaptionisattop
\sphinxcaption{Optimal system metrics found by the differential evolution algorithm with a seed of 2.}\label{\detokenize{optimizers:id5}}\label{\detokenize{optimizers:tbl-diffevmetrics1}}
\sphinxaftertopcaption
\begin{tabulary}{\linewidth}[t]{|T|T|}
\hline
\sphinxstyletheadfamily 
\sphinxAtStartPar
System Metric
&\sphinxstyletheadfamily 
\sphinxAtStartPar
Optimized Value
\\
\hline
\sphinxAtStartPar
GHG
&
\sphinxAtStartPar
41.70
\\
\hline
\sphinxAtStartPar
LCOE
&
\sphinxAtStartPar
0.038
\\
\hline
\sphinxAtStartPar
Labor
&
\sphinxAtStartPar
\sphinxhyphen{}0.456
\\
\hline
\end{tabulary}
\par
\sphinxattableend\end{savenotes}


\subsubsection{Solution 2}
\label{\detokenize{optimizers:solution-2}}
\sphinxAtStartPar
Starting with a random seed of 1, the solution time was 589 seconds.


\begin{savenotes}\sphinxattablestart
\centering
\sphinxcapstartof{table}
\sphinxthecaptionisattop
\sphinxcaption{Optimal decision variables found by \sphinxtitleref{differential\_evolution} as called by \sphinxtitleref{EpsilonConstraints.opt\_diffev} with a seed of 1.}\label{\detokenize{optimizers:id6}}\label{\detokenize{optimizers:tbl-diffevvars2}}
\sphinxaftertopcaption
\begin{tabulary}{\linewidth}[t]{|T|T|}
\hline
\sphinxstyletheadfamily 
\sphinxAtStartPar
Decision Variable
&\sphinxstyletheadfamily 
\sphinxAtStartPar
Optimized Value
\\
\hline
\sphinxAtStartPar
BoS R\&D
&
\sphinxAtStartPar
4.70 E+03
\\
\hline
\sphinxAtStartPar
Inverter R\&D
&
\sphinxAtStartPar
3.71 E+02
\\
\hline
\sphinxAtStartPar
Module R\&D
&
\sphinxAtStartPar
2.99 E+06
\\
\hline
\end{tabulary}
\par
\sphinxattableend\end{savenotes}


\begin{savenotes}\sphinxattablestart
\centering
\sphinxcapstartof{table}
\sphinxthecaptionisattop
\sphinxcaption{Optimal system metrics found by \sphinxtitleref{differential\_evolution} as called by \sphinxtitleref{EpsilonConstraints.opt\_diffev} with a seed of 1.}\label{\detokenize{optimizers:id7}}\label{\detokenize{optimizers:tbl-diffevmetrics2}}
\sphinxaftertopcaption
\begin{tabulary}{\linewidth}[t]{|T|T|}
\hline
\sphinxstyletheadfamily 
\sphinxAtStartPar
System Metric
&\sphinxstyletheadfamily 
\sphinxAtStartPar
Optimized Value
\\
\hline
\sphinxAtStartPar
GHG
&
\sphinxAtStartPar
41.63
\\
\hline
\sphinxAtStartPar
LCOE
&
\sphinxAtStartPar
0.037
\\
\hline
\sphinxAtStartPar
Labor
&
\sphinxAtStartPar
\sphinxhyphen{}2.29
\\
\hline
\end{tabulary}
\par
\sphinxattableend\end{savenotes}


\section{Simplicial Homology Global Optimization}
\label{\detokenize{optimizers:simplicial-homology-global-optimization}}
\sphinxAtStartPar
The Simplicial Homology Global Optimization (SHGO) algorithm applies simplicial homology to general non\sphinxhyphen{}linear, low\sphinxhyphen{}dimensional optimization problems. {[}4{]} SHGO provides fast solutions using default parameters and options, but the optimum found may not be as precise as that found by the differential evolution algorithm. Constraints for \sphinxtitleref{shgo} must be provided as a dictionary or sequence of dictionaries with the following format:

\begin{sphinxVerbatim}[commandchars=\\\{\}]
\PYG{n}{constraints} \PYG{o}{=} \PYG{p}{[} \PYG{p}{\PYGZob{}}\PYG{l+s+s1}{\PYGZsq{}}\PYG{l+s+s1}{type}\PYG{l+s+s1}{\PYGZsq{}}\PYG{p}{:} \PYG{l+s+s1}{\PYGZsq{}}\PYG{l+s+s1}{ineq}\PYG{l+s+s1}{\PYGZsq{}}\PYG{p}{,} \PYG{l+s+s1}{\PYGZsq{}}\PYG{l+s+s1}{fun}\PYG{l+s+s1}{\PYGZsq{}}\PYG{p}{:} \PYG{n}{g1}\PYG{p}{(}\PYG{n}{x}\PYG{p}{)}\PYG{p}{\PYGZcb{}}\PYG{p}{,}
                \PYG{p}{\PYGZob{}}\PYG{l+s+s1}{\PYGZsq{}}\PYG{l+s+s1}{type}\PYG{l+s+s1}{\PYGZsq{}}\PYG{p}{:} \PYG{l+s+s1}{\PYGZsq{}}\PYG{l+s+s1}{ineq}\PYG{l+s+s1}{\PYGZsq{}}\PYG{p}{,} \PYG{l+s+s1}{\PYGZsq{}}\PYG{l+s+s1}{fun}\PYG{l+s+s1}{\PYGZsq{}}\PYG{p}{:} \PYG{n}{g2}\PYG{p}{(}\PYG{n}{x}\PYG{p}{)}\PYG{p}{\PYGZcb{}}\PYG{p}{,}
                \PYG{o}{.}\PYG{o}{.}\PYG{o}{.}
                \PYG{p}{\PYGZob{}}\PYG{l+s+s1}{\PYGZsq{}}\PYG{l+s+s1}{type}\PYG{l+s+s1}{\PYGZsq{}}\PYG{p}{:} \PYG{l+s+s1}{\PYGZsq{}}\PYG{l+s+s1}{eq}\PYG{l+s+s1}{\PYGZsq{}}\PYG{p}{,} \PYG{l+s+s1}{\PYGZsq{}}\PYG{l+s+s1}{fun}\PYG{l+s+s1}{\PYGZsq{}}\PYG{p}{:} \PYG{n}{h1}\PYG{p}{(}\PYG{n}{x}\PYG{p}{)}\PYG{p}{\PYGZcb{}}\PYG{p}{,}
                \PYG{p}{\PYGZob{}}\PYG{l+s+s1}{\PYGZsq{}}\PYG{l+s+s1}{type}\PYG{l+s+s1}{\PYGZsq{}}\PYG{p}{:} \PYG{l+s+s1}{\PYGZsq{}}\PYG{l+s+s1}{eq}\PYG{l+s+s1}{\PYGZsq{}}\PYG{p}{,} \PYG{l+s+s1}{\PYGZsq{}}\PYG{l+s+s1}{fun}\PYG{l+s+s1}{\PYGZsq{}}\PYG{p}{:} \PYG{n}{h2}\PYG{p}{(}\PYG{n}{x}\PYG{p}{)}\PYG{p}{\PYGZcb{}}\PYG{p}{,}
                \PYG{o}{.}\PYG{o}{.}\PYG{o}{.} \PYG{p}{]}
\end{sphinxVerbatim}

\sphinxAtStartPar
Each of the constraint functions \sphinxtitleref{g1(x)}, \sphinxtitleref{h1(x)}, and so on are functions that take decision variable values as inputs and return the value of the constraint. Inequality constraints (\sphinxtitleref{g1(x)} and \sphinxtitleref{g2(x)} above) are formulated as \(g(x) \geq 0\) and equality constraints (\sphinxtitleref{h1(x)} and \sphinxtitleref{h2(x)} above) are formulated as \(h(x) = 0\). Each constraint in the optimization problem is defined as a separate function, with a separate dictionary giving the constraint type. With \sphinxtitleref{shgo} it is not possible to use one function that returns a vector of constraint values.


\section{Piecewise Linear (MILP) Formulation}
\label{\detokenize{optimizers:piecewise-linear-milp-formulation}}

\subsection{Notation}
\label{\detokenize{optimizers:notation}}

\begin{savenotes}\sphinxattablestart
\centering
\sphinxcapstartof{table}
\sphinxthecaptionisattop
\sphinxcaption{Index definitions for the MILP formulation.}\label{\detokenize{optimizers:id8}}\label{\detokenize{optimizers:tbl-milpindex}}
\sphinxaftertopcaption
\begin{tabulary}{\linewidth}[t]{|T|T|}
\hline
\sphinxstyletheadfamily 
\sphinxAtStartPar
Index
&\sphinxstyletheadfamily 
\sphinxAtStartPar
Description
\\
\hline
\sphinxAtStartPar
\(I\)
&
\sphinxAtStartPar
Number of elicited data points (investment levels and metrics)
\\
\hline
\sphinxAtStartPar
\(J\)
&
\sphinxAtStartPar
Number of investment categories
\\
\hline
\sphinxAtStartPar
\(K\)
&
\sphinxAtStartPar
Number of metrics
\\
\hline
\end{tabulary}
\par
\sphinxattableend\end{savenotes}


\begin{savenotes}\sphinxattablestart
\centering
\sphinxcapstartof{table}
\sphinxthecaptionisattop
\sphinxcaption{Data definitions for the MILP formulation.}\label{\detokenize{optimizers:id9}}\label{\detokenize{optimizers:tbl-milpdat}}
\sphinxaftertopcaption
\begin{tabulary}{\linewidth}[t]{|T|T|T|}
\hline
\sphinxstyletheadfamily 
\sphinxAtStartPar
Data
&\sphinxstyletheadfamily 
\sphinxAtStartPar
Notation
&\sphinxstyletheadfamily 
\sphinxAtStartPar
Information
\\
\hline
\sphinxAtStartPar
Investment amounts
&
\sphinxAtStartPar
\(c_{ij}, i \in \{1, ..., I\}\)
&
\sphinxAtStartPar
\(c_i\) is a point in \(J\)\sphinxhyphen{}dimensional space
\\
\hline
\sphinxAtStartPar
Metric value
&
\sphinxAtStartPar
\(q_{ik}, i \in \{1, ..., I \}, k \in \{1, ..., K \}\)
&
\sphinxAtStartPar
One metric will form the objective function, leaving up to \(K-1\) metrics for constraints
\\
\hline
\end{tabulary}
\par
\sphinxattableend\end{savenotes}


\begin{savenotes}\sphinxattablestart
\centering
\sphinxcapstartof{table}
\sphinxthecaptionisattop
\sphinxcaption{Variable definitions for the MILP formulation.}\label{\detokenize{optimizers:id10}}\label{\detokenize{optimizers:tbl-milpvar}}
\sphinxaftertopcaption
\begin{tabulary}{\linewidth}[t]{|T|T|T|}
\hline
\sphinxstyletheadfamily 
\sphinxAtStartPar
Variable
&\sphinxstyletheadfamily 
\sphinxAtStartPar
Notation
&\sphinxstyletheadfamily 
\sphinxAtStartPar
Information
\\
\hline
\sphinxAtStartPar
Binary variables
&
\sphinxAtStartPar
\(y_{ii'}, i, i' \in \{1, ..., I\}, i' > i\)
&
\sphinxAtStartPar
Number of linear intervals between elicited data points.
\\
\hline
\sphinxAtStartPar
Combination variables
&
\sphinxAtStartPar
\(\lambda_{i}, i \in \{1, ..., I\}\)
&
\sphinxAtStartPar
Used to construct linear combinations of elicited data points. \(\lambda_{i} \geq 0 \forall i\)
\\
\hline
\end{tabulary}
\par
\sphinxattableend\end{savenotes}

\sphinxAtStartPar
Each metric and investment amount can be written as a linear combination of elicited data points and the newly introduced variables \(\lambda_{i}\) and \(y_{ii'}\). Additional constraints on \(y_{ii'}\) and \(\lambda_{i}\) take care of the piecewise linearity by ensuring that the corners used to calculate \(q_k\) reflect the interval that \(c_i\) is in. There will be a total of \(\binom{I}{2}\) binary \(y\) variables, which reduces to \(\frac{I(I-1)}{2}\) binary variables.


\subsection{One\sphinxhyphen{}Investment\sphinxhyphen{}Category, One\sphinxhyphen{}Metric Example}
\label{\detokenize{optimizers:one-investment-category-one-metric-example}}
\sphinxAtStartPar
Suppose we have an elicited data set for one metric (\(K = 1\)) and one investment category (\(J = 1\)) with three possible investment levels (\(I = 3\)). We can write the total investment amount as a linear combination of the three investment levels \(c_{i1}, i \in \{1, 2, 3\}\), using the \(\lambda\) variables:

\sphinxAtStartPar
\(\lambda_{1}c_{11} + \lambda_{2}c_{21} + \lambda_{13}c_{31} = \sum_{i} \lambda_{i}c_{i1}\)

\sphinxAtStartPar
We can likewise write the metric as a linear combination of \(q_{1i}\) and the \(\lambda\) variables:

\sphinxAtStartPar
\(\lambda_{1}q_{11} + \lambda_{2}q_{21} + \lambda_{3}q_{31} = \sum_{i} \lambda_{i}q_{i1}\)

\sphinxAtStartPar
We have the additional constraint on the \(\lambda\) variables that

\sphinxAtStartPar
\(\sum_{i} \lambda_{i} = 1\)

\sphinxAtStartPar
These equations, combined with the integer variables \(y_{ii'} = \{ y_{12}, y_{13}, y_{23} \}\), can be used to construct a mixed\sphinxhyphen{}integer linear optimization problem.

\sphinxAtStartPar
The MILP that uses this formulation to minimize a technology metric subject to a investment budget \(B\) is as follows:

\sphinxAtStartPar
\(\min_{y, \lambda} \lambda_{1}q_{11} + \lambda_{2}q_{21} + \lambda_{3}q_{31}\)

\sphinxAtStartPar
subject to

\sphinxAtStartPar
\(\lambda_{1}c_{11} + \lambda_{2}c_{21} + \lambda_{3}c_{31} \leq B\) , (1) Total budget constraint

\sphinxAtStartPar
\(\lambda_1 + \lambda_2 + \lambda_3 = 1\) , (2)

\sphinxAtStartPar
\(y_{12} + y_{23} + y_{13} = 1\) , (3)

\sphinxAtStartPar
\(y_{12} \leq \lambda_1 + \lambda_2\) , (4)

\sphinxAtStartPar
\(y_{23} \leq \lambda_2 + \lambda_3\) , (5)

\sphinxAtStartPar
\(y_{13} \leq \lambda_1 + \lambda_3\) , (6)

\sphinxAtStartPar
\(0 \leq \lambda_1, \lambda_2, \lambda_3 \leq 1\) , (7)

\sphinxAtStartPar
\(y_{12}, y_{23}, y_{13} \in \{ 0, 1 \}\) , (8)

\sphinxAtStartPar
(We’ve effectively removed the investments and the metrics as variables, replacing them with the elicited data points and the new \(\lambda\) and \(y\) variables.)


\subsection{Extension to N x N Problem}
\label{\detokenize{optimizers:extension-to-n-x-n-problem}}
\sphinxAtStartPar
Note: \(k'\) indicates the metric which is being constrained. \(k*\) indicates the metric being optimized. \(J'\) indicates the set of investment categories which have a budget limit (there may be more than one budget\sphinxhyphen{}constrained category in a problem).

\sphinxAtStartPar
\sphinxstylestrong{No metric constraint or investment category\sphinxhyphen{}specific budget constraint}

\sphinxAtStartPar
\(\min_{y, \lambda} \sum_i \lambda_{i}q_{ik*}\)

\sphinxAtStartPar
subject to

\sphinxAtStartPar
\(\sum_i \sum_j \lambda_{i}c_{ij} \leq B\) , (1) Total budget constraint

\sphinxAtStartPar
\(\sum_i \lambda_i = 1\) , (2)

\sphinxAtStartPar
\(\sum_{i,i'} y_{ii'} = 1\) , (3)

\sphinxAtStartPar
\(y_{ii'} \leq \lambda_i + \lambda_{i'} \forall i, i'\) , (4)

\sphinxAtStartPar
\(0 \leq \lambda_i \leq 1 \forall i\) , (5)

\sphinxAtStartPar
\(y_{ii'} \in \{ 0, 1 \} \forall i, i'\) , (6)

\sphinxAtStartPar
\sphinxstylestrong{With investment category\sphinxhyphen{}specific budget constraint}

\sphinxAtStartPar
\(\min_{y, \lambda} \sum_i \lambda_{i}q_{ik*}\)

\sphinxAtStartPar
subject to

\sphinxAtStartPar
\(\sum_i \sum_j \lambda_{i}c_{ij} \leq B\) , (1) Total budget constraint

\sphinxAtStartPar
\(\sum_i \lambda_{i}c_{ij'} \leq B_{j'} \forall j' \in J'\),   (2) Investment category budget constraint(s)

\sphinxAtStartPar
\(\sum_i \lambda_i = 1\) , (3)

\sphinxAtStartPar
\(\sum_{i,i'} y_{ii'} = 1\) , (4)

\sphinxAtStartPar
\(y_{ii'} \leq \lambda_i + \lambda_{i'} \forall i, i'\) , (5)

\sphinxAtStartPar
\(0 \leq \lambda_i \leq 1 \forall i\) , (6)

\sphinxAtStartPar
\(y_{ii'} \in \{ 0, 1 \} \forall i, i'\) , (7)

\sphinxAtStartPar
\sphinxstylestrong{With metric constraint and investment category\sphinxhyphen{}specific budget constraint}

\sphinxAtStartPar
\(\min_{y, \lambda} \sum_i \lambda_{i}q_{ik*}\)

\sphinxAtStartPar
subject to

\sphinxAtStartPar
\(\sum_i \sum_j \lambda_{i}c_{ij} \leq B\), (1) Total budget constraint

\sphinxAtStartPar
\(\sum_i \lambda_{i}c_{ij'} \leq B_{j'} \forall j' \in J'\)   (2) Investment category budget constraint(s)

\sphinxAtStartPar
\(\sum_i \lambda_{i}q_{ik'} \leq M_{k'}\) , (3) Metric constraint

\sphinxAtStartPar
\(\sum_i \lambda_i = 1\) , (4)

\sphinxAtStartPar
\(\sum_{i,i'} y_{ii'} = 1\) , (5)

\sphinxAtStartPar
\(y_{ii'} \leq \lambda_i + \lambda_{i'} \forall i, i'\) , (6)

\sphinxAtStartPar
\(0 \leq \lambda_i \leq 1 \forall i\) , (7)

\sphinxAtStartPar
\(y_{ii'} \in \{ 0, 1 \} \forall i, i'\) , (8)

\sphinxAtStartPar
\sphinxstylestrong{Problem Size}

\sphinxAtStartPar
In general, \(I\) is the number of rows in the dataset of elicited data. In the case that all investment categories have elicited data at the same number of levels (not necessarily the same levels themselves), \(I\) can also be calculated as \(l^J\) where \(l\) is the number of investment levels.

\sphinxAtStartPar
The problem will involve \(\frac{I(I-1)}{2}\) binary variables and \(I\) continuous (\(\lambda\)) variables.


\section{References}
\label{\detokenize{optimizers:references}}\begin{enumerate}
\sphinxsetlistlabels{\arabic}{enumi}{enumii}{}{.}%
\item {} 
\sphinxAtStartPar
\sphinxcode{\sphinxupquote{scipy.optimize.shgo}} SciPy v1.5.4 Reference Guide: Optimization
and root finding (\sphinxcode{\sphinxupquote{scipy.optimize}}) URL:
\sphinxurl{https://docs.scipy.org/doc/scipy/reference/generated/scipy.optimize.shgo.html\#rb2e152d227b3-1}
Last accessed 12/28/2020.

\item {} 
\sphinxAtStartPar
\sphinxcode{\sphinxupquote{scipy.optimize.differential\_evolution}} SciPy v1.5.4 Reference
Guide: Optimization and root finding (\sphinxcode{\sphinxupquote{scipy.optimize}}) URL:
\sphinxurl{https://docs.scipy.org/doc/scipy/reference/generated/scipy.optimize.differential\_evolution.html}
Last accessed 12/28/2020.

\item {} 
\sphinxAtStartPar
\sphinxcode{\sphinxupquote{scipy.optimize.fmin\_slsqp}} SciPy v1.5.4 Reference Guide:
Optimization and root finding (\sphinxcode{\sphinxupquote{scipy.optimize}}) URL:
\sphinxurl{https://docs.scipy.org/doc/scipy/reference/generated/scipy.optimize.fmin\_slsqp.html}
Last accessed 12/28/2020.

\item {} 
\sphinxAtStartPar
Endres, SC, Sandrock, C, Focke, WW. (2018) “A simplicial homology
algorithm for Lipschitz optimisation”, Journal of Global Optimization
(72): 181\sphinxhyphen{}217. URL:
\sphinxurl{https://link.springer.com/article/10.1007/s10898-018-0645-y}

\item {} 
\sphinxAtStartPar
Storn, R and Price, K. (1997) “Differential Evolution \sphinxhyphen{} a Simple and
Efficient Heuristic for Global Optimization over Continuous Spaces”,
Journal of Global Optimization (11): 341 \sphinxhyphen{} 359. URL:
\sphinxurl{https://link.springer.com/article/10.1023/A:1008202821328}

\item {} 
\sphinxAtStartPar
Kraft D (1988) A software package for sequential quadratic
programming. Tech. Rep. DFVLR\sphinxhyphen{}FB 88\sphinxhyphen{}28, DLR German Aerospace Center —
Institute for Flight Mechanics, Koln, Germany.

\item {} 
\sphinxAtStartPar
\sphinxcode{\sphinxupquote{scipy.optimize.NonlinearConstraint}} SciPy v1.5.4 Reference Guide:
Optimization and root finding (\sphinxcode{\sphinxupquote{scipy.optimize}}) URL:
\sphinxurl{https://docs.scipy.org/doc/scipy/reference/generated/scipy.optimize.NonlinearConstraint.html}
Last accessed 12/29/2020.

\end{enumerate}

\sphinxstepscope


\chapter{Python API}
\label{\detokenize{modules:python-api}}\label{\detokenize{modules:sec-modules}}\label{\detokenize{modules::doc}}
\sphinxAtStartPar
The module {\hyperref[\detokenize{tyche:module-tyche}]{\sphinxcrossref{\sphinxcode{\sphinxupquote{tyche}}}}} contains defines and solves multi\sphinxhyphen{}objective R\&D
optimization problems, which the module \sphinxcode{\sphinxupquote{eutychia}} provides a server
of a web\sphinxhyphen{}based user interface. The module {\hyperref[\detokenize{technology:module-technology}]{\sphinxcrossref{\sphinxcode{\sphinxupquote{technology}}}}} definies
individual R\&D technologies.

\sphinxstepscope


\section{Tyche Module}
\label{\detokenize{tyche:tyche-module}}\label{\detokenize{tyche::doc}}

\subsection{tyche.DecisionGUI}
\label{\detokenize{tyche:module-tyche.DecisionGUI}}\label{\detokenize{tyche:tyche-decisiongui}}\index{module@\spxentry{module}!tyche.DecisionGUI@\spxentry{tyche.DecisionGUI}}\index{tyche.DecisionGUI@\spxentry{tyche.DecisionGUI}!module@\spxentry{module}}
\sphinxAtStartPar
Interactive exploration of a technology.
\index{DecisionWindow (class in tyche.DecisionGUI)@\spxentry{DecisionWindow}\spxextra{class in tyche.DecisionGUI}}

\begin{fulllineitems}
\phantomsection\label{\detokenize{tyche:tyche.DecisionGUI.DecisionWindow}}
\pysigstartsignatures
\pysiglinewithargsret{\sphinxbfcode{\sphinxupquote{class\DUrole{w}{  }}}\sphinxcode{\sphinxupquote{tyche.DecisionGUI.}}\sphinxbfcode{\sphinxupquote{DecisionWindow}}}{\emph{\DUrole{n}{evaluator}}}{}
\pysigstopsignatures
\sphinxAtStartPar
Bases: \sphinxcode{\sphinxupquote{object}}

\sphinxAtStartPar
Class for displaying an interactive interface to explore cost\sphinxhyphen{}benefit tradeoffs for a technology.
\index{create\_figure() (tyche.DecisionGUI.DecisionWindow method)@\spxentry{create\_figure()}\spxextra{tyche.DecisionGUI.DecisionWindow method}}

\begin{fulllineitems}
\phantomsection\label{\detokenize{tyche:tyche.DecisionGUI.DecisionWindow.create_figure}}
\pysigstartsignatures
\pysiglinewithargsret{\sphinxbfcode{\sphinxupquote{create\_figure}}}{\emph{\DUrole{n}{i}}, \emph{\DUrole{n}{j}}}{{ $\rightarrow$ Figure}}
\pysigstopsignatures
\end{fulllineitems}

\index{mainloop() (tyche.DecisionGUI.DecisionWindow method)@\spxentry{mainloop()}\spxextra{tyche.DecisionGUI.DecisionWindow method}}

\begin{fulllineitems}
\phantomsection\label{\detokenize{tyche:tyche.DecisionGUI.DecisionWindow.mainloop}}
\pysigstartsignatures
\pysiglinewithargsret{\sphinxbfcode{\sphinxupquote{mainloop}}}{}{}
\pysigstopsignatures
\sphinxAtStartPar
Run the interactive interface.

\end{fulllineitems}

\index{reevaluate() (tyche.DecisionGUI.DecisionWindow method)@\spxentry{reevaluate()}\spxextra{tyche.DecisionGUI.DecisionWindow method}}

\begin{fulllineitems}
\phantomsection\label{\detokenize{tyche:tyche.DecisionGUI.DecisionWindow.reevaluate}}
\pysigstartsignatures
\pysiglinewithargsret{\sphinxbfcode{\sphinxupquote{reevaluate}}}{\emph{\DUrole{n}{next=<function DecisionWindow.<lambda>>}}, \emph{\DUrole{n}{delay=200}}}{}
\pysigstopsignatures
\sphinxAtStartPar
Recalculate the results after a delay.
\begin{quote}\begin{description}
\sphinxlineitem{Parameters}\begin{itemize}
\item {} 
\sphinxAtStartPar
\sphinxstyleliteralstrong{\sphinxupquote{next}} (\sphinxstyleliteralemphasis{\sphinxupquote{function}}) – The operation to perform after completing the recalculation.

\item {} 
\sphinxAtStartPar
\sphinxstyleliteralstrong{\sphinxupquote{delay}} (\sphinxstyleliteralemphasis{\sphinxupquote{int}}) – The number of milliseconds to delay before the recalculation.

\end{itemize}

\end{description}\end{quote}

\end{fulllineitems}

\index{reevaluate\_immediate() (tyche.DecisionGUI.DecisionWindow method)@\spxentry{reevaluate\_immediate()}\spxextra{tyche.DecisionGUI.DecisionWindow method}}

\begin{fulllineitems}
\phantomsection\label{\detokenize{tyche:tyche.DecisionGUI.DecisionWindow.reevaluate_immediate}}
\pysigstartsignatures
\pysiglinewithargsret{\sphinxbfcode{\sphinxupquote{reevaluate\_immediate}}}{\emph{\DUrole{n}{next=<function DecisionWindow.<lambda>>}}}{}
\pysigstopsignatures
\sphinxAtStartPar
Recalculate the results immediately.
\begin{quote}\begin{description}
\sphinxlineitem{Parameters}
\sphinxAtStartPar
\sphinxstyleliteralstrong{\sphinxupquote{next}} (\sphinxstyleliteralemphasis{\sphinxupquote{function}}) – The operation to perform after completing the recalculation.

\end{description}\end{quote}

\end{fulllineitems}

\index{refresh() (tyche.DecisionGUI.DecisionWindow method)@\spxentry{refresh()}\spxextra{tyche.DecisionGUI.DecisionWindow method}}

\begin{fulllineitems}
\phantomsection\label{\detokenize{tyche:tyche.DecisionGUI.DecisionWindow.refresh}}
\pysigstartsignatures
\pysiglinewithargsret{\sphinxbfcode{\sphinxupquote{refresh}}}{}{}
\pysigstopsignatures
\sphinxAtStartPar
Refresh the graphics after a delay.

\end{fulllineitems}

\index{refresh\_immediate() (tyche.DecisionGUI.DecisionWindow method)@\spxentry{refresh\_immediate()}\spxextra{tyche.DecisionGUI.DecisionWindow method}}

\begin{fulllineitems}
\phantomsection\label{\detokenize{tyche:tyche.DecisionGUI.DecisionWindow.refresh_immediate}}
\pysigstartsignatures
\pysiglinewithargsret{\sphinxbfcode{\sphinxupquote{refresh\_immediate}}}{}{}
\pysigstopsignatures
\sphinxAtStartPar
Refresh the graphics immediately.

\end{fulllineitems}


\end{fulllineitems}



\subsection{tyche.Designs}
\label{\detokenize{tyche:module-tyche.Designs}}\label{\detokenize{tyche:tyche-designs}}\index{module@\spxentry{module}!tyche.Designs@\spxentry{tyche.Designs}}\index{tyche.Designs@\spxentry{tyche.Designs}!module@\spxentry{module}}
\sphinxAtStartPar
Designs for technologies.
\index{Designs (class in tyche.Designs)@\spxentry{Designs}\spxextra{class in tyche.Designs}}

\begin{fulllineitems}
\phantomsection\label{\detokenize{tyche:tyche.Designs.Designs}}
\pysigstartsignatures
\pysiglinewithargsret{\sphinxbfcode{\sphinxupquote{class\DUrole{w}{  }}}\sphinxcode{\sphinxupquote{tyche.Designs.}}\sphinxbfcode{\sphinxupquote{Designs}}}{\emph{\DUrole{n}{path}\DUrole{o}{=}\DUrole{default_value}{None}}, \emph{\DUrole{n}{name}\DUrole{o}{=}\DUrole{default_value}{'technology.xlsx'}}, \emph{\DUrole{n}{uncertain}\DUrole{o}{=}\DUrole{default_value}{True}}}{}
\pysigstopsignatures
\sphinxAtStartPar
Bases: \sphinxcode{\sphinxupquote{object}}

\sphinxAtStartPar
Designs for a technology.
\index{indices (tyche.Designs.Designs attribute)@\spxentry{indices}\spxextra{tyche.Designs.Designs attribute}}

\begin{fulllineitems}
\phantomsection\label{\detokenize{tyche:tyche.Designs.Designs.indices}}
\pysigstartsignatures
\pysigline{\sphinxbfcode{\sphinxupquote{indices}}}
\pysigstopsignatures
\sphinxAtStartPar
The \sphinxstyleemphasis{indices} table.
\begin{quote}\begin{description}
\sphinxlineitem{Type}
\sphinxAtStartPar
DataFrame

\end{description}\end{quote}

\end{fulllineitems}

\index{functions (tyche.Designs.Designs attribute)@\spxentry{functions}\spxextra{tyche.Designs.Designs attribute}}

\begin{fulllineitems}
\phantomsection\label{\detokenize{tyche:tyche.Designs.Designs.functions}}
\pysigstartsignatures
\pysigline{\sphinxbfcode{\sphinxupquote{functions}}}
\pysigstopsignatures
\sphinxAtStartPar
The \sphinxstyleemphasis{functions} table.
\begin{quote}\begin{description}
\sphinxlineitem{Type}
\sphinxAtStartPar
DataFrame

\end{description}\end{quote}

\end{fulllineitems}

\index{designs (tyche.Designs.Designs attribute)@\spxentry{designs}\spxextra{tyche.Designs.Designs attribute}}

\begin{fulllineitems}
\phantomsection\label{\detokenize{tyche:tyche.Designs.Designs.designs}}
\pysigstartsignatures
\pysigline{\sphinxbfcode{\sphinxupquote{designs}}}
\pysigstopsignatures
\sphinxAtStartPar
The \sphinxstyleemphasis{designs} table.
\begin{quote}\begin{description}
\sphinxlineitem{Type}
\sphinxAtStartPar
DataFrame

\end{description}\end{quote}

\end{fulllineitems}

\index{parameters (tyche.Designs.Designs attribute)@\spxentry{parameters}\spxextra{tyche.Designs.Designs attribute}}

\begin{fulllineitems}
\phantomsection\label{\detokenize{tyche:tyche.Designs.Designs.parameters}}
\pysigstartsignatures
\pysigline{\sphinxbfcode{\sphinxupquote{parameters}}}
\pysigstopsignatures
\sphinxAtStartPar
The \sphinxstyleemphasis{parameters} table.
\begin{quote}\begin{description}
\sphinxlineitem{Type}
\sphinxAtStartPar
DataFrame

\end{description}\end{quote}

\end{fulllineitems}

\index{results (tyche.Designs.Designs attribute)@\spxentry{results}\spxextra{tyche.Designs.Designs attribute}}

\begin{fulllineitems}
\phantomsection\label{\detokenize{tyche:tyche.Designs.Designs.results}}
\pysigstartsignatures
\pysigline{\sphinxbfcode{\sphinxupquote{results}}}
\pysigstopsignatures
\sphinxAtStartPar
The \sphinxstyleemphasis{results} table.
\begin{quote}\begin{description}
\sphinxlineitem{Type}
\sphinxAtStartPar
DataFrame

\end{description}\end{quote}

\end{fulllineitems}

\index{compile() (tyche.Designs.Designs method)@\spxentry{compile()}\spxextra{tyche.Designs.Designs method}}

\begin{fulllineitems}
\phantomsection\label{\detokenize{tyche:tyche.Designs.Designs.compile}}
\pysigstartsignatures
\pysiglinewithargsret{\sphinxbfcode{\sphinxupquote{compile}}}{}{}
\pysigstopsignatures
\sphinxAtStartPar
Compile the production and metrics functions.

\end{fulllineitems}

\index{evaluate() (tyche.Designs.Designs method)@\spxentry{evaluate()}\spxextra{tyche.Designs.Designs method}}

\begin{fulllineitems}
\phantomsection\label{\detokenize{tyche:tyche.Designs.Designs.evaluate}}
\pysigstartsignatures
\pysiglinewithargsret{\sphinxbfcode{\sphinxupquote{evaluate}}}{\emph{\DUrole{n}{technology}}, \emph{\DUrole{n}{sample\_count}\DUrole{o}{=}\DUrole{default_value}{1}}}{}
\pysigstopsignatures
\sphinxAtStartPar
Evaluate the performance of a technology.
\begin{quote}\begin{description}
\sphinxlineitem{Parameters}\begin{itemize}
\item {} 
\sphinxAtStartPar
\sphinxstyleliteralstrong{\sphinxupquote{technology}} (\sphinxstyleliteralemphasis{\sphinxupquote{str}}) – The name of the technology.

\item {} 
\sphinxAtStartPar
\sphinxstyleliteralstrong{\sphinxupquote{sample\_count}} (\sphinxstyleliteralemphasis{\sphinxupquote{int}}) – The number of random samples.

\end{itemize}

\end{description}\end{quote}

\end{fulllineitems}

\index{evaluate\_scenarios() (tyche.Designs.Designs method)@\spxentry{evaluate\_scenarios()}\spxextra{tyche.Designs.Designs method}}

\begin{fulllineitems}
\phantomsection\label{\detokenize{tyche:tyche.Designs.Designs.evaluate_scenarios}}
\pysigstartsignatures
\pysiglinewithargsret{\sphinxbfcode{\sphinxupquote{evaluate\_scenarios}}}{\emph{\DUrole{n}{sample\_count}\DUrole{o}{=}\DUrole{default_value}{1}}}{}
\pysigstopsignatures
\sphinxAtStartPar
Evaluate scenarios.
\begin{quote}\begin{description}
\sphinxlineitem{Parameters}
\sphinxAtStartPar
\sphinxstyleliteralstrong{\sphinxupquote{sample\_count}} (\sphinxstyleliteralemphasis{\sphinxupquote{int}}) – The number of random samples.

\end{description}\end{quote}

\end{fulllineitems}

\index{vectorize\_designs() (tyche.Designs.Designs method)@\spxentry{vectorize\_designs()}\spxextra{tyche.Designs.Designs method}}

\begin{fulllineitems}
\phantomsection\label{\detokenize{tyche:tyche.Designs.Designs.vectorize_designs}}
\pysigstartsignatures
\pysiglinewithargsret{\sphinxbfcode{\sphinxupquote{vectorize\_designs}}}{\emph{\DUrole{n}{technology}}, \emph{\DUrole{n}{scenario\_count}}, \emph{\DUrole{n}{sample\_count}\DUrole{o}{=}\DUrole{default_value}{1}}}{}
\pysigstopsignatures
\sphinxAtStartPar
Make an array of designs.

\end{fulllineitems}

\index{vectorize\_indices() (tyche.Designs.Designs method)@\spxentry{vectorize\_indices()}\spxextra{tyche.Designs.Designs method}}

\begin{fulllineitems}
\phantomsection\label{\detokenize{tyche:tyche.Designs.Designs.vectorize_indices}}
\pysigstartsignatures
\pysiglinewithargsret{\sphinxbfcode{\sphinxupquote{vectorize\_indices}}}{\emph{\DUrole{n}{technology}}}{}
\pysigstopsignatures
\sphinxAtStartPar
Make an array of indices.

\end{fulllineitems}

\index{vectorize\_parameters() (tyche.Designs.Designs method)@\spxentry{vectorize\_parameters()}\spxextra{tyche.Designs.Designs method}}

\begin{fulllineitems}
\phantomsection\label{\detokenize{tyche:tyche.Designs.Designs.vectorize_parameters}}
\pysigstartsignatures
\pysiglinewithargsret{\sphinxbfcode{\sphinxupquote{vectorize\_parameters}}}{\emph{\DUrole{n}{technology}}, \emph{\DUrole{n}{scenario\_count}}, \emph{\DUrole{n}{sample\_count}\DUrole{o}{=}\DUrole{default_value}{1}}}{}
\pysigstopsignatures
\sphinxAtStartPar
Make an array of parameters.

\end{fulllineitems}

\index{vectorize\_scenarios() (tyche.Designs.Designs method)@\spxentry{vectorize\_scenarios()}\spxextra{tyche.Designs.Designs method}}

\begin{fulllineitems}
\phantomsection\label{\detokenize{tyche:tyche.Designs.Designs.vectorize_scenarios}}
\pysigstartsignatures
\pysiglinewithargsret{\sphinxbfcode{\sphinxupquote{vectorize\_scenarios}}}{\emph{\DUrole{n}{technology}}}{}
\pysigstopsignatures
\sphinxAtStartPar
Make an array of scenarios.

\end{fulllineitems}

\index{vectorize\_technologies() (tyche.Designs.Designs method)@\spxentry{vectorize\_technologies()}\spxextra{tyche.Designs.Designs method}}

\begin{fulllineitems}
\phantomsection\label{\detokenize{tyche:tyche.Designs.Designs.vectorize_technologies}}
\pysigstartsignatures
\pysiglinewithargsret{\sphinxbfcode{\sphinxupquote{vectorize\_technologies}}}{}{}
\pysigstopsignatures
\sphinxAtStartPar
Make an array of technologies.

\end{fulllineitems}


\end{fulllineitems}

\index{sampler() (in module tyche.Designs)@\spxentry{sampler()}\spxextra{in module tyche.Designs}}

\begin{fulllineitems}
\phantomsection\label{\detokenize{tyche:tyche.Designs.sampler}}
\pysigstartsignatures
\pysiglinewithargsret{\sphinxcode{\sphinxupquote{tyche.Designs.}}\sphinxbfcode{\sphinxupquote{sampler}}}{\emph{\DUrole{n}{x}}, \emph{\DUrole{n}{sample\_count}}}{}
\pysigstopsignatures
\sphinxAtStartPar
Sample from an array.
\begin{quote}\begin{description}
\sphinxlineitem{Parameters}\begin{itemize}
\item {} 
\sphinxAtStartPar
\sphinxstyleliteralstrong{\sphinxupquote{x}} (\sphinxstyleliteralemphasis{\sphinxupquote{array}}) – The array.

\item {} 
\sphinxAtStartPar
\sphinxstyleliteralstrong{\sphinxupquote{sample\_count}} (\sphinxstyleliteralemphasis{\sphinxupquote{int}}) – The sample size.

\end{itemize}

\end{description}\end{quote}

\end{fulllineitems}



\subsection{tyche.Distributions}
\label{\detokenize{tyche:module-tyche.Distributions}}\label{\detokenize{tyche:tyche-distributions}}\index{module@\spxentry{module}!tyche.Distributions@\spxentry{tyche.Distributions}}\index{tyche.Distributions@\spxentry{tyche.Distributions}!module@\spxentry{module}}
\sphinxAtStartPar
Utilities for probability distributions.
\index{choice() (in module tyche.Distributions)@\spxentry{choice()}\spxextra{in module tyche.Distributions}}

\begin{fulllineitems}
\phantomsection\label{\detokenize{tyche:tyche.Distributions.choice}}
\pysigstartsignatures
\pysiglinewithargsret{\sphinxcode{\sphinxupquote{tyche.Distributions.}}\sphinxbfcode{\sphinxupquote{choice}}}{\emph{\DUrole{n}{a}}, \emph{\DUrole{n}{size}\DUrole{o}{=}\DUrole{default_value}{None}}, \emph{\DUrole{n}{replace}\DUrole{o}{=}\DUrole{default_value}{True}}, \emph{\DUrole{n}{p}\DUrole{o}{=}\DUrole{default_value}{None}}}{}
\pysigstopsignatures
\sphinxAtStartPar
Generates a random sample from a given 1\sphinxhyphen{}D array

\sphinxAtStartPar
\DUrole{versionmodified,added}{New in version 1.7.0.}

\begin{sphinxadmonition}{note}{Note:}
\sphinxAtStartPar
New code should use the \sphinxtitleref{\textasciitilde{}numpy.random.Generator.choice}
method of a \sphinxtitleref{\textasciitilde{}numpy.random.Generator} instance instead;
please see the \DUrole{xref,std,std-ref}{random\sphinxhyphen{}quick\sphinxhyphen{}start}.
\end{sphinxadmonition}
\begin{quote}\begin{description}
\sphinxlineitem{Parameters}\begin{itemize}
\item {} 
\sphinxAtStartPar
\sphinxstyleliteralstrong{\sphinxupquote{a}} (\sphinxstyleliteralemphasis{\sphinxupquote{1\sphinxhyphen{}D array\sphinxhyphen{}like}}\sphinxstyleliteralemphasis{\sphinxupquote{ or }}\sphinxstyleliteralemphasis{\sphinxupquote{int}}) – If an ndarray, a random sample is generated from its elements.
If an int, the random sample is generated as if it were \sphinxcode{\sphinxupquote{np.arange(a)}}

\item {} 
\sphinxAtStartPar
\sphinxstyleliteralstrong{\sphinxupquote{size}} (\sphinxstyleliteralemphasis{\sphinxupquote{int}}\sphinxstyleliteralemphasis{\sphinxupquote{ or }}\sphinxstyleliteralemphasis{\sphinxupquote{tuple of ints}}\sphinxstyleliteralemphasis{\sphinxupquote{, }}\sphinxstyleliteralemphasis{\sphinxupquote{optional}}) – Output shape.  If the given shape is, e.g., \sphinxcode{\sphinxupquote{(m, n, k)}}, then
\sphinxcode{\sphinxupquote{m * n * k}} samples are drawn.  Default is None, in which case a
single value is returned.

\item {} 
\sphinxAtStartPar
\sphinxstyleliteralstrong{\sphinxupquote{replace}} (\sphinxstyleliteralemphasis{\sphinxupquote{boolean}}\sphinxstyleliteralemphasis{\sphinxupquote{, }}\sphinxstyleliteralemphasis{\sphinxupquote{optional}}) – Whether the sample is with or without replacement. Default is True,
meaning that a value of \sphinxcode{\sphinxupquote{a}} can be selected multiple times.

\item {} 
\sphinxAtStartPar
\sphinxstyleliteralstrong{\sphinxupquote{p}} (\sphinxstyleliteralemphasis{\sphinxupquote{1\sphinxhyphen{}D array\sphinxhyphen{}like}}\sphinxstyleliteralemphasis{\sphinxupquote{, }}\sphinxstyleliteralemphasis{\sphinxupquote{optional}}) – The probabilities associated with each entry in a.
If not given, the sample assumes a uniform distribution over all
entries in \sphinxcode{\sphinxupquote{a}}.

\end{itemize}

\sphinxlineitem{Returns}
\sphinxAtStartPar
\sphinxstylestrong{samples} – The generated random samples

\sphinxlineitem{Return type}
\sphinxAtStartPar
single item or ndarray

\sphinxlineitem{Raises}
\sphinxAtStartPar
\sphinxstyleliteralstrong{\sphinxupquote{ValueError}} – If a is an int and less than zero, if a or p are not 1\sphinxhyphen{}dimensional,
    if a is an array\sphinxhyphen{}like of size 0, if p is not a vector of
    probabilities, if a and p have different lengths, or if
    replace=False and the sample size is greater than the population
    size

\end{description}\end{quote}


\sphinxstrong{See also:}
\nopagebreak


\sphinxAtStartPar
\sphinxcode{\sphinxupquote{randint}}, \sphinxcode{\sphinxupquote{shuffle}}, \sphinxcode{\sphinxupquote{permutation}}
\begin{description}
\sphinxlineitem{\sphinxcode{\sphinxupquote{random.Generator.choice}}}
\sphinxAtStartPar
which should be used in new code

\end{description}


\subsubsection*{Notes}

\sphinxAtStartPar
Setting user\sphinxhyphen{}specified probabilities through \sphinxcode{\sphinxupquote{p}} uses a more general but less
efficient sampler than the default. The general sampler produces a different sample
than the optimized sampler even if each element of \sphinxcode{\sphinxupquote{p}} is 1 / len(a).

\sphinxAtStartPar
Sampling random rows from a 2\sphinxhyphen{}D array is not possible with this function,
but is possible with \sphinxtitleref{Generator.choice} through its \sphinxcode{\sphinxupquote{axis}} keyword.
\subsubsection*{Examples}

\sphinxAtStartPar
Generate a uniform random sample from np.arange(5) of size 3:

\begin{sphinxVerbatim}[commandchars=\\\{\}]
\PYG{g+gp}{\PYGZgt{}\PYGZgt{}\PYGZgt{} }\PYG{n}{np}\PYG{o}{.}\PYG{n}{random}\PYG{o}{.}\PYG{n}{choice}\PYG{p}{(}\PYG{l+m+mi}{5}\PYG{p}{,} \PYG{l+m+mi}{3}\PYG{p}{)}
\PYG{g+go}{array([0, 3, 4]) \PYGZsh{} random}
\PYG{g+gp}{\PYGZgt{}\PYGZgt{}\PYGZgt{} }\PYG{c+c1}{\PYGZsh{}This is equivalent to np.random.randint(0,5,3)}
\end{sphinxVerbatim}

\sphinxAtStartPar
Generate a non\sphinxhyphen{}uniform random sample from np.arange(5) of size 3:

\begin{sphinxVerbatim}[commandchars=\\\{\}]
\PYG{g+gp}{\PYGZgt{}\PYGZgt{}\PYGZgt{} }\PYG{n}{np}\PYG{o}{.}\PYG{n}{random}\PYG{o}{.}\PYG{n}{choice}\PYG{p}{(}\PYG{l+m+mi}{5}\PYG{p}{,} \PYG{l+m+mi}{3}\PYG{p}{,} \PYG{n}{p}\PYG{o}{=}\PYG{p}{[}\PYG{l+m+mf}{0.1}\PYG{p}{,} \PYG{l+m+mi}{0}\PYG{p}{,} \PYG{l+m+mf}{0.3}\PYG{p}{,} \PYG{l+m+mf}{0.6}\PYG{p}{,} \PYG{l+m+mi}{0}\PYG{p}{]}\PYG{p}{)}
\PYG{g+go}{array([3, 3, 0]) \PYGZsh{} random}
\end{sphinxVerbatim}

\sphinxAtStartPar
Generate a uniform random sample from np.arange(5) of size 3 without
replacement:

\begin{sphinxVerbatim}[commandchars=\\\{\}]
\PYG{g+gp}{\PYGZgt{}\PYGZgt{}\PYGZgt{} }\PYG{n}{np}\PYG{o}{.}\PYG{n}{random}\PYG{o}{.}\PYG{n}{choice}\PYG{p}{(}\PYG{l+m+mi}{5}\PYG{p}{,} \PYG{l+m+mi}{3}\PYG{p}{,} \PYG{n}{replace}\PYG{o}{=}\PYG{k+kc}{False}\PYG{p}{)}
\PYG{g+go}{array([3,1,0]) \PYGZsh{} random}
\PYG{g+gp}{\PYGZgt{}\PYGZgt{}\PYGZgt{} }\PYG{c+c1}{\PYGZsh{}This is equivalent to np.random.permutation(np.arange(5))[:3]}
\end{sphinxVerbatim}

\sphinxAtStartPar
Generate a non\sphinxhyphen{}uniform random sample from np.arange(5) of size
3 without replacement:

\begin{sphinxVerbatim}[commandchars=\\\{\}]
\PYG{g+gp}{\PYGZgt{}\PYGZgt{}\PYGZgt{} }\PYG{n}{np}\PYG{o}{.}\PYG{n}{random}\PYG{o}{.}\PYG{n}{choice}\PYG{p}{(}\PYG{l+m+mi}{5}\PYG{p}{,} \PYG{l+m+mi}{3}\PYG{p}{,} \PYG{n}{replace}\PYG{o}{=}\PYG{k+kc}{False}\PYG{p}{,} \PYG{n}{p}\PYG{o}{=}\PYG{p}{[}\PYG{l+m+mf}{0.1}\PYG{p}{,} \PYG{l+m+mi}{0}\PYG{p}{,} \PYG{l+m+mf}{0.3}\PYG{p}{,} \PYG{l+m+mf}{0.6}\PYG{p}{,} \PYG{l+m+mi}{0}\PYG{p}{]}\PYG{p}{)}
\PYG{g+go}{array([2, 3, 0]) \PYGZsh{} random}
\end{sphinxVerbatim}

\sphinxAtStartPar
Any of the above can be repeated with an arbitrary array\sphinxhyphen{}like
instead of just integers. For instance:

\begin{sphinxVerbatim}[commandchars=\\\{\}]
\PYG{g+gp}{\PYGZgt{}\PYGZgt{}\PYGZgt{} }\PYG{n}{aa\PYGZus{}milne\PYGZus{}arr} \PYG{o}{=} \PYG{p}{[}\PYG{l+s+s1}{\PYGZsq{}}\PYG{l+s+s1}{pooh}\PYG{l+s+s1}{\PYGZsq{}}\PYG{p}{,} \PYG{l+s+s1}{\PYGZsq{}}\PYG{l+s+s1}{rabbit}\PYG{l+s+s1}{\PYGZsq{}}\PYG{p}{,} \PYG{l+s+s1}{\PYGZsq{}}\PYG{l+s+s1}{piglet}\PYG{l+s+s1}{\PYGZsq{}}\PYG{p}{,} \PYG{l+s+s1}{\PYGZsq{}}\PYG{l+s+s1}{Christopher}\PYG{l+s+s1}{\PYGZsq{}}\PYG{p}{]}
\PYG{g+gp}{\PYGZgt{}\PYGZgt{}\PYGZgt{} }\PYG{n}{np}\PYG{o}{.}\PYG{n}{random}\PYG{o}{.}\PYG{n}{choice}\PYG{p}{(}\PYG{n}{aa\PYGZus{}milne\PYGZus{}arr}\PYG{p}{,} \PYG{l+m+mi}{5}\PYG{p}{,} \PYG{n}{p}\PYG{o}{=}\PYG{p}{[}\PYG{l+m+mf}{0.5}\PYG{p}{,} \PYG{l+m+mf}{0.1}\PYG{p}{,} \PYG{l+m+mf}{0.1}\PYG{p}{,} \PYG{l+m+mf}{0.3}\PYG{p}{]}\PYG{p}{)}
\PYG{g+go}{array([\PYGZsq{}pooh\PYGZsq{}, \PYGZsq{}pooh\PYGZsq{}, \PYGZsq{}pooh\PYGZsq{}, \PYGZsq{}Christopher\PYGZsq{}, \PYGZsq{}piglet\PYGZsq{}], \PYGZsh{} random}
\PYG{g+go}{      dtype=\PYGZsq{}\PYGZlt{}U11\PYGZsq{})}
\end{sphinxVerbatim}

\end{fulllineitems}

\index{constant() (in module tyche.Distributions)@\spxentry{constant()}\spxextra{in module tyche.Distributions}}

\begin{fulllineitems}
\phantomsection\label{\detokenize{tyche:tyche.Distributions.constant}}
\pysigstartsignatures
\pysiglinewithargsret{\sphinxcode{\sphinxupquote{tyche.Distributions.}}\sphinxbfcode{\sphinxupquote{constant}}}{\emph{\DUrole{n}{value}}}{}
\pysigstopsignatures
\sphinxAtStartPar
The constant distribution.
\begin{quote}\begin{description}
\sphinxlineitem{Parameters}
\sphinxAtStartPar
\sphinxstyleliteralstrong{\sphinxupquote{value}} (\sphinxstyleliteralemphasis{\sphinxupquote{float}}) – The constant value.

\end{description}\end{quote}

\end{fulllineitems}

\index{mixture() (in module tyche.Distributions)@\spxentry{mixture()}\spxextra{in module tyche.Distributions}}

\begin{fulllineitems}
\phantomsection\label{\detokenize{tyche:tyche.Distributions.mixture}}
\pysigstartsignatures
\pysiglinewithargsret{\sphinxcode{\sphinxupquote{tyche.Distributions.}}\sphinxbfcode{\sphinxupquote{mixture}}}{\emph{\DUrole{n}{weights}}, \emph{\DUrole{n}{distributions}}}{}
\pysigstopsignatures
\sphinxAtStartPar
A mixture of two distributions.
\begin{quote}\begin{description}
\sphinxlineitem{Parameters}\begin{itemize}
\item {} 
\sphinxAtStartPar
\sphinxstyleliteralstrong{\sphinxupquote{weights}} (\sphinxstyleliteralemphasis{\sphinxupquote{array of float}}) – The weights of the distributions to be mixed.

\item {} 
\sphinxAtStartPar
\sphinxstyleliteralstrong{\sphinxupquote{distributions}} (\sphinxstyleliteralemphasis{\sphinxupquote{array of distributions}}) – The distributions to be mixed.

\end{itemize}

\end{description}\end{quote}

\end{fulllineitems}

\index{parse\_distribution() (in module tyche.Distributions)@\spxentry{parse\_distribution()}\spxextra{in module tyche.Distributions}}

\begin{fulllineitems}
\phantomsection\label{\detokenize{tyche:tyche.Distributions.parse_distribution}}
\pysigstartsignatures
\pysiglinewithargsret{\sphinxcode{\sphinxupquote{tyche.Distributions.}}\sphinxbfcode{\sphinxupquote{parse\_distribution}}}{\emph{\DUrole{n}{text}}}{}
\pysigstopsignatures
\sphinxAtStartPar
Make the Python object for the distribution, if any is specified.
\begin{quote}\begin{description}
\sphinxlineitem{Parameters}
\sphinxAtStartPar
\sphinxstyleliteralstrong{\sphinxupquote{text}} (\sphinxstyleliteralemphasis{\sphinxupquote{str}}) – The Python expression for the distribution, or plain text.

\end{description}\end{quote}

\end{fulllineitems}



\subsection{tyche.EpsilonConstraints}
\label{\detokenize{tyche:module-tyche.EpsilonConstraints}}\label{\detokenize{tyche:tyche-epsilonconstraints}}\label{\detokenize{tyche:sec-epsconstraint}}\index{module@\spxentry{module}!tyche.EpsilonConstraints@\spxentry{tyche.EpsilonConstraints}}\index{tyche.EpsilonConstraints@\spxentry{tyche.EpsilonConstraints}!module@\spxentry{module}}
\sphinxAtStartPar
Epsilon\sphinxhyphen{}constraint optimization.
\index{EpsilonConstraintOptimizer (class in tyche.EpsilonConstraints)@\spxentry{EpsilonConstraintOptimizer}\spxextra{class in tyche.EpsilonConstraints}}

\begin{fulllineitems}
\phantomsection\label{\detokenize{tyche:tyche.EpsilonConstraints.EpsilonConstraintOptimizer}}
\pysigstartsignatures
\pysiglinewithargsret{\sphinxbfcode{\sphinxupquote{class\DUrole{w}{  }}}\sphinxcode{\sphinxupquote{tyche.EpsilonConstraints.}}\sphinxbfcode{\sphinxupquote{EpsilonConstraintOptimizer}}}{\emph{\DUrole{n}{evaluator}}, \emph{\DUrole{n}{scale}\DUrole{o}{=}\DUrole{default_value}{1000000.0}}}{}
\pysigstopsignatures
\sphinxAtStartPar
Bases: \sphinxcode{\sphinxupquote{object}}

\sphinxAtStartPar
An epsilon\sphinxhyphen{}constration multi\sphinxhyphen{}objective optimizer.
\index{evaluator (tyche.EpsilonConstraints.EpsilonConstraintOptimizer attribute)@\spxentry{evaluator}\spxextra{tyche.EpsilonConstraints.EpsilonConstraintOptimizer attribute}}

\begin{fulllineitems}
\phantomsection\label{\detokenize{tyche:tyche.EpsilonConstraints.EpsilonConstraintOptimizer.evaluator}}
\pysigstartsignatures
\pysigline{\sphinxbfcode{\sphinxupquote{evaluator}}}
\pysigstopsignatures
\sphinxAtStartPar
The technology evaluator.
\begin{quote}\begin{description}
\sphinxlineitem{Type}
\sphinxAtStartPar
tyche.Evaluator

\end{description}\end{quote}

\end{fulllineitems}

\index{scale (tyche.EpsilonConstraints.EpsilonConstraintOptimizer attribute)@\spxentry{scale}\spxextra{tyche.EpsilonConstraints.EpsilonConstraintOptimizer attribute}}

\begin{fulllineitems}
\phantomsection\label{\detokenize{tyche:tyche.EpsilonConstraints.EpsilonConstraintOptimizer.scale}}
\pysigstartsignatures
\pysigline{\sphinxbfcode{\sphinxupquote{scale}}}
\pysigstopsignatures
\sphinxAtStartPar
The scaling factor for output.
\begin{quote}\begin{description}
\sphinxlineitem{Type}
\sphinxAtStartPar
float

\end{description}\end{quote}

\end{fulllineitems}

\index{opt\_diffev() (tyche.EpsilonConstraints.EpsilonConstraintOptimizer method)@\spxentry{opt\_diffev()}\spxextra{tyche.EpsilonConstraints.EpsilonConstraintOptimizer method}}

\begin{fulllineitems}
\phantomsection\label{\detokenize{tyche:tyche.EpsilonConstraints.EpsilonConstraintOptimizer.opt_diffev}}
\pysigstartsignatures
\pysiglinewithargsret{\sphinxbfcode{\sphinxupquote{opt\_diffev}}}{\emph{\DUrole{n}{metric}}, \emph{\DUrole{n}{sense=None}}, \emph{\DUrole{n}{max\_amount=None}}, \emph{\DUrole{n}{total\_amount=None}}, \emph{\DUrole{n}{eps\_metric=None}}, \emph{\DUrole{n}{statistic=<function mean>}}, \emph{\DUrole{n}{strategy='best1bin'}}, \emph{\DUrole{n}{seed=2}}, \emph{\DUrole{n}{tol=0.01}}, \emph{\DUrole{n}{maxiter=75}}, \emph{\DUrole{n}{init='latinhypercube'}}, \emph{\DUrole{n}{verbose=0}}}{}
\pysigstopsignatures
\sphinxAtStartPar
Maximize the objective function using the differential\_evoluaion algorithm.
\begin{quote}\begin{description}
\sphinxlineitem{Parameters}\begin{itemize}
\item {} 
\sphinxAtStartPar
\sphinxstyleliteralstrong{\sphinxupquote{metric}} (\sphinxstyleliteralemphasis{\sphinxupquote{str}}) – Name of metric to maximize. The objective function.

\item {} 
\sphinxAtStartPar
\sphinxstyleliteralstrong{\sphinxupquote{sense}} (\sphinxstyleliteralemphasis{\sphinxupquote{str}}) – \begin{description}
\sphinxlineitem{Optimization sense (‘min’ or ‘max’). If no value is provided to}
\sphinxAtStartPar
this method, the sense value used to create the
EpsilonConstraintOptimizer object is used instead.

\end{description}


\item {} 
\sphinxAtStartPar
\sphinxstyleliteralstrong{\sphinxupquote{max\_amount}} (\sphinxstyleliteralemphasis{\sphinxupquote{Series}}) – Maximum investment amounts by R\&D category. The index (research area)
order must follow that of self.evaluator.max\_amount.index

\item {} 
\sphinxAtStartPar
\sphinxstyleliteralstrong{\sphinxupquote{total\_amount}} (\sphinxstyleliteralemphasis{\sphinxupquote{float}}) – Upper limit on total investments summed across all R\&D categories.

\item {} 
\sphinxAtStartPar
\sphinxstyleliteralstrong{\sphinxupquote{eps\_metric}} (\sphinxstyleliteralemphasis{\sphinxupquote{Dict}}) – RHS of the epsilon constraint(s) on one or more metrics. Keys are metric
names, and the values are dictionaries of the form \{‘limit’: float, ‘sense’: str\}.
The sense defines whether the epsilon constraint is a lower or an upper bound,
and the value must be either ‘upper’ or ‘lower’.

\item {} 
\sphinxAtStartPar
\sphinxstyleliteralstrong{\sphinxupquote{statistic}} (\sphinxstyleliteralemphasis{\sphinxupquote{function}}) – Summary statistic used on the sample evaluations; the metric measure that
is fed to the optimizer.

\item {} 
\sphinxAtStartPar
\sphinxstyleliteralstrong{\sphinxupquote{strategy}} (\sphinxstyleliteralemphasis{\sphinxupquote{str}}) – Which differential evolution strategy to use. ‘best1bin’ is the default.
See algorithm docs for full list.

\item {} 
\sphinxAtStartPar
\sphinxstyleliteralstrong{\sphinxupquote{seed}} (\sphinxstyleliteralemphasis{\sphinxupquote{int}}) – Sets the random seed for optimization by creating a new \sphinxtitleref{RandomState}
instance. Defaults to 1. Not setting this parameter means the solutions
will not be reproducible.

\item {} 
\sphinxAtStartPar
\sphinxstyleliteralstrong{\sphinxupquote{init}} (\sphinxstyleliteralemphasis{\sphinxupquote{str}}\sphinxstyleliteralemphasis{\sphinxupquote{ or }}\sphinxstyleliteralemphasis{\sphinxupquote{array\sphinxhyphen{}like}}) – Type of population initialization. Default is Latin hypercube;
alternatives are ‘random’ or specifying every member of the initial
population in an array of shape (popsize, len(variables)).

\item {} 
\sphinxAtStartPar
\sphinxstyleliteralstrong{\sphinxupquote{tol}} (\sphinxstyleliteralemphasis{\sphinxupquote{float}}) – Relative tolerance for convergence

\item {} 
\sphinxAtStartPar
\sphinxstyleliteralstrong{\sphinxupquote{maxiter}} (\sphinxstyleliteralemphasis{\sphinxupquote{int}}) – Upper limit on generations of evolution (analogous to algorithm
iterations)

\item {} 
\sphinxAtStartPar
\sphinxstyleliteralstrong{\sphinxupquote{verbose}} (\sphinxstyleliteralemphasis{\sphinxupquote{int}}) – Verbosity level returned by this outer function and the
differential\_evolution algorithm.
verbose = 0     No messages
verbose = 1     Objective function value at every algorithm iteration
verbose = 2     Investment constraint status, metric constraint status,
and objective function value
verbose = 3     Decision variable values, investment constraint status,
metric constraint status, and objective function value
verbose > 3     All metric values, decision variable values, investment
constraint status, metric constraint status, and
objective function value

\end{itemize}

\end{description}\end{quote}

\end{fulllineitems}

\index{opt\_milp() (tyche.EpsilonConstraints.EpsilonConstraintOptimizer method)@\spxentry{opt\_milp()}\spxextra{tyche.EpsilonConstraints.EpsilonConstraintOptimizer method}}

\begin{fulllineitems}
\phantomsection\label{\detokenize{tyche:tyche.EpsilonConstraints.EpsilonConstraintOptimizer.opt_milp}}
\pysigstartsignatures
\pysiglinewithargsret{\sphinxbfcode{\sphinxupquote{opt\_milp}}}{\emph{\DUrole{n}{metric}}, \emph{\DUrole{n}{sense=None}}, \emph{\DUrole{n}{max\_amount=None}}, \emph{\DUrole{n}{total\_amount=None}}, \emph{\DUrole{n}{eps\_metric=None}}, \emph{\DUrole{n}{statistic=<function mean>}}, \emph{\DUrole{n}{sizelimit=1000000.0}}, \emph{\DUrole{n}{verbose=0}}}{}
\pysigstopsignatures
\sphinxAtStartPar
Maximize the objective function using a piecewise linear
representation to create a mixed integer linear program.
\begin{quote}\begin{description}
\sphinxlineitem{Parameters}\begin{itemize}
\item {} 
\sphinxAtStartPar
\sphinxstyleliteralstrong{\sphinxupquote{metric}} (\sphinxstyleliteralemphasis{\sphinxupquote{str}}) – Name of metric to maximize

\item {} 
\sphinxAtStartPar
\sphinxstyleliteralstrong{\sphinxupquote{sense}} (\sphinxstyleliteralemphasis{\sphinxupquote{str}}) – Optimization sense (‘min’ or ‘max’). If no value is provided to this
method, the sense value used to create the EpsilonConstraintOptimizer
object is used instead.

\item {} 
\sphinxAtStartPar
\sphinxstyleliteralstrong{\sphinxupquote{max\_amount}} (\sphinxstyleliteralemphasis{\sphinxupquote{Series}}) – Maximum investment amounts by R\&D category. The index (research area)
order must follow that of self.evaluator.max\_amount.index

\item {} 
\sphinxAtStartPar
\sphinxstyleliteralstrong{\sphinxupquote{total\_amount}} (\sphinxstyleliteralemphasis{\sphinxupquote{float}}) – Upper limit on total investments summed across all R\&D categories.

\item {} 
\sphinxAtStartPar
\sphinxstyleliteralstrong{\sphinxupquote{eps\_metric}} (\sphinxstyleliteralemphasis{\sphinxupquote{Dict}}) – RHS of the epsilon constraint(s) on one or more metrics. Keys are metric
names, and the values are dictionaries of the form
\{‘limit’: float, ‘sense’: str\}. The sense defines whether the epsilon
constraint is a lower or an upper bound, and the value must be either
‘upper’ or ‘lower’.

\item {} 
\sphinxAtStartPar
\sphinxstyleliteralstrong{\sphinxupquote{statistic}} (\sphinxstyleliteralemphasis{\sphinxupquote{function}}) – Summary statistic (metric measure) fed to evaluator\_corners\_wide method
in Evaluator

\item {} 
\sphinxAtStartPar
\sphinxstyleliteralstrong{\sphinxupquote{total\_amount}} – Upper limit on total investments summed across all R\&D categories

\item {} 
\sphinxAtStartPar
\sphinxstyleliteralstrong{\sphinxupquote{sizelimit}} (\sphinxstyleliteralemphasis{\sphinxupquote{int}}) – Maximum allowed number of binary variables. If the problem size exceeds
this limit, pwlinear\_milp will exit before building or solving the model.

\item {} 
\sphinxAtStartPar
\sphinxstyleliteralstrong{\sphinxupquote{verbose}} (\sphinxstyleliteralemphasis{\sphinxupquote{int}}) – A value greater than zero will save the optimization model as a .lp file
A value greater than 1 will print out status messages

\end{itemize}

\sphinxlineitem{Returns}
\sphinxAtStartPar
\sphinxstylestrong{Optimum} – exit\_code
exit\_message
amounts (None, if no solution found)
metrics (None, if no solution found)
solve\_time
opt\_sense

\sphinxlineitem{Return type}
\sphinxAtStartPar
NamedTuple

\end{description}\end{quote}

\end{fulllineitems}

\index{opt\_shgo() (tyche.EpsilonConstraints.EpsilonConstraintOptimizer method)@\spxentry{opt\_shgo()}\spxextra{tyche.EpsilonConstraints.EpsilonConstraintOptimizer method}}

\begin{fulllineitems}
\phantomsection\label{\detokenize{tyche:tyche.EpsilonConstraints.EpsilonConstraintOptimizer.opt_shgo}}
\pysigstartsignatures
\pysiglinewithargsret{\sphinxbfcode{\sphinxupquote{opt\_shgo}}}{\emph{\DUrole{n}{metric}}, \emph{\DUrole{n}{sense=None}}, \emph{\DUrole{n}{max\_amount=None}}, \emph{\DUrole{n}{total\_amount=None}}, \emph{\DUrole{n}{eps\_metric=None}}, \emph{\DUrole{n}{statistic=<function mean>}}, \emph{\DUrole{n}{tol=0.01}}, \emph{\DUrole{n}{maxiter=None}}, \emph{\DUrole{n}{sampling\_method='simplicial'}}, \emph{\DUrole{n}{verbose=0}}}{}
\pysigstopsignatures
\sphinxAtStartPar
Maximize the objective function using the shgo global optimization
algorithm.
\begin{quote}\begin{description}
\sphinxlineitem{Parameters}\begin{itemize}
\item {} 
\sphinxAtStartPar
\sphinxstyleliteralstrong{\sphinxupquote{metric}} (\sphinxstyleliteralemphasis{\sphinxupquote{str}}) – Name of metric to maximize.

\item {} 
\sphinxAtStartPar
\sphinxstyleliteralstrong{\sphinxupquote{sense}} (\sphinxstyleliteralemphasis{\sphinxupquote{str}}) – Optimization sense (‘min’ or ‘max’). If no value is provided to
this method, the sense value used to create the
EpsilonConstraintOptimizer object is used instead.

\item {} 
\sphinxAtStartPar
\sphinxstyleliteralstrong{\sphinxupquote{max\_amount}} (\sphinxstyleliteralemphasis{\sphinxupquote{Series}}) – Maximum investment amounts by R\&D category. The index (research area)
order must follow that of self.evaluator.max\_amount.index

\item {} 
\sphinxAtStartPar
\sphinxstyleliteralstrong{\sphinxupquote{total\_amount}} (\sphinxstyleliteralemphasis{\sphinxupquote{float}}) – Upper metric\_limit on total investments summed across all R\&D categories.

\item {} 
\sphinxAtStartPar
\sphinxstyleliteralstrong{\sphinxupquote{eps\_metric}} (\sphinxstyleliteralemphasis{\sphinxupquote{Dict}}) – RHS of the epsilon constraint(s) on one or more metrics. Keys are metric
names, and the values are dictionaries of the form
\{‘limit’: float, ‘sense’: str\}. The sense defines whether the epsilon
constraint is a lower or an upper bound, and the value must be either
‘upper’ or ‘lower’.

\item {} 
\sphinxAtStartPar
\sphinxstyleliteralstrong{\sphinxupquote{statistic}} (\sphinxstyleliteralemphasis{\sphinxupquote{function}}) – Summary metric\_statistic used on the sample evaluations; the metric
measure that is fed to the optimizer.

\item {} 
\sphinxAtStartPar
\sphinxstyleliteralstrong{\sphinxupquote{tol}} (\sphinxstyleliteralemphasis{\sphinxupquote{float}}) – Objective function tolerance in stopping criterion.

\item {} 
\sphinxAtStartPar
\sphinxstyleliteralstrong{\sphinxupquote{maxiter}} (\sphinxstyleliteralemphasis{\sphinxupquote{int}}) – Upper metric\_limit on iterations that can be performed. Defaults to None.
Specifying this parameter can cause shgo to stall out instead of solving.

\item {} 
\sphinxAtStartPar
\sphinxstyleliteralstrong{\sphinxupquote{sampling\_method}} (\sphinxstyleliteralemphasis{\sphinxupquote{str}}) – Allowable values are ‘sobol and ‘simplicial’. Simplicial is default, uses
less memory, and guarantees convergence (theoretically). Sobol is faster,
uses more memory and does not guarantee convergence. Per documentation,
Sobol is better for “easier” problems.

\item {} 
\sphinxAtStartPar
\sphinxstyleliteralstrong{\sphinxupquote{verbose}} (\sphinxstyleliteralemphasis{\sphinxupquote{int}}) – Verbosity level returned by this outer function and the SHGO algorithm.
verbose = 0     No messages
verbose = 1     Convergence messages from SHGO algorithm
verbose = 2     Investment constraint status, metric constraint status,
and convergence messages
verbose = 3     Decision variable values, investment constraint status,
metric constraint status, and convergence messages
verbose > 3     All metric values, decision variable values, investment
constraint status, metric constraint status, and
convergence messages

\end{itemize}

\end{description}\end{quote}

\end{fulllineitems}

\index{opt\_slsqp() (tyche.EpsilonConstraints.EpsilonConstraintOptimizer method)@\spxentry{opt\_slsqp()}\spxextra{tyche.EpsilonConstraints.EpsilonConstraintOptimizer method}}

\begin{fulllineitems}
\phantomsection\label{\detokenize{tyche:tyche.EpsilonConstraints.EpsilonConstraintOptimizer.opt_slsqp}}
\pysigstartsignatures
\pysiglinewithargsret{\sphinxbfcode{\sphinxupquote{opt\_slsqp}}}{\emph{\DUrole{n}{metric}}, \emph{\DUrole{n}{sense=None}}, \emph{\DUrole{n}{max\_amount=None}}, \emph{\DUrole{n}{total\_amount=None}}, \emph{\DUrole{n}{eps\_metric=None}}, \emph{\DUrole{n}{statistic=<function mean>}}, \emph{\DUrole{n}{initial=None}}, \emph{\DUrole{n}{tol=1e\sphinxhyphen{}08}}, \emph{\DUrole{n}{maxiter=50}}, \emph{\DUrole{n}{verbose=0}}}{}
\pysigstopsignatures
\sphinxAtStartPar
Optimize the objective function using the fmin\_slsqp algorithm.
\begin{quote}\begin{description}
\sphinxlineitem{Parameters}\begin{itemize}
\item {} 
\sphinxAtStartPar
\sphinxstyleliteralstrong{\sphinxupquote{metric}} (\sphinxstyleliteralemphasis{\sphinxupquote{str}}) – Name of metric to maximize.

\item {} 
\sphinxAtStartPar
\sphinxstyleliteralstrong{\sphinxupquote{sense}} (\sphinxstyleliteralemphasis{\sphinxupquote{str}}) – Optimization sense (‘min’ or ‘max’). If no value is provided to
this method, the sense value used to create the
EpsilonConstraintOptimizer object is used instead.

\item {} 
\sphinxAtStartPar
\sphinxstyleliteralstrong{\sphinxupquote{max\_amount}} (\sphinxstyleliteralemphasis{\sphinxupquote{Series}}) – Maximum investment amounts by R\&D category. The index (research area)
order must follow that of self.evaluator.max\_amount.index

\item {} 
\sphinxAtStartPar
\sphinxstyleliteralstrong{\sphinxupquote{total\_amount}} (\sphinxstyleliteralemphasis{\sphinxupquote{float}}) – Upper limit on total investments summed across all R\&D categories.

\item {} 
\sphinxAtStartPar
\sphinxstyleliteralstrong{\sphinxupquote{eps\_metric}} (\sphinxstyleliteralemphasis{\sphinxupquote{Dict}}) – RHS of the epsilon constraint(s) on one or more metrics. Keys are metric
names, and the values are dictionaries of the form
\{‘limit’: float, ‘sense’: str\}. The sense defines whether the epsilon
constraint is a lower or an upper bound, and the value must be either
‘upper’ or ‘lower’.

\item {} 
\sphinxAtStartPar
\sphinxstyleliteralstrong{\sphinxupquote{statistic}} (\sphinxstyleliteralemphasis{\sphinxupquote{function}}) – Summary statistic used on the sample evaluations; the metric measure that
is fed to the optimizer.

\item {} 
\sphinxAtStartPar
\sphinxstyleliteralstrong{\sphinxupquote{initial}} (\sphinxstyleliteralemphasis{\sphinxupquote{array of float}}) – Initial value of decision variable(s) fed to the optimizer.

\item {} 
\sphinxAtStartPar
\sphinxstyleliteralstrong{\sphinxupquote{tol}} (\sphinxstyleliteralemphasis{\sphinxupquote{float}}) – Search tolerance fed to the optimizer.

\item {} 
\sphinxAtStartPar
\sphinxstyleliteralstrong{\sphinxupquote{maxiter}} (\sphinxstyleliteralemphasis{\sphinxupquote{int}}) – Maximum number of iterations the optimizer is permitted to execute.

\item {} 
\sphinxAtStartPar
\sphinxstyleliteralstrong{\sphinxupquote{verbose}} (\sphinxstyleliteralemphasis{\sphinxupquote{int}}) – Verbosity level returned by the optimizer and this outer function.
Defaults to 0.
verbose = 0     No messages
verbose = 1     Summary message when fmin\_slsqp completes
verbose = 2     Status of each algorithm iteration and summary message
verbose = 3     Investment constraint status, metric constraint status,
status of each algorithm iteration, and summary message
verbose > 3     All metric values, decision variable values, investment
constraint status, metric constraint status, status of
each algorithm iteration, and summary message

\end{itemize}

\end{description}\end{quote}

\end{fulllineitems}

\index{optimum\_metrics() (tyche.EpsilonConstraints.EpsilonConstraintOptimizer method)@\spxentry{optimum\_metrics()}\spxextra{tyche.EpsilonConstraints.EpsilonConstraintOptimizer method}}

\begin{fulllineitems}
\phantomsection\label{\detokenize{tyche:tyche.EpsilonConstraints.EpsilonConstraintOptimizer.optimum_metrics}}
\pysigstartsignatures
\pysiglinewithargsret{\sphinxbfcode{\sphinxupquote{optimum\_metrics}}}{\emph{\DUrole{n}{max\_amount=None}}, \emph{\DUrole{n}{total\_amount=None}}, \emph{\DUrole{n}{sense=None}}, \emph{\DUrole{n}{statistic=<function mean>}}, \emph{\DUrole{n}{tol=1e\sphinxhyphen{}08}}, \emph{\DUrole{n}{maxiter=50}}, \emph{\DUrole{n}{verbose=0}}}{}
\pysigstopsignatures
\sphinxAtStartPar
Maximum value of metrics.
\begin{quote}\begin{description}
\sphinxlineitem{Parameters}\begin{itemize}
\item {} 
\sphinxAtStartPar
\sphinxstyleliteralstrong{\sphinxupquote{max\_amount}} (\sphinxstyleliteralemphasis{\sphinxupquote{DataFrame}}) – The maximum amounts that can be invested in each category.

\item {} 
\sphinxAtStartPar
\sphinxstyleliteralstrong{\sphinxupquote{total\_amount}} (\sphinxstyleliteralemphasis{\sphinxupquote{float}}) – The maximum amount that can be invested \sphinxstyleemphasis{in toto}.

\item {} 
\sphinxAtStartPar
\sphinxstyleliteralstrong{\sphinxupquote{sense}} (\sphinxstyleliteralemphasis{\sphinxupquote{Dict}}\sphinxstyleliteralemphasis{\sphinxupquote{ or }}\sphinxstyleliteralemphasis{\sphinxupquote{str}}) – Optimization sense for each metric. Must be ‘min’ or ‘max’. If None, then
the sense provided to the EpsilonConstraintOptimizer class is used for
all metrics. If string, the sense is used for all metrics.

\item {} 
\sphinxAtStartPar
\sphinxstyleliteralstrong{\sphinxupquote{statistic}} (\sphinxstyleliteralemphasis{\sphinxupquote{function}}) – The statistic used on the sample evaluations.

\item {} 
\sphinxAtStartPar
\sphinxstyleliteralstrong{\sphinxupquote{tol}} (\sphinxstyleliteralemphasis{\sphinxupquote{float}}) – The search tolerance.

\item {} 
\sphinxAtStartPar
\sphinxstyleliteralstrong{\sphinxupquote{maxiter}} (\sphinxstyleliteralemphasis{\sphinxupquote{int}}) – The maximum iterations for the search.

\item {} 
\sphinxAtStartPar
\sphinxstyleliteralstrong{\sphinxupquote{verbose}} (\sphinxstyleliteralemphasis{\sphinxupquote{int}}) – Verbosity level.

\end{itemize}

\end{description}\end{quote}

\end{fulllineitems}


\end{fulllineitems}



\subsection{tyche.Evaluator}
\label{\detokenize{tyche:module-tyche.Evaluator}}\label{\detokenize{tyche:tyche-evaluator}}\index{module@\spxentry{module}!tyche.Evaluator@\spxentry{tyche.Evaluator}}\index{tyche.Evaluator@\spxentry{tyche.Evaluator}!module@\spxentry{module}}
\sphinxAtStartPar
Fast evaluation of technology investments.
\index{Evaluator (class in tyche.Evaluator)@\spxentry{Evaluator}\spxextra{class in tyche.Evaluator}}

\begin{fulllineitems}
\phantomsection\label{\detokenize{tyche:tyche.Evaluator.Evaluator}}
\pysigstartsignatures
\pysiglinewithargsret{\sphinxbfcode{\sphinxupquote{class\DUrole{w}{  }}}\sphinxcode{\sphinxupquote{tyche.Evaluator.}}\sphinxbfcode{\sphinxupquote{Evaluator}}}{\emph{\DUrole{n}{tranches}}}{}
\pysigstopsignatures
\sphinxAtStartPar
Bases: \sphinxcode{\sphinxupquote{object}}

\sphinxAtStartPar
Evalutate technology investments using a response surface.
\index{amounts (tyche.Evaluator.Evaluator attribute)@\spxentry{amounts}\spxextra{tyche.Evaluator.Evaluator attribute}}

\begin{fulllineitems}
\phantomsection\label{\detokenize{tyche:tyche.Evaluator.Evaluator.amounts}}
\pysigstartsignatures
\pysigline{\sphinxbfcode{\sphinxupquote{amounts}}}
\pysigstopsignatures
\sphinxAtStartPar
Cost of tranches.
\begin{quote}\begin{description}
\sphinxlineitem{Type}
\sphinxAtStartPar
DataFrame

\end{description}\end{quote}

\end{fulllineitems}

\index{categories (tyche.Evaluator.Evaluator attribute)@\spxentry{categories}\spxextra{tyche.Evaluator.Evaluator attribute}}

\begin{fulllineitems}
\phantomsection\label{\detokenize{tyche:tyche.Evaluator.Evaluator.categories}}
\pysigstartsignatures
\pysigline{\sphinxbfcode{\sphinxupquote{categories}}}
\pysigstopsignatures
\sphinxAtStartPar
Categories of investment.
\begin{quote}\begin{description}
\sphinxlineitem{Type}
\sphinxAtStartPar
DataFrame

\end{description}\end{quote}

\end{fulllineitems}

\index{metrics (tyche.Evaluator.Evaluator attribute)@\spxentry{metrics}\spxextra{tyche.Evaluator.Evaluator attribute}}

\begin{fulllineitems}
\phantomsection\label{\detokenize{tyche:tyche.Evaluator.Evaluator.metrics}}
\pysigstartsignatures
\pysigline{\sphinxbfcode{\sphinxupquote{metrics}}}
\pysigstopsignatures
\sphinxAtStartPar
Metrics for technologies.
\begin{quote}\begin{description}
\sphinxlineitem{Type}
\sphinxAtStartPar
DataFrame

\end{description}\end{quote}

\end{fulllineitems}

\index{units (tyche.Evaluator.Evaluator attribute)@\spxentry{units}\spxextra{tyche.Evaluator.Evaluator attribute}}

\begin{fulllineitems}
\phantomsection\label{\detokenize{tyche:tyche.Evaluator.Evaluator.units}}
\pysigstartsignatures
\pysigline{\sphinxbfcode{\sphinxupquote{units}}}
\pysigstopsignatures
\sphinxAtStartPar
Units of measure for metrics.
\begin{quote}\begin{description}
\sphinxlineitem{Type}
\sphinxAtStartPar
DataFrame

\end{description}\end{quote}

\end{fulllineitems}

\index{interpolators (tyche.Evaluator.Evaluator attribute)@\spxentry{interpolators}\spxextra{tyche.Evaluator.Evaluator attribute}}

\begin{fulllineitems}
\phantomsection\label{\detokenize{tyche:tyche.Evaluator.Evaluator.interpolators}}
\pysigstartsignatures
\pysigline{\sphinxbfcode{\sphinxupquote{interpolators}}}
\pysigstopsignatures
\sphinxAtStartPar
Interpolation functions for technology metrics.
\begin{quote}\begin{description}
\sphinxlineitem{Type}
\sphinxAtStartPar
DataFrame

\end{description}\end{quote}

\end{fulllineitems}

\index{evaluate() (tyche.Evaluator.Evaluator method)@\spxentry{evaluate()}\spxextra{tyche.Evaluator.Evaluator method}}

\begin{fulllineitems}
\phantomsection\label{\detokenize{tyche:tyche.Evaluator.Evaluator.evaluate}}
\pysigstartsignatures
\pysiglinewithargsret{\sphinxbfcode{\sphinxupquote{evaluate}}}{\emph{\DUrole{n}{amounts}}}{}
\pysigstopsignatures
\sphinxAtStartPar
Sample the distribution for an investment.
\begin{quote}\begin{description}
\sphinxlineitem{Parameters}
\sphinxAtStartPar
\sphinxstyleliteralstrong{\sphinxupquote{amounts}} (\sphinxstyleliteralemphasis{\sphinxupquote{DataFrame}}) – The investment levels.

\end{description}\end{quote}

\end{fulllineitems}

\index{evaluate\_corners\_semilong() (tyche.Evaluator.Evaluator method)@\spxentry{evaluate\_corners\_semilong()}\spxextra{tyche.Evaluator.Evaluator method}}

\begin{fulllineitems}
\phantomsection\label{\detokenize{tyche:tyche.Evaluator.Evaluator.evaluate_corners_semilong}}
\pysigstartsignatures
\pysiglinewithargsret{\sphinxbfcode{\sphinxupquote{evaluate\_corners\_semilong}}}{\emph{\DUrole{n}{statistic=<function mean>}}}{}
\pysigstopsignatures
\sphinxAtStartPar
Return a dataframe indexed my investment amounts in each category,
with columns for each metric.
\begin{quote}\begin{description}
\sphinxlineitem{Parameters}
\sphinxAtStartPar
\sphinxstyleliteralstrong{\sphinxupquote{statistic}} (\sphinxstyleliteralemphasis{\sphinxupquote{function}}) – The statistic to evaluate.

\end{description}\end{quote}

\end{fulllineitems}

\index{evaluate\_corners\_wide() (tyche.Evaluator.Evaluator method)@\spxentry{evaluate\_corners\_wide()}\spxextra{tyche.Evaluator.Evaluator method}}

\begin{fulllineitems}
\phantomsection\label{\detokenize{tyche:tyche.Evaluator.Evaluator.evaluate_corners_wide}}
\pysigstartsignatures
\pysiglinewithargsret{\sphinxbfcode{\sphinxupquote{evaluate\_corners\_wide}}}{\emph{\DUrole{n}{statistic=<function mean>}}}{}
\pysigstopsignatures
\sphinxAtStartPar
Return a dataframe indexed my investment amounts in each category,
with columns for each metric.
\begin{quote}\begin{description}
\sphinxlineitem{Parameters}
\sphinxAtStartPar
\sphinxstyleliteralstrong{\sphinxupquote{statistic}} (\sphinxstyleliteralemphasis{\sphinxupquote{function}}) – The statistic to evaluate.

\end{description}\end{quote}

\end{fulllineitems}

\index{evaluate\_statistic() (tyche.Evaluator.Evaluator method)@\spxentry{evaluate\_statistic()}\spxextra{tyche.Evaluator.Evaluator method}}

\begin{fulllineitems}
\phantomsection\label{\detokenize{tyche:tyche.Evaluator.Evaluator.evaluate_statistic}}
\pysigstartsignatures
\pysiglinewithargsret{\sphinxbfcode{\sphinxupquote{evaluate\_statistic}}}{\emph{\DUrole{n}{amounts}}, \emph{\DUrole{n}{statistic=<function mean>}}}{}
\pysigstopsignatures
\sphinxAtStartPar
Evaluate a statistic for an investment.
\begin{quote}\begin{description}
\sphinxlineitem{Parameters}\begin{itemize}
\item {} 
\sphinxAtStartPar
\sphinxstyleliteralstrong{\sphinxupquote{amounts}} (\sphinxstyleliteralemphasis{\sphinxupquote{DataFrame}}) – The investment levels.

\item {} 
\sphinxAtStartPar
\sphinxstyleliteralstrong{\sphinxupquote{statistic}} (\sphinxstyleliteralemphasis{\sphinxupquote{function}}) – The statistic to evaluate.

\end{itemize}

\end{description}\end{quote}

\end{fulllineitems}

\index{make\_statistic\_evaluator() (tyche.Evaluator.Evaluator method)@\spxentry{make\_statistic\_evaluator()}\spxextra{tyche.Evaluator.Evaluator method}}

\begin{fulllineitems}
\phantomsection\label{\detokenize{tyche:tyche.Evaluator.Evaluator.make_statistic_evaluator}}
\pysigstartsignatures
\pysiglinewithargsret{\sphinxbfcode{\sphinxupquote{make\_statistic\_evaluator}}}{\emph{\DUrole{n}{statistic=<function mean>}}}{}
\pysigstopsignatures
\sphinxAtStartPar
Return a function that evaluates a statistic for an investment.
\begin{quote}\begin{description}
\sphinxlineitem{Parameters}
\sphinxAtStartPar
\sphinxstyleliteralstrong{\sphinxupquote{statistic}} (\sphinxstyleliteralemphasis{\sphinxupquote{function}}) – The statistic to evaluate.

\end{description}\end{quote}

\end{fulllineitems}


\end{fulllineitems}



\subsection{tyche.IO}
\label{\detokenize{tyche:module-tyche.IO}}\label{\detokenize{tyche:tyche-io}}\index{module@\spxentry{module}!tyche.IO@\spxentry{tyche.IO}}\index{tyche.IO@\spxentry{tyche.IO}!module@\spxentry{module}}
\sphinxAtStartPar
I/O utilities for Tyche.
\index{check\_tables() (in module tyche.IO)@\spxentry{check\_tables()}\spxextra{in module tyche.IO}}

\begin{fulllineitems}
\phantomsection\label{\detokenize{tyche:tyche.IO.check_tables}}
\pysigstartsignatures
\pysiglinewithargsret{\sphinxcode{\sphinxupquote{tyche.IO.}}\sphinxbfcode{\sphinxupquote{check\_tables}}}{\emph{\DUrole{n}{path}}, \emph{\DUrole{n}{name}}}{}
\pysigstopsignatures
\sphinxAtStartPar
Perform validity checks on input datasets.

\sphinxAtStartPar
All checks are run before this method terminates; that is, data errors are found
all at once rather than one at a time from several calls to this method. A list of
errors found is printed if any check fails. The errors include a summary of the check
and identify the dataset that needs to be changed.
\begin{quote}\begin{description}
\sphinxlineitem{Parameters}\begin{itemize}
\item {} 
\sphinxAtStartPar
\sphinxstyleliteralstrong{\sphinxupquote{path}} (\sphinxstyleliteralemphasis{\sphinxupquote{str}}) – Path to directory of datasets

\item {} 
\sphinxAtStartPar
\sphinxstyleliteralstrong{\sphinxupquote{name}} (\sphinxstyleliteralemphasis{\sphinxupquote{str}}) – Name of datasets file (XLSX)

\end{itemize}

\sphinxlineitem{Returns}
\sphinxAtStartPar
\sphinxstylestrong{Boolean}

\sphinxlineitem{Return type}
\sphinxAtStartPar
True if data is valid, False otherwise

\end{description}\end{quote}

\end{fulllineitems}



\subsection{tyche.Investments}
\label{\detokenize{tyche:module-tyche.Investments}}\label{\detokenize{tyche:tyche-investments}}\index{module@\spxentry{module}!tyche.Investments@\spxentry{tyche.Investments}}\index{tyche.Investments@\spxentry{tyche.Investments}!module@\spxentry{module}}
\sphinxAtStartPar
Investments in technologies.
\index{Investments (class in tyche.Investments)@\spxentry{Investments}\spxextra{class in tyche.Investments}}

\begin{fulllineitems}
\phantomsection\label{\detokenize{tyche:tyche.Investments.Investments}}
\pysigstartsignatures
\pysiglinewithargsret{\sphinxbfcode{\sphinxupquote{class\DUrole{w}{  }}}\sphinxcode{\sphinxupquote{tyche.Investments.}}\sphinxbfcode{\sphinxupquote{Investments}}}{\emph{\DUrole{n}{path}\DUrole{o}{=}\DUrole{default_value}{None}}, \emph{\DUrole{n}{name}\DUrole{o}{=}\DUrole{default_value}{'technology.xlsx'}}, \emph{\DUrole{n}{uncertain}\DUrole{o}{=}\DUrole{default_value}{False}}, \emph{\DUrole{n}{tranches}\DUrole{o}{=}\DUrole{default_value}{'tranches'}}, \emph{\DUrole{n}{investments}\DUrole{o}{=}\DUrole{default_value}{'investments'}}}{}
\pysigstopsignatures
\sphinxAtStartPar
Bases: \sphinxcode{\sphinxupquote{object}}

\sphinxAtStartPar
Investments in a technology.
\index{tranches (tyche.Investments.Investments attribute)@\spxentry{tranches}\spxextra{tyche.Investments.Investments attribute}}

\begin{fulllineitems}
\phantomsection\label{\detokenize{tyche:tyche.Investments.Investments.tranches}}
\pysigstartsignatures
\pysigline{\sphinxbfcode{\sphinxupquote{tranches}}}
\pysigstopsignatures
\sphinxAtStartPar
The \sphinxstyleemphasis{tranches} table.
\begin{quote}\begin{description}
\sphinxlineitem{Type}
\sphinxAtStartPar
DataFrame

\end{description}\end{quote}

\end{fulllineitems}

\index{investments (tyche.Investments.Investments attribute)@\spxentry{investments}\spxextra{tyche.Investments.Investments attribute}}

\begin{fulllineitems}
\phantomsection\label{\detokenize{tyche:tyche.Investments.Investments.investments}}
\pysigstartsignatures
\pysigline{\sphinxbfcode{\sphinxupquote{investments}}}
\pysigstopsignatures
\sphinxAtStartPar
The \sphinxstyleemphasis{investments} table.
\begin{quote}\begin{description}
\sphinxlineitem{Type}
\sphinxAtStartPar
DataFrame

\end{description}\end{quote}

\end{fulllineitems}

\index{compile() (tyche.Investments.Investments method)@\spxentry{compile()}\spxextra{tyche.Investments.Investments method}}

\begin{fulllineitems}
\phantomsection\label{\detokenize{tyche:tyche.Investments.Investments.compile}}
\pysigstartsignatures
\pysiglinewithargsret{\sphinxbfcode{\sphinxupquote{compile}}}{}{}
\pysigstopsignatures
\sphinxAtStartPar
Parse any probability distributions in the tranches.

\end{fulllineitems}

\index{evaluate\_investments() (tyche.Investments.Investments method)@\spxentry{evaluate\_investments()}\spxextra{tyche.Investments.Investments method}}

\begin{fulllineitems}
\phantomsection\label{\detokenize{tyche:tyche.Investments.Investments.evaluate_investments}}
\pysigstartsignatures
\pysiglinewithargsret{\sphinxbfcode{\sphinxupquote{evaluate\_investments}}}{\emph{\DUrole{n}{designs}}, \emph{\DUrole{n}{tranche\_results}\DUrole{o}{=}\DUrole{default_value}{None}}, \emph{\DUrole{n}{sample\_count}\DUrole{o}{=}\DUrole{default_value}{1}}}{}
\pysigstopsignatures
\sphinxAtStartPar
Evaluate the investments for a design.
\begin{quote}\begin{description}
\sphinxlineitem{Parameters}\begin{itemize}
\item {} 
\sphinxAtStartPar
\sphinxstyleliteralstrong{\sphinxupquote{designs}} (\sphinxstyleliteralemphasis{\sphinxupquote{tyche.Designs}}) – The designs.

\item {} 
\sphinxAtStartPar
\sphinxstyleliteralstrong{\sphinxupquote{tranche\_results}} (\sphinxstyleliteralemphasis{\sphinxupquote{tyche.Evaluations}}) – Output of evaluate\_tranches method. Necessary only if the investment amounts contain uncertainty.

\item {} 
\sphinxAtStartPar
\sphinxstyleliteralstrong{\sphinxupquote{sample\_count}} (\sphinxstyleliteralemphasis{\sphinxupquote{int}}) – The number of random samples.

\end{itemize}

\end{description}\end{quote}

\end{fulllineitems}

\index{evaluate\_tranches() (tyche.Investments.Investments method)@\spxentry{evaluate\_tranches()}\spxextra{tyche.Investments.Investments method}}

\begin{fulllineitems}
\phantomsection\label{\detokenize{tyche:tyche.Investments.Investments.evaluate_tranches}}
\pysigstartsignatures
\pysiglinewithargsret{\sphinxbfcode{\sphinxupquote{evaluate\_tranches}}}{\emph{\DUrole{n}{designs}}, \emph{\DUrole{n}{sample\_count}\DUrole{o}{=}\DUrole{default_value}{1}}}{}
\pysigstopsignatures
\sphinxAtStartPar
Evaluate the tranches of investment for a design.
\begin{quote}\begin{description}
\sphinxlineitem{Parameters}\begin{itemize}
\item {} 
\sphinxAtStartPar
\sphinxstyleliteralstrong{\sphinxupquote{designs}} (\sphinxstyleliteralemphasis{\sphinxupquote{tyche.Designs}}) – The designs.

\item {} 
\sphinxAtStartPar
\sphinxstyleliteralstrong{\sphinxupquote{sample\_count}} (\sphinxstyleliteralemphasis{\sphinxupquote{int}}) – The number of random samples.

\end{itemize}

\end{description}\end{quote}

\end{fulllineitems}


\end{fulllineitems}



\subsection{tyche.Types}
\label{\detokenize{tyche:module-tyche.Types}}\label{\detokenize{tyche:tyche-types}}\index{module@\spxentry{module}!tyche.Types@\spxentry{tyche.Types}}\index{tyche.Types@\spxentry{tyche.Types}!module@\spxentry{module}}
\sphinxAtStartPar
Data types for Tyche.
\index{Evaluations (class in tyche.Types)@\spxentry{Evaluations}\spxextra{class in tyche.Types}}

\begin{fulllineitems}
\phantomsection\label{\detokenize{tyche:tyche.Types.Evaluations}}
\pysigstartsignatures
\pysiglinewithargsret{\sphinxbfcode{\sphinxupquote{class\DUrole{w}{  }}}\sphinxcode{\sphinxupquote{tyche.Types.}}\sphinxbfcode{\sphinxupquote{Evaluations}}}{\emph{\DUrole{n}{amounts}}, \emph{\DUrole{n}{metrics}}, \emph{\DUrole{n}{summary}}, \emph{\DUrole{n}{uncertain}}}{}
\pysigstopsignatures
\sphinxAtStartPar
Bases: \sphinxcode{\sphinxupquote{tuple}}

\sphinxAtStartPar
Named tuple type for rows in the \sphinxstyleemphasis{evaluations} table.
\index{amounts (tyche.Types.Evaluations attribute)@\spxentry{amounts}\spxextra{tyche.Types.Evaluations attribute}}

\begin{fulllineitems}
\phantomsection\label{\detokenize{tyche:tyche.Types.Evaluations.amounts}}
\pysigstartsignatures
\pysigline{\sphinxbfcode{\sphinxupquote{amounts}}}
\pysigstopsignatures
\sphinxAtStartPar
Alias for field number 0

\end{fulllineitems}

\index{metrics (tyche.Types.Evaluations attribute)@\spxentry{metrics}\spxextra{tyche.Types.Evaluations attribute}}

\begin{fulllineitems}
\phantomsection\label{\detokenize{tyche:tyche.Types.Evaluations.metrics}}
\pysigstartsignatures
\pysigline{\sphinxbfcode{\sphinxupquote{metrics}}}
\pysigstopsignatures
\sphinxAtStartPar
Alias for field number 1

\end{fulllineitems}

\index{summary (tyche.Types.Evaluations attribute)@\spxentry{summary}\spxextra{tyche.Types.Evaluations attribute}}

\begin{fulllineitems}
\phantomsection\label{\detokenize{tyche:tyche.Types.Evaluations.summary}}
\pysigstartsignatures
\pysigline{\sphinxbfcode{\sphinxupquote{summary}}}
\pysigstopsignatures
\sphinxAtStartPar
Alias for field number 2

\end{fulllineitems}

\index{uncertain (tyche.Types.Evaluations attribute)@\spxentry{uncertain}\spxextra{tyche.Types.Evaluations attribute}}

\begin{fulllineitems}
\phantomsection\label{\detokenize{tyche:tyche.Types.Evaluations.uncertain}}
\pysigstartsignatures
\pysigline{\sphinxbfcode{\sphinxupquote{uncertain}}}
\pysigstopsignatures
\sphinxAtStartPar
Alias for field number 3

\end{fulllineitems}


\end{fulllineitems}

\index{Functions (class in tyche.Types)@\spxentry{Functions}\spxextra{class in tyche.Types}}

\begin{fulllineitems}
\phantomsection\label{\detokenize{tyche:tyche.Types.Functions}}
\pysigstartsignatures
\pysiglinewithargsret{\sphinxbfcode{\sphinxupquote{class\DUrole{w}{  }}}\sphinxcode{\sphinxupquote{tyche.Types.}}\sphinxbfcode{\sphinxupquote{Functions}}}{\emph{\DUrole{n}{style}}, \emph{\DUrole{n}{capital}}, \emph{\DUrole{n}{fixed}}, \emph{\DUrole{n}{production}}, \emph{\DUrole{n}{metric}}}{}
\pysigstopsignatures
\sphinxAtStartPar
Bases: \sphinxcode{\sphinxupquote{tuple}}

\sphinxAtStartPar
Name tuple type for rows in the \sphinxstyleemphasis{functions} table.
\index{capital (tyche.Types.Functions attribute)@\spxentry{capital}\spxextra{tyche.Types.Functions attribute}}

\begin{fulllineitems}
\phantomsection\label{\detokenize{tyche:tyche.Types.Functions.capital}}
\pysigstartsignatures
\pysigline{\sphinxbfcode{\sphinxupquote{capital}}}
\pysigstopsignatures
\sphinxAtStartPar
Alias for field number 1

\end{fulllineitems}

\index{fixed (tyche.Types.Functions attribute)@\spxentry{fixed}\spxextra{tyche.Types.Functions attribute}}

\begin{fulllineitems}
\phantomsection\label{\detokenize{tyche:tyche.Types.Functions.fixed}}
\pysigstartsignatures
\pysigline{\sphinxbfcode{\sphinxupquote{fixed}}}
\pysigstopsignatures
\sphinxAtStartPar
Alias for field number 2

\end{fulllineitems}

\index{metric (tyche.Types.Functions attribute)@\spxentry{metric}\spxextra{tyche.Types.Functions attribute}}

\begin{fulllineitems}
\phantomsection\label{\detokenize{tyche:tyche.Types.Functions.metric}}
\pysigstartsignatures
\pysigline{\sphinxbfcode{\sphinxupquote{metric}}}
\pysigstopsignatures
\sphinxAtStartPar
Alias for field number 4

\end{fulllineitems}

\index{production (tyche.Types.Functions attribute)@\spxentry{production}\spxextra{tyche.Types.Functions attribute}}

\begin{fulllineitems}
\phantomsection\label{\detokenize{tyche:tyche.Types.Functions.production}}
\pysigstartsignatures
\pysigline{\sphinxbfcode{\sphinxupquote{production}}}
\pysigstopsignatures
\sphinxAtStartPar
Alias for field number 3

\end{fulllineitems}

\index{style (tyche.Types.Functions attribute)@\spxentry{style}\spxextra{tyche.Types.Functions attribute}}

\begin{fulllineitems}
\phantomsection\label{\detokenize{tyche:tyche.Types.Functions.style}}
\pysigstartsignatures
\pysigline{\sphinxbfcode{\sphinxupquote{style}}}
\pysigstopsignatures
\sphinxAtStartPar
Alias for field number 0

\end{fulllineitems}


\end{fulllineitems}

\index{Indices (class in tyche.Types)@\spxentry{Indices}\spxextra{class in tyche.Types}}

\begin{fulllineitems}
\phantomsection\label{\detokenize{tyche:tyche.Types.Indices}}
\pysigstartsignatures
\pysiglinewithargsret{\sphinxbfcode{\sphinxupquote{class\DUrole{w}{  }}}\sphinxcode{\sphinxupquote{tyche.Types.}}\sphinxbfcode{\sphinxupquote{Indices}}}{\emph{\DUrole{n}{capital}}, \emph{\DUrole{n}{input}}, \emph{\DUrole{n}{output}}, \emph{\DUrole{n}{metric}}}{}
\pysigstopsignatures
\sphinxAtStartPar
Bases: \sphinxcode{\sphinxupquote{tuple}}

\sphinxAtStartPar
Name tuple type for rows in the \sphinxstyleemphasis{indices} table.
\index{capital (tyche.Types.Indices attribute)@\spxentry{capital}\spxextra{tyche.Types.Indices attribute}}

\begin{fulllineitems}
\phantomsection\label{\detokenize{tyche:tyche.Types.Indices.capital}}
\pysigstartsignatures
\pysigline{\sphinxbfcode{\sphinxupquote{capital}}}
\pysigstopsignatures
\sphinxAtStartPar
Alias for field number 0

\end{fulllineitems}

\index{input (tyche.Types.Indices attribute)@\spxentry{input}\spxextra{tyche.Types.Indices attribute}}

\begin{fulllineitems}
\phantomsection\label{\detokenize{tyche:tyche.Types.Indices.input}}
\pysigstartsignatures
\pysigline{\sphinxbfcode{\sphinxupquote{input}}}
\pysigstopsignatures
\sphinxAtStartPar
Alias for field number 1

\end{fulllineitems}

\index{metric (tyche.Types.Indices attribute)@\spxentry{metric}\spxextra{tyche.Types.Indices attribute}}

\begin{fulllineitems}
\phantomsection\label{\detokenize{tyche:tyche.Types.Indices.metric}}
\pysigstartsignatures
\pysigline{\sphinxbfcode{\sphinxupquote{metric}}}
\pysigstopsignatures
\sphinxAtStartPar
Alias for field number 3

\end{fulllineitems}

\index{output (tyche.Types.Indices attribute)@\spxentry{output}\spxextra{tyche.Types.Indices attribute}}

\begin{fulllineitems}
\phantomsection\label{\detokenize{tyche:tyche.Types.Indices.output}}
\pysigstartsignatures
\pysigline{\sphinxbfcode{\sphinxupquote{output}}}
\pysigstopsignatures
\sphinxAtStartPar
Alias for field number 2

\end{fulllineitems}


\end{fulllineitems}

\index{Inputs (class in tyche.Types)@\spxentry{Inputs}\spxextra{class in tyche.Types}}

\begin{fulllineitems}
\phantomsection\label{\detokenize{tyche:tyche.Types.Inputs}}
\pysigstartsignatures
\pysiglinewithargsret{\sphinxbfcode{\sphinxupquote{class\DUrole{w}{  }}}\sphinxcode{\sphinxupquote{tyche.Types.}}\sphinxbfcode{\sphinxupquote{Inputs}}}{\emph{\DUrole{n}{lifetime}}, \emph{\DUrole{n}{scale}}, \emph{\DUrole{n}{input}}, \emph{\DUrole{n}{input\_efficiency}}, \emph{\DUrole{n}{input\_price}}, \emph{\DUrole{n}{output\_efficiency}}, \emph{\DUrole{n}{output\_price}}}{}
\pysigstopsignatures
\sphinxAtStartPar
Bases: \sphinxcode{\sphinxupquote{tuple}}

\sphinxAtStartPar
Named tuple type for rows in the \sphinxstyleemphasis{inputs} table.
\index{input (tyche.Types.Inputs attribute)@\spxentry{input}\spxextra{tyche.Types.Inputs attribute}}

\begin{fulllineitems}
\phantomsection\label{\detokenize{tyche:tyche.Types.Inputs.input}}
\pysigstartsignatures
\pysigline{\sphinxbfcode{\sphinxupquote{input}}}
\pysigstopsignatures
\sphinxAtStartPar
Alias for field number 2

\end{fulllineitems}

\index{input\_efficiency (tyche.Types.Inputs attribute)@\spxentry{input\_efficiency}\spxextra{tyche.Types.Inputs attribute}}

\begin{fulllineitems}
\phantomsection\label{\detokenize{tyche:tyche.Types.Inputs.input_efficiency}}
\pysigstartsignatures
\pysigline{\sphinxbfcode{\sphinxupquote{input\_efficiency}}}
\pysigstopsignatures
\sphinxAtStartPar
Alias for field number 3

\end{fulllineitems}

\index{input\_price (tyche.Types.Inputs attribute)@\spxentry{input\_price}\spxextra{tyche.Types.Inputs attribute}}

\begin{fulllineitems}
\phantomsection\label{\detokenize{tyche:tyche.Types.Inputs.input_price}}
\pysigstartsignatures
\pysigline{\sphinxbfcode{\sphinxupquote{input\_price}}}
\pysigstopsignatures
\sphinxAtStartPar
Alias for field number 4

\end{fulllineitems}

\index{lifetime (tyche.Types.Inputs attribute)@\spxentry{lifetime}\spxextra{tyche.Types.Inputs attribute}}

\begin{fulllineitems}
\phantomsection\label{\detokenize{tyche:tyche.Types.Inputs.lifetime}}
\pysigstartsignatures
\pysigline{\sphinxbfcode{\sphinxupquote{lifetime}}}
\pysigstopsignatures
\sphinxAtStartPar
Alias for field number 0

\end{fulllineitems}

\index{output\_efficiency (tyche.Types.Inputs attribute)@\spxentry{output\_efficiency}\spxextra{tyche.Types.Inputs attribute}}

\begin{fulllineitems}
\phantomsection\label{\detokenize{tyche:tyche.Types.Inputs.output_efficiency}}
\pysigstartsignatures
\pysigline{\sphinxbfcode{\sphinxupquote{output\_efficiency}}}
\pysigstopsignatures
\sphinxAtStartPar
Alias for field number 5

\end{fulllineitems}

\index{output\_price (tyche.Types.Inputs attribute)@\spxentry{output\_price}\spxextra{tyche.Types.Inputs attribute}}

\begin{fulllineitems}
\phantomsection\label{\detokenize{tyche:tyche.Types.Inputs.output_price}}
\pysigstartsignatures
\pysigline{\sphinxbfcode{\sphinxupquote{output\_price}}}
\pysigstopsignatures
\sphinxAtStartPar
Alias for field number 6

\end{fulllineitems}

\index{scale (tyche.Types.Inputs attribute)@\spxentry{scale}\spxextra{tyche.Types.Inputs attribute}}

\begin{fulllineitems}
\phantomsection\label{\detokenize{tyche:tyche.Types.Inputs.scale}}
\pysigstartsignatures
\pysigline{\sphinxbfcode{\sphinxupquote{scale}}}
\pysigstopsignatures
\sphinxAtStartPar
Alias for field number 1

\end{fulllineitems}


\end{fulllineitems}

\index{Optimum (class in tyche.Types)@\spxentry{Optimum}\spxextra{class in tyche.Types}}

\begin{fulllineitems}
\phantomsection\label{\detokenize{tyche:tyche.Types.Optimum}}
\pysigstartsignatures
\pysiglinewithargsret{\sphinxbfcode{\sphinxupquote{class\DUrole{w}{  }}}\sphinxcode{\sphinxupquote{tyche.Types.}}\sphinxbfcode{\sphinxupquote{Optimum}}}{\emph{\DUrole{n}{exit\_code}}, \emph{\DUrole{n}{exit\_message}}, \emph{\DUrole{n}{amounts}}, \emph{\DUrole{n}{metrics}}, \emph{\DUrole{n}{solve\_time}}, \emph{\DUrole{n}{opt\_sense}}}{}
\pysigstopsignatures
\sphinxAtStartPar
Bases: \sphinxcode{\sphinxupquote{tuple}}

\sphinxAtStartPar
Named tuple type for optimization results.
\index{amounts (tyche.Types.Optimum attribute)@\spxentry{amounts}\spxextra{tyche.Types.Optimum attribute}}

\begin{fulllineitems}
\phantomsection\label{\detokenize{tyche:tyche.Types.Optimum.amounts}}
\pysigstartsignatures
\pysigline{\sphinxbfcode{\sphinxupquote{amounts}}}
\pysigstopsignatures
\sphinxAtStartPar
Alias for field number 2

\end{fulllineitems}

\index{exit\_code (tyche.Types.Optimum attribute)@\spxentry{exit\_code}\spxextra{tyche.Types.Optimum attribute}}

\begin{fulllineitems}
\phantomsection\label{\detokenize{tyche:tyche.Types.Optimum.exit_code}}
\pysigstartsignatures
\pysigline{\sphinxbfcode{\sphinxupquote{exit\_code}}}
\pysigstopsignatures
\sphinxAtStartPar
Alias for field number 0

\end{fulllineitems}

\index{exit\_message (tyche.Types.Optimum attribute)@\spxentry{exit\_message}\spxextra{tyche.Types.Optimum attribute}}

\begin{fulllineitems}
\phantomsection\label{\detokenize{tyche:tyche.Types.Optimum.exit_message}}
\pysigstartsignatures
\pysigline{\sphinxbfcode{\sphinxupquote{exit\_message}}}
\pysigstopsignatures
\sphinxAtStartPar
Alias for field number 1

\end{fulllineitems}

\index{metrics (tyche.Types.Optimum attribute)@\spxentry{metrics}\spxextra{tyche.Types.Optimum attribute}}

\begin{fulllineitems}
\phantomsection\label{\detokenize{tyche:tyche.Types.Optimum.metrics}}
\pysigstartsignatures
\pysigline{\sphinxbfcode{\sphinxupquote{metrics}}}
\pysigstopsignatures
\sphinxAtStartPar
Alias for field number 3

\end{fulllineitems}

\index{opt\_sense (tyche.Types.Optimum attribute)@\spxentry{opt\_sense}\spxextra{tyche.Types.Optimum attribute}}

\begin{fulllineitems}
\phantomsection\label{\detokenize{tyche:tyche.Types.Optimum.opt_sense}}
\pysigstartsignatures
\pysigline{\sphinxbfcode{\sphinxupquote{opt\_sense}}}
\pysigstopsignatures
\sphinxAtStartPar
Alias for field number 5

\end{fulllineitems}

\index{solve\_time (tyche.Types.Optimum attribute)@\spxentry{solve\_time}\spxextra{tyche.Types.Optimum attribute}}

\begin{fulllineitems}
\phantomsection\label{\detokenize{tyche:tyche.Types.Optimum.solve_time}}
\pysigstartsignatures
\pysigline{\sphinxbfcode{\sphinxupquote{solve\_time}}}
\pysigstopsignatures
\sphinxAtStartPar
Alias for field number 4

\end{fulllineitems}


\end{fulllineitems}

\index{Results (class in tyche.Types)@\spxentry{Results}\spxextra{class in tyche.Types}}

\begin{fulllineitems}
\phantomsection\label{\detokenize{tyche:tyche.Types.Results}}
\pysigstartsignatures
\pysiglinewithargsret{\sphinxbfcode{\sphinxupquote{class\DUrole{w}{  }}}\sphinxcode{\sphinxupquote{tyche.Types.}}\sphinxbfcode{\sphinxupquote{Results}}}{\emph{\DUrole{n}{cost}}, \emph{\DUrole{n}{output}}, \emph{\DUrole{n}{metric}}}{}
\pysigstopsignatures
\sphinxAtStartPar
Bases: \sphinxcode{\sphinxupquote{tuple}}

\sphinxAtStartPar
Named tuple type for rows in the \sphinxstyleemphasis{results} table.
\index{cost (tyche.Types.Results attribute)@\spxentry{cost}\spxextra{tyche.Types.Results attribute}}

\begin{fulllineitems}
\phantomsection\label{\detokenize{tyche:tyche.Types.Results.cost}}
\pysigstartsignatures
\pysigline{\sphinxbfcode{\sphinxupquote{cost}}}
\pysigstopsignatures
\sphinxAtStartPar
Alias for field number 0

\end{fulllineitems}

\index{metric (tyche.Types.Results attribute)@\spxentry{metric}\spxextra{tyche.Types.Results attribute}}

\begin{fulllineitems}
\phantomsection\label{\detokenize{tyche:tyche.Types.Results.metric}}
\pysigstartsignatures
\pysigline{\sphinxbfcode{\sphinxupquote{metric}}}
\pysigstopsignatures
\sphinxAtStartPar
Alias for field number 2

\end{fulllineitems}

\index{output (tyche.Types.Results attribute)@\spxentry{output}\spxextra{tyche.Types.Results attribute}}

\begin{fulllineitems}
\phantomsection\label{\detokenize{tyche:tyche.Types.Results.output}}
\pysigstartsignatures
\pysigline{\sphinxbfcode{\sphinxupquote{output}}}
\pysigstopsignatures
\sphinxAtStartPar
Alias for field number 1

\end{fulllineitems}


\end{fulllineitems}

\index{SynthesizedDistribution (class in tyche.Types)@\spxentry{SynthesizedDistribution}\spxextra{class in tyche.Types}}

\begin{fulllineitems}
\phantomsection\label{\detokenize{tyche:tyche.Types.SynthesizedDistribution}}
\pysigstartsignatures
\pysiglinewithargsret{\sphinxbfcode{\sphinxupquote{class\DUrole{w}{  }}}\sphinxcode{\sphinxupquote{tyche.Types.}}\sphinxbfcode{\sphinxupquote{SynthesizedDistribution}}}{\emph{\DUrole{n}{rvs}}}{}
\pysigstopsignatures
\sphinxAtStartPar
Bases: \sphinxcode{\sphinxupquote{tuple}}

\sphinxAtStartPar
Named tuple type for a synthesized distribution.
\index{rvs (tyche.Types.SynthesizedDistribution attribute)@\spxentry{rvs}\spxextra{tyche.Types.SynthesizedDistribution attribute}}

\begin{fulllineitems}
\phantomsection\label{\detokenize{tyche:tyche.Types.SynthesizedDistribution.rvs}}
\pysigstartsignatures
\pysigline{\sphinxbfcode{\sphinxupquote{rvs}}}
\pysigstopsignatures
\sphinxAtStartPar
Alias for field number 0

\end{fulllineitems}


\end{fulllineitems}



\subsection{Module contents}
\label{\detokenize{tyche:module-tyche}}\label{\detokenize{tyche:module-contents}}\index{module@\spxentry{module}!tyche@\spxentry{tyche}}\index{tyche@\spxentry{tyche}!module@\spxentry{module}}
\sphinxAtStartPar
Tyche: a Python package for R\&D pathways analysis and evaluation.

\sphinxstepscope


\section{Technology Module}
\label{\detokenize{technology:technology-module}}\label{\detokenize{technology:sec-technologymodule}}\label{\detokenize{technology::doc}}

\subsection{Pre\sphinxhyphen{}Built Technology Models and Datasets}
\label{\detokenize{technology:pre-built-technology-models-and-datasets}}

\subsubsection{Residential Photovoltaics}
\label{\detokenize{technology:module-technology.pv_residential_large}}\label{\detokenize{technology:residential-photovoltaics}}\index{module@\spxentry{module}!technology.pv\_residential\_large@\spxentry{technology.pv\_residential\_large}}\index{technology.pv\_residential\_large@\spxentry{technology.pv\_residential\_large}!module@\spxentry{module}}
\sphinxAtStartPar
Generic model for residential PV.

\sphinxAtStartPar
This PV model tracks components, technologies, critical materials, and hazardous waste.


\begin{savenotes}\sphinxattablestart
\centering
\sphinxcapstartof{table}
\sphinxthecaptionisattop
\sphinxcaption{Elements of \sphinxstyleliteralintitle{\sphinxupquote{capital}} arrays.}\label{\detokenize{technology:id1}}
\sphinxaftertopcaption
\begin{tabulary}{\linewidth}[t]{|T|T|T|}
\hline
\sphinxstyletheadfamily 
\sphinxAtStartPar
Index
&\sphinxstyletheadfamily 
\sphinxAtStartPar
Description
&\sphinxstyletheadfamily 
\sphinxAtStartPar
Units
\\
\hline
\sphinxAtStartPar
0
&
\sphinxAtStartPar
module capital cost
&
\sphinxAtStartPar
\$/system
\\
\hline
\sphinxAtStartPar
1
&
\sphinxAtStartPar
inverter capital cost
&
\sphinxAtStartPar
\$/system
\\
\hline
\sphinxAtStartPar
2
&
\sphinxAtStartPar
balance capital cost
&
\sphinxAtStartPar
\$/system
\\
\hline
\end{tabulary}
\par
\sphinxattableend\end{savenotes}


\begin{savenotes}\sphinxattablestart
\centering
\sphinxcapstartof{table}
\sphinxthecaptionisattop
\sphinxcaption{Elements of \sphinxstyleliteralintitle{\sphinxupquote{fixed}} arrays.}\label{\detokenize{technology:id2}}
\sphinxaftertopcaption
\begin{tabulary}{\linewidth}[t]{|T|T|T|}
\hline
\sphinxstyletheadfamily 
\sphinxAtStartPar
Index
&\sphinxstyletheadfamily 
\sphinxAtStartPar
Description
&\sphinxstyletheadfamily 
\sphinxAtStartPar
Units
\\
\hline
\sphinxAtStartPar
0
&
\sphinxAtStartPar
fixed cost
&
\sphinxAtStartPar
\$/system
\\
\hline
\end{tabulary}
\par
\sphinxattableend\end{savenotes}


\begin{savenotes}\sphinxattablestart
\centering
\sphinxcapstartof{table}
\sphinxthecaptionisattop
\sphinxcaption{Elements of \sphinxstyleliteralintitle{\sphinxupquote{input}} arrays.}\label{\detokenize{technology:id3}}
\sphinxaftertopcaption
\begin{tabulary}{\linewidth}[t]{|T|T|T|}
\hline
\sphinxstyletheadfamily 
\sphinxAtStartPar
Index
&\sphinxstyletheadfamily 
\sphinxAtStartPar
Description
&\sphinxstyletheadfamily 
\sphinxAtStartPar
Units
\\
\hline
\sphinxAtStartPar
0
&
\sphinxAtStartPar
strategic metals
&
\sphinxAtStartPar
g/system
\\
\hline
\end{tabulary}
\par
\sphinxattableend\end{savenotes}


\begin{savenotes}\sphinxattablestart
\centering
\sphinxcapstartof{table}
\sphinxthecaptionisattop
\sphinxcaption{Elements of \sphinxstyleliteralintitle{\sphinxupquote{output}} arrays.}\label{\detokenize{technology:id4}}
\sphinxaftertopcaption
\begin{tabulary}{\linewidth}[t]{|T|T|T|}
\hline
\sphinxstyletheadfamily 
\sphinxAtStartPar
Index
&\sphinxstyletheadfamily 
\sphinxAtStartPar
Description
&\sphinxstyletheadfamily 
\sphinxAtStartPar
Units
\\
\hline
\sphinxAtStartPar
0
&
\sphinxAtStartPar
lifetime energy production
&
\sphinxAtStartPar
kWh/system
\\
\hline
\sphinxAtStartPar
1
&
\sphinxAtStartPar
lifecycle hazardous waste
&
\sphinxAtStartPar
g/system
\\
\hline
\sphinxAtStartPar
2
&
\sphinxAtStartPar
lifetime greenhouse gas production
&
\sphinxAtStartPar
gCO2e/system
\\
\hline
\end{tabulary}
\par
\sphinxattableend\end{savenotes}


\begin{savenotes}\sphinxattablestart
\centering
\sphinxcapstartof{table}
\sphinxthecaptionisattop
\sphinxcaption{Elements of \sphinxstyleliteralintitle{\sphinxupquote{metric}} arrays.}\label{\detokenize{technology:id5}}
\sphinxaftertopcaption
\begin{tabulary}{\linewidth}[t]{|T|T|T|}
\hline
\sphinxstyletheadfamily 
\sphinxAtStartPar
Index
&\sphinxstyletheadfamily 
\sphinxAtStartPar
Description
&\sphinxstyletheadfamily 
\sphinxAtStartPar
Units
\\
\hline
\sphinxAtStartPar
0
&
\sphinxAtStartPar
system cost
&
\sphinxAtStartPar
\$/Wdc
\\
\hline
\sphinxAtStartPar
1
&
\sphinxAtStartPar
levelized energy cost
&
\sphinxAtStartPar
\$/kWh
\\
\hline
\sphinxAtStartPar
2
&
\sphinxAtStartPar
greenhouse gas
&
\sphinxAtStartPar
gCO2e/kWh
\\
\hline
\sphinxAtStartPar
3
&
\sphinxAtStartPar
strategic metal
&
\sphinxAtStartPar
g/kWh
\\
\hline
\sphinxAtStartPar
4
&
\sphinxAtStartPar
hazardous waste
&
\sphinxAtStartPar
g/kWh
\\
\hline
\sphinxAtStartPar
5
&
\sphinxAtStartPar
specific yield
&
\sphinxAtStartPar
hr/yr
\\
\hline
\sphinxAtStartPar
6
&
\sphinxAtStartPar
module efficiency
&
\sphinxAtStartPar
\%/100
\\
\hline
\sphinxAtStartPar
7
&
\sphinxAtStartPar
module lifetime
&
\sphinxAtStartPar
yr
\\
\hline
\end{tabulary}
\par
\sphinxattableend\end{savenotes}


\begin{savenotes}\sphinxattablestart
\centering
\sphinxcapstartof{table}
\sphinxthecaptionisattop
\sphinxcaption{Elements of \sphinxstyleliteralintitle{\sphinxupquote{parameter}} arrays.}\label{\detokenize{technology:id6}}
\sphinxaftertopcaption
\begin{tabulary}{\linewidth}[t]{|T|T|T|}
\hline
\sphinxstyletheadfamily 
\sphinxAtStartPar
Index
&\sphinxstyletheadfamily 
\sphinxAtStartPar
Description
&\sphinxstyletheadfamily 
\sphinxAtStartPar
Units
\\
\hline
\sphinxAtStartPar
0
&
\sphinxAtStartPar
discount rate
&
\sphinxAtStartPar
1/yr
\\
\hline
\sphinxAtStartPar
1
&
\sphinxAtStartPar
insolation
&
\sphinxAtStartPar
W/m\textasciicircum{}2
\\
\hline
\sphinxAtStartPar
2
&
\sphinxAtStartPar
system size
&
\sphinxAtStartPar
m\textasciicircum{}2
\\
\hline
\sphinxAtStartPar
3
&
\sphinxAtStartPar
module capital cost
&
\sphinxAtStartPar
\$/m\textasciicircum{}2
\\
\hline
\sphinxAtStartPar
4
&
\sphinxAtStartPar
module lifetime
&
\sphinxAtStartPar
yr
\\
\hline
\sphinxAtStartPar
5
&
\sphinxAtStartPar
module efficiency
&
\sphinxAtStartPar
\%/100
\\
\hline
\sphinxAtStartPar
6
&
\sphinxAtStartPar
module aperture
&
\sphinxAtStartPar
\%/100
\\
\hline
\sphinxAtStartPar
7
&
\sphinxAtStartPar
module fixed cost
&
\sphinxAtStartPar
\$/kW/yr
\\
\hline
\sphinxAtStartPar
8
&
\sphinxAtStartPar
module degradation rate
&
\sphinxAtStartPar
1/yr
\\
\hline
\sphinxAtStartPar
9
&
\sphinxAtStartPar
location capacity factor
&
\sphinxAtStartPar
\%/100
\\
\hline
\sphinxAtStartPar
10
&
\sphinxAtStartPar
module soiling loss
&
\sphinxAtStartPar
\%/100
\\
\hline
\sphinxAtStartPar
11
&
\sphinxAtStartPar
inverter capital cost
&
\sphinxAtStartPar
\$/W
\\
\hline
\sphinxAtStartPar
12
&
\sphinxAtStartPar
inverter lifetime
&
\sphinxAtStartPar
yr
\\
\hline
\sphinxAtStartPar
13
&
\sphinxAtStartPar
inverter replacement cost
&
\sphinxAtStartPar
\%/100
\\
\hline
\sphinxAtStartPar
14
&
\sphinxAtStartPar
inverter efficiency
&
\sphinxAtStartPar
\%/100
\\
\hline
\sphinxAtStartPar
15
&
\sphinxAtStartPar
hardware capital cost
&
\sphinxAtStartPar
\$/m\textasciicircum{}2
\\
\hline
\sphinxAtStartPar
16
&
\sphinxAtStartPar
installation labor cost
&
\sphinxAtStartPar
\$/system
\\
\hline
\sphinxAtStartPar
17
&
\sphinxAtStartPar
permitting cost
&
\sphinxAtStartPar
\$/system
\\
\hline
\sphinxAtStartPar
18
&
\sphinxAtStartPar
customer acquisition cost
&
\sphinxAtStartPar
\$/system
\\
\hline
\sphinxAtStartPar
19
&
\sphinxAtStartPar
installer overhead cost
&
\sphinxAtStartPar
\%/100
\\
\hline
\sphinxAtStartPar
20
&
\sphinxAtStartPar
hazardous waste content
&
\sphinxAtStartPar
g/m\textasciicircum{}2
\\
\hline
\sphinxAtStartPar
21
&
\sphinxAtStartPar
greenhouse gas offset
&
\sphinxAtStartPar
gCO2e/kWh
\\
\hline
\sphinxAtStartPar
22
&
\sphinxAtStartPar
benchmark LCOC
&
\sphinxAtStartPar
\$/Wdc
\\
\hline
\sphinxAtStartPar
23
&
\sphinxAtStartPar
benchmark LCOE
&
\sphinxAtStartPar
\$/kWh
\\
\hline
\end{tabulary}
\par
\sphinxattableend\end{savenotes}
\index{capital\_cost() (in module technology.pv\_residential\_large)@\spxentry{capital\_cost()}\spxextra{in module technology.pv\_residential\_large}}

\begin{fulllineitems}
\phantomsection\label{\detokenize{technology:technology.pv_residential_large.capital_cost}}
\pysigstartsignatures
\pysiglinewithargsret{\sphinxcode{\sphinxupquote{technology.pv\_residential\_large.}}\sphinxbfcode{\sphinxupquote{capital\_cost}}}{\emph{\DUrole{n}{scale}}, \emph{\DUrole{n}{parameter}}}{}
\pysigstopsignatures
\sphinxAtStartPar
Capital cost function.
\begin{quote}\begin{description}
\sphinxlineitem{Parameters}\begin{itemize}
\item {} 
\sphinxAtStartPar
\sphinxstyleliteralstrong{\sphinxupquote{scale}} (\sphinxstyleliteralemphasis{\sphinxupquote{float}}) – The scale of operation.

\item {} 
\sphinxAtStartPar
\sphinxstyleliteralstrong{\sphinxupquote{parameter}} (\sphinxstyleliteralemphasis{\sphinxupquote{array}}) – The technological parameterization.

\end{itemize}

\end{description}\end{quote}

\end{fulllineitems}

\index{discount() (in module technology.pv\_residential\_large)@\spxentry{discount()}\spxextra{in module technology.pv\_residential\_large}}

\begin{fulllineitems}
\phantomsection\label{\detokenize{technology:technology.pv_residential_large.discount}}
\pysigstartsignatures
\pysiglinewithargsret{\sphinxcode{\sphinxupquote{technology.pv\_residential\_large.}}\sphinxbfcode{\sphinxupquote{discount}}}{\emph{\DUrole{n}{rate}}, \emph{\DUrole{n}{time}}}{}
\pysigstopsignatures
\sphinxAtStartPar
Discount factor over a time period.
\begin{quote}\begin{description}
\sphinxlineitem{Parameters}\begin{itemize}
\item {} 
\sphinxAtStartPar
\sphinxstyleliteralstrong{\sphinxupquote{rate}} (\sphinxstyleliteralemphasis{\sphinxupquote{float}}) – The discount rate per time period.

\item {} 
\sphinxAtStartPar
\sphinxstyleliteralstrong{\sphinxupquote{time}} (\sphinxstyleliteralemphasis{\sphinxupquote{int}}) – The number of time periods.

\end{itemize}

\end{description}\end{quote}

\end{fulllineitems}

\index{fixed\_cost() (in module technology.pv\_residential\_large)@\spxentry{fixed\_cost()}\spxextra{in module technology.pv\_residential\_large}}

\begin{fulllineitems}
\phantomsection\label{\detokenize{technology:technology.pv_residential_large.fixed_cost}}
\pysigstartsignatures
\pysiglinewithargsret{\sphinxcode{\sphinxupquote{technology.pv\_residential\_large.}}\sphinxbfcode{\sphinxupquote{fixed\_cost}}}{\emph{\DUrole{n}{scale}}, \emph{\DUrole{n}{parameter}}}{}
\pysigstopsignatures
\sphinxAtStartPar
Fixed cost function.
\begin{quote}\begin{description}
\sphinxlineitem{Parameters}\begin{itemize}
\item {} 
\sphinxAtStartPar
\sphinxstyleliteralstrong{\sphinxupquote{scale}} (\sphinxstyleliteralemphasis{\sphinxupquote{float}}) – The scale of operation.

\item {} 
\sphinxAtStartPar
\sphinxstyleliteralstrong{\sphinxupquote{parameter}} (\sphinxstyleliteralemphasis{\sphinxupquote{array}}) – The technological parameterization.

\end{itemize}

\end{description}\end{quote}

\end{fulllineitems}

\index{metrics() (in module technology.pv\_residential\_large)@\spxentry{metrics()}\spxextra{in module technology.pv\_residential\_large}}

\begin{fulllineitems}
\phantomsection\label{\detokenize{technology:technology.pv_residential_large.metrics}}
\pysigstartsignatures
\pysiglinewithargsret{\sphinxcode{\sphinxupquote{technology.pv\_residential\_large.}}\sphinxbfcode{\sphinxupquote{metrics}}}{\emph{\DUrole{n}{scale}}, \emph{\DUrole{n}{capital}}, \emph{\DUrole{n}{lifetime}}, \emph{\DUrole{n}{fixed}}, \emph{\DUrole{n}{input\_raw}}, \emph{\DUrole{n}{input}}, \emph{\DUrole{n}{input\_price}}, \emph{\DUrole{n}{output\_raw}}, \emph{\DUrole{n}{output}}, \emph{\DUrole{n}{cost}}, \emph{\DUrole{n}{parameter}}}{}
\pysigstopsignatures
\sphinxAtStartPar
Metrics function.
\begin{quote}\begin{description}
\sphinxlineitem{Parameters}\begin{itemize}
\item {} 
\sphinxAtStartPar
\sphinxstyleliteralstrong{\sphinxupquote{scale}} (\sphinxstyleliteralemphasis{\sphinxupquote{float}}) – The scale of operation.

\item {} 
\sphinxAtStartPar
\sphinxstyleliteralstrong{\sphinxupquote{capital}} (\sphinxstyleliteralemphasis{\sphinxupquote{array}}) – Capital costs.

\item {} 
\sphinxAtStartPar
\sphinxstyleliteralstrong{\sphinxupquote{lifetime}} (\sphinxstyleliteralemphasis{\sphinxupquote{float}}) – Technology lifetime.

\item {} 
\sphinxAtStartPar
\sphinxstyleliteralstrong{\sphinxupquote{fixed}} (\sphinxstyleliteralemphasis{\sphinxupquote{array}}) – Fixed costs.

\item {} 
\sphinxAtStartPar
\sphinxstyleliteralstrong{\sphinxupquote{input\_raw}} (\sphinxstyleliteralemphasis{\sphinxupquote{array}}) – Raw input quantities (before losses).

\item {} 
\sphinxAtStartPar
\sphinxstyleliteralstrong{\sphinxupquote{input}} (\sphinxstyleliteralemphasis{\sphinxupquote{array}}) – Input quantities.

\item {} 
\sphinxAtStartPar
\sphinxstyleliteralstrong{\sphinxupquote{output\_raw}} (\sphinxstyleliteralemphasis{\sphinxupquote{array}}) – Raw output quantities (before losses).

\item {} 
\sphinxAtStartPar
\sphinxstyleliteralstrong{\sphinxupquote{output}} (\sphinxstyleliteralemphasis{\sphinxupquote{array}}) – Output quantities.

\item {} 
\sphinxAtStartPar
\sphinxstyleliteralstrong{\sphinxupquote{cost}} (\sphinxstyleliteralemphasis{\sphinxupquote{array}}) – Costs.

\item {} 
\sphinxAtStartPar
\sphinxstyleliteralstrong{\sphinxupquote{parameter}} (\sphinxstyleliteralemphasis{\sphinxupquote{array}}) – The technological parameterization.

\end{itemize}

\end{description}\end{quote}

\end{fulllineitems}

\index{module\_power() (in module technology.pv\_residential\_large)@\spxentry{module\_power()}\spxextra{in module technology.pv\_residential\_large}}

\begin{fulllineitems}
\phantomsection\label{\detokenize{technology:technology.pv_residential_large.module_power}}
\pysigstartsignatures
\pysiglinewithargsret{\sphinxcode{\sphinxupquote{technology.pv\_residential\_large.}}\sphinxbfcode{\sphinxupquote{module\_power}}}{\emph{\DUrole{n}{parameter}}}{}
\pysigstopsignatures
\sphinxAtStartPar
Nominal module energy production.
\begin{quote}\begin{description}
\sphinxlineitem{Parameters}
\sphinxAtStartPar
\sphinxstyleliteralstrong{\sphinxupquote{parameter}} (\sphinxstyleliteralemphasis{\sphinxupquote{array}}) – The technological parameterization.

\end{description}\end{quote}

\end{fulllineitems}

\index{npv() (in module technology.pv\_residential\_large)@\spxentry{npv()}\spxextra{in module technology.pv\_residential\_large}}

\begin{fulllineitems}
\phantomsection\label{\detokenize{technology:technology.pv_residential_large.npv}}
\pysigstartsignatures
\pysiglinewithargsret{\sphinxcode{\sphinxupquote{technology.pv\_residential\_large.}}\sphinxbfcode{\sphinxupquote{npv}}}{\emph{\DUrole{n}{rate}}, \emph{\DUrole{n}{time}}}{}
\pysigstopsignatures
\sphinxAtStartPar
Net present value of constant cash flow.
\begin{quote}\begin{description}
\sphinxlineitem{Parameters}\begin{itemize}
\item {} 
\sphinxAtStartPar
\sphinxstyleliteralstrong{\sphinxupquote{rate}} (\sphinxstyleliteralemphasis{\sphinxupquote{float}}) – The discount rate per time period.

\item {} 
\sphinxAtStartPar
\sphinxstyleliteralstrong{\sphinxupquote{time}} (\sphinxstyleliteralemphasis{\sphinxupquote{int}}) – The number of time periods.

\end{itemize}

\end{description}\end{quote}

\end{fulllineitems}

\index{performance\_ratio() (in module technology.pv\_residential\_large)@\spxentry{performance\_ratio()}\spxextra{in module technology.pv\_residential\_large}}

\begin{fulllineitems}
\phantomsection\label{\detokenize{technology:technology.pv_residential_large.performance_ratio}}
\pysigstartsignatures
\pysiglinewithargsret{\sphinxcode{\sphinxupquote{technology.pv\_residential\_large.}}\sphinxbfcode{\sphinxupquote{performance\_ratio}}}{\emph{\DUrole{n}{parameter}}}{}
\pysigstopsignatures
\sphinxAtStartPar
Performance ratio for the system.
\begin{quote}\begin{description}
\sphinxlineitem{Parameters}
\sphinxAtStartPar
\sphinxstyleliteralstrong{\sphinxupquote{parameter}} (\sphinxstyleliteralemphasis{\sphinxupquote{array}}) – The technological parameterization.

\end{description}\end{quote}

\end{fulllineitems}

\index{production() (in module technology.pv\_residential\_large)@\spxentry{production()}\spxextra{in module technology.pv\_residential\_large}}

\begin{fulllineitems}
\phantomsection\label{\detokenize{technology:technology.pv_residential_large.production}}
\pysigstartsignatures
\pysiglinewithargsret{\sphinxcode{\sphinxupquote{technology.pv\_residential\_large.}}\sphinxbfcode{\sphinxupquote{production}}}{\emph{\DUrole{n}{scale}}, \emph{\DUrole{n}{capital}}, \emph{\DUrole{n}{lifetime}}, \emph{\DUrole{n}{fixed}}, \emph{\DUrole{n}{input}}, \emph{\DUrole{n}{parameter}}}{}
\pysigstopsignatures
\sphinxAtStartPar
Production function.
\begin{quote}\begin{description}
\sphinxlineitem{Parameters}\begin{itemize}
\item {} 
\sphinxAtStartPar
\sphinxstyleliteralstrong{\sphinxupquote{scale}} (\sphinxstyleliteralemphasis{\sphinxupquote{float}}) – The scale of operation.

\item {} 
\sphinxAtStartPar
\sphinxstyleliteralstrong{\sphinxupquote{capital}} (\sphinxstyleliteralemphasis{\sphinxupquote{array}}) – Capital costs.

\item {} 
\sphinxAtStartPar
\sphinxstyleliteralstrong{\sphinxupquote{lifetime}} (\sphinxstyleliteralemphasis{\sphinxupquote{float}}) – Technology lifetime.

\item {} 
\sphinxAtStartPar
\sphinxstyleliteralstrong{\sphinxupquote{fixed}} (\sphinxstyleliteralemphasis{\sphinxupquote{array}}) – Fixed costs.

\item {} 
\sphinxAtStartPar
\sphinxstyleliteralstrong{\sphinxupquote{input}} (\sphinxstyleliteralemphasis{\sphinxupquote{array}}) – Input quantities.

\item {} 
\sphinxAtStartPar
\sphinxstyleliteralstrong{\sphinxupquote{parameter}} (\sphinxstyleliteralemphasis{\sphinxupquote{array}}) – The technological parameterization.

\end{itemize}

\end{description}\end{quote}

\end{fulllineitems}

\index{specific\_yield() (in module technology.pv\_residential\_large)@\spxentry{specific\_yield()}\spxextra{in module technology.pv\_residential\_large}}

\begin{fulllineitems}
\phantomsection\label{\detokenize{technology:technology.pv_residential_large.specific_yield}}
\pysigstartsignatures
\pysiglinewithargsret{\sphinxcode{\sphinxupquote{technology.pv\_residential\_large.}}\sphinxbfcode{\sphinxupquote{specific\_yield}}}{\emph{\DUrole{n}{parameter}}}{}
\pysigstopsignatures
\sphinxAtStartPar
Specific yield for the system.
\begin{quote}\begin{description}
\sphinxlineitem{Parameters}
\sphinxAtStartPar
\sphinxstyleliteralstrong{\sphinxupquote{parameter}} (\sphinxstyleliteralemphasis{\sphinxupquote{array}}) – The technological parameterization.

\end{description}\end{quote}

\end{fulllineitems}



\subsubsection{Simple Residential Photovoltaics}
\label{\detokenize{technology:module-technology.pv_residential_simple}}\label{\detokenize{technology:simple-residential-photovoltaics}}\label{\detokenize{technology:sec-simplerespv}}\index{module@\spxentry{module}!technology.pv\_residential\_simple@\spxentry{technology.pv\_residential\_simple}}\index{technology.pv\_residential\_simple@\spxentry{technology.pv\_residential\_simple}!module@\spxentry{module}}
\sphinxAtStartPar
Simple residential PV.
\index{capital\_cost() (in module technology.pv\_residential\_simple)@\spxentry{capital\_cost()}\spxextra{in module technology.pv\_residential\_simple}}

\begin{fulllineitems}
\phantomsection\label{\detokenize{technology:technology.pv_residential_simple.capital_cost}}
\pysigstartsignatures
\pysiglinewithargsret{\sphinxcode{\sphinxupquote{technology.pv\_residential\_simple.}}\sphinxbfcode{\sphinxupquote{capital\_cost}}}{\emph{\DUrole{n}{scale}}, \emph{\DUrole{n}{parameter}}}{}
\pysigstopsignatures
\sphinxAtStartPar
Capital cost function.
\begin{quote}\begin{description}
\sphinxlineitem{Parameters}\begin{itemize}
\item {} 
\sphinxAtStartPar
\sphinxstyleliteralstrong{\sphinxupquote{scale}} (\sphinxstyleliteralemphasis{\sphinxupquote{float}}) – The scale of operation.

\item {} 
\sphinxAtStartPar
\sphinxstyleliteralstrong{\sphinxupquote{parameter}} (\sphinxstyleliteralemphasis{\sphinxupquote{array}}) – The technological parameterization.

\end{itemize}

\end{description}\end{quote}

\end{fulllineitems}

\index{discount() (in module technology.pv\_residential\_simple)@\spxentry{discount()}\spxextra{in module technology.pv\_residential\_simple}}

\begin{fulllineitems}
\phantomsection\label{\detokenize{technology:technology.pv_residential_simple.discount}}
\pysigstartsignatures
\pysiglinewithargsret{\sphinxcode{\sphinxupquote{technology.pv\_residential\_simple.}}\sphinxbfcode{\sphinxupquote{discount}}}{\emph{\DUrole{n}{rate}}, \emph{\DUrole{n}{time}}}{}
\pysigstopsignatures
\sphinxAtStartPar
Discount factor over a time period.
\begin{quote}\begin{description}
\sphinxlineitem{Parameters}\begin{itemize}
\item {} 
\sphinxAtStartPar
\sphinxstyleliteralstrong{\sphinxupquote{rate}} (\sphinxstyleliteralemphasis{\sphinxupquote{float}}) – The discount rate per time period.

\item {} 
\sphinxAtStartPar
\sphinxstyleliteralstrong{\sphinxupquote{time}} (\sphinxstyleliteralemphasis{\sphinxupquote{int}}) – The number of time periods.

\end{itemize}

\end{description}\end{quote}

\end{fulllineitems}

\index{fixed\_cost() (in module technology.pv\_residential\_simple)@\spxentry{fixed\_cost()}\spxextra{in module technology.pv\_residential\_simple}}

\begin{fulllineitems}
\phantomsection\label{\detokenize{technology:technology.pv_residential_simple.fixed_cost}}
\pysigstartsignatures
\pysiglinewithargsret{\sphinxcode{\sphinxupquote{technology.pv\_residential\_simple.}}\sphinxbfcode{\sphinxupquote{fixed\_cost}}}{\emph{\DUrole{n}{scale}}, \emph{\DUrole{n}{parameter}}}{}
\pysigstopsignatures
\sphinxAtStartPar
Fixed cost function.
\begin{quote}\begin{description}
\sphinxlineitem{Parameters}\begin{itemize}
\item {} 
\sphinxAtStartPar
\sphinxstyleliteralstrong{\sphinxupquote{scale}} (\sphinxstyleliteralemphasis{\sphinxupquote{float}}) – The scale of operation.

\item {} 
\sphinxAtStartPar
\sphinxstyleliteralstrong{\sphinxupquote{parameter}} (\sphinxstyleliteralemphasis{\sphinxupquote{array}}) – The technological parameterization.

\end{itemize}

\end{description}\end{quote}

\end{fulllineitems}

\index{metrics() (in module technology.pv\_residential\_simple)@\spxentry{metrics()}\spxextra{in module technology.pv\_residential\_simple}}

\begin{fulllineitems}
\phantomsection\label{\detokenize{technology:technology.pv_residential_simple.metrics}}
\pysigstartsignatures
\pysiglinewithargsret{\sphinxcode{\sphinxupquote{technology.pv\_residential\_simple.}}\sphinxbfcode{\sphinxupquote{metrics}}}{\emph{\DUrole{n}{scale}}, \emph{\DUrole{n}{capital}}, \emph{\DUrole{n}{lifetime}}, \emph{\DUrole{n}{fixed}}, \emph{\DUrole{n}{input\_raw}}, \emph{\DUrole{n}{input}}, \emph{\DUrole{n}{input\_price}}, \emph{\DUrole{n}{output\_raw}}, \emph{\DUrole{n}{output}}, \emph{\DUrole{n}{cost}}, \emph{\DUrole{n}{parameter}}}{}
\pysigstopsignatures
\sphinxAtStartPar
Metrics function.
\begin{quote}\begin{description}
\sphinxlineitem{Parameters}\begin{itemize}
\item {} 
\sphinxAtStartPar
\sphinxstyleliteralstrong{\sphinxupquote{scale}} (\sphinxstyleliteralemphasis{\sphinxupquote{float}}) – The scale of operation.

\item {} 
\sphinxAtStartPar
\sphinxstyleliteralstrong{\sphinxupquote{capital}} (\sphinxstyleliteralemphasis{\sphinxupquote{array}}) – Capital costs.

\item {} 
\sphinxAtStartPar
\sphinxstyleliteralstrong{\sphinxupquote{lifetime}} (\sphinxstyleliteralemphasis{\sphinxupquote{float}}) – Technology lifetime.

\item {} 
\sphinxAtStartPar
\sphinxstyleliteralstrong{\sphinxupquote{fixed}} (\sphinxstyleliteralemphasis{\sphinxupquote{array}}) – Fixed costs.

\item {} 
\sphinxAtStartPar
\sphinxstyleliteralstrong{\sphinxupquote{input\_raw}} (\sphinxstyleliteralemphasis{\sphinxupquote{array}}) – Raw input quantities (before losses).

\item {} 
\sphinxAtStartPar
\sphinxstyleliteralstrong{\sphinxupquote{input}} (\sphinxstyleliteralemphasis{\sphinxupquote{array}}) – Input quantities.

\item {} 
\sphinxAtStartPar
\sphinxstyleliteralstrong{\sphinxupquote{output\_raw}} (\sphinxstyleliteralemphasis{\sphinxupquote{array}}) – Raw output quantities (before losses).

\item {} 
\sphinxAtStartPar
\sphinxstyleliteralstrong{\sphinxupquote{output}} (\sphinxstyleliteralemphasis{\sphinxupquote{array}}) – Output quantities.

\item {} 
\sphinxAtStartPar
\sphinxstyleliteralstrong{\sphinxupquote{cost}} (\sphinxstyleliteralemphasis{\sphinxupquote{array}}) – Costs.

\item {} 
\sphinxAtStartPar
\sphinxstyleliteralstrong{\sphinxupquote{parameter}} (\sphinxstyleliteralemphasis{\sphinxupquote{array}}) – The technological parameterization.

\end{itemize}

\end{description}\end{quote}

\end{fulllineitems}

\index{npv() (in module technology.pv\_residential\_simple)@\spxentry{npv()}\spxextra{in module technology.pv\_residential\_simple}}

\begin{fulllineitems}
\phantomsection\label{\detokenize{technology:technology.pv_residential_simple.npv}}
\pysigstartsignatures
\pysiglinewithargsret{\sphinxcode{\sphinxupquote{technology.pv\_residential\_simple.}}\sphinxbfcode{\sphinxupquote{npv}}}{\emph{\DUrole{n}{rate}}, \emph{\DUrole{n}{time}}}{}
\pysigstopsignatures
\sphinxAtStartPar
Net present value of constant cash flow.
\begin{quote}\begin{description}
\sphinxlineitem{Parameters}\begin{itemize}
\item {} 
\sphinxAtStartPar
\sphinxstyleliteralstrong{\sphinxupquote{rate}} (\sphinxstyleliteralemphasis{\sphinxupquote{float}}) – The discount rate per time period.

\item {} 
\sphinxAtStartPar
\sphinxstyleliteralstrong{\sphinxupquote{time}} (\sphinxstyleliteralemphasis{\sphinxupquote{int}}) – The number of time periods.

\end{itemize}

\end{description}\end{quote}

\end{fulllineitems}

\index{production() (in module technology.pv\_residential\_simple)@\spxentry{production()}\spxextra{in module technology.pv\_residential\_simple}}

\begin{fulllineitems}
\phantomsection\label{\detokenize{technology:technology.pv_residential_simple.production}}
\pysigstartsignatures
\pysiglinewithargsret{\sphinxcode{\sphinxupquote{technology.pv\_residential\_simple.}}\sphinxbfcode{\sphinxupquote{production}}}{\emph{\DUrole{n}{scale}}, \emph{\DUrole{n}{capital}}, \emph{\DUrole{n}{lifetime}}, \emph{\DUrole{n}{fixed}}, \emph{\DUrole{n}{input}}, \emph{\DUrole{n}{parameter}}}{}
\pysigstopsignatures
\sphinxAtStartPar
Production function.
\begin{quote}\begin{description}
\sphinxlineitem{Parameters}\begin{itemize}
\item {} 
\sphinxAtStartPar
\sphinxstyleliteralstrong{\sphinxupquote{scale}} (\sphinxstyleliteralemphasis{\sphinxupquote{float}}) – The scale of operation.

\item {} 
\sphinxAtStartPar
\sphinxstyleliteralstrong{\sphinxupquote{capital}} (\sphinxstyleliteralemphasis{\sphinxupquote{array}}) – Capital costs.

\item {} 
\sphinxAtStartPar
\sphinxstyleliteralstrong{\sphinxupquote{lifetime}} (\sphinxstyleliteralemphasis{\sphinxupquote{float}}) – Technology lifetime.

\item {} 
\sphinxAtStartPar
\sphinxstyleliteralstrong{\sphinxupquote{fixed}} (\sphinxstyleliteralemphasis{\sphinxupquote{array}}) – Fixed costs.

\item {} 
\sphinxAtStartPar
\sphinxstyleliteralstrong{\sphinxupquote{input}} (\sphinxstyleliteralemphasis{\sphinxupquote{array}}) – Input quantities.

\item {} 
\sphinxAtStartPar
\sphinxstyleliteralstrong{\sphinxupquote{parameter}} (\sphinxstyleliteralemphasis{\sphinxupquote{array}}) – The technological parameterization.

\end{itemize}

\end{description}\end{quote}

\end{fulllineitems}



\subsubsection{Utility\sphinxhyphen{}Scale Photovoltaics}
\label{\detokenize{technology:module-technology.utility_pv}}\label{\detokenize{technology:utility-scale-photovoltaics}}\index{module@\spxentry{module}!technology.utility\_pv@\spxentry{technology.utility\_pv}}\index{technology.utility\_pv@\spxentry{technology.utility\_pv}!module@\spxentry{module}}
\sphinxAtStartPar
Simple pv utility\sphinxhyphen{}scale module example.  Inspired by Kavlak et al. Energy Policy 123 (2018) 700–710.
\index{capital\_cost() (in module technology.utility\_pv)@\spxentry{capital\_cost()}\spxextra{in module technology.utility\_pv}}

\begin{fulllineitems}
\phantomsection\label{\detokenize{technology:technology.utility_pv.capital_cost}}
\pysigstartsignatures
\pysiglinewithargsret{\sphinxcode{\sphinxupquote{technology.utility\_pv.}}\sphinxbfcode{\sphinxupquote{capital\_cost}}}{\emph{\DUrole{n}{scale}}, \emph{\DUrole{n}{parameter}}}{}
\pysigstopsignatures
\sphinxAtStartPar
Capital cost function.
\begin{quote}\begin{description}
\sphinxlineitem{Parameters}\begin{itemize}
\item {} 
\sphinxAtStartPar
\sphinxstyleliteralstrong{\sphinxupquote{scale}} (\sphinxstyleliteralemphasis{\sphinxupquote{float}}) – The scale of operation.

\item {} 
\sphinxAtStartPar
\sphinxstyleliteralstrong{\sphinxupquote{parameter}} (\sphinxstyleliteralemphasis{\sphinxupquote{array}}) – The technological parameterization.

\end{itemize}

\end{description}\end{quote}

\end{fulllineitems}

\index{fixed\_cost() (in module technology.utility\_pv)@\spxentry{fixed\_cost()}\spxextra{in module technology.utility\_pv}}

\begin{fulllineitems}
\phantomsection\label{\detokenize{technology:technology.utility_pv.fixed_cost}}
\pysigstartsignatures
\pysiglinewithargsret{\sphinxcode{\sphinxupquote{technology.utility\_pv.}}\sphinxbfcode{\sphinxupquote{fixed\_cost}}}{\emph{\DUrole{n}{scale}}, \emph{\DUrole{n}{parameter}}}{}
\pysigstopsignatures
\sphinxAtStartPar
Fixed cost function.
\begin{quote}\begin{description}
\sphinxlineitem{Parameters}\begin{itemize}
\item {} 
\sphinxAtStartPar
\sphinxstyleliteralstrong{\sphinxupquote{scale}} (\sphinxstyleliteralemphasis{\sphinxupquote{float}}) – The scale of operation.

\item {} 
\sphinxAtStartPar
\sphinxstyleliteralstrong{\sphinxupquote{parameter}} (\sphinxstyleliteralemphasis{\sphinxupquote{array}}) – The technological parameterization.

\end{itemize}

\end{description}\end{quote}

\end{fulllineitems}

\index{metrics() (in module technology.utility\_pv)@\spxentry{metrics()}\spxextra{in module technology.utility\_pv}}

\begin{fulllineitems}
\phantomsection\label{\detokenize{technology:technology.utility_pv.metrics}}
\pysigstartsignatures
\pysiglinewithargsret{\sphinxcode{\sphinxupquote{technology.utility\_pv.}}\sphinxbfcode{\sphinxupquote{metrics}}}{\emph{\DUrole{n}{scale}}, \emph{\DUrole{n}{capital}}, \emph{\DUrole{n}{lifetime}}, \emph{\DUrole{n}{fixed}}, \emph{\DUrole{n}{input\_raw}}, \emph{\DUrole{n}{input}}, \emph{\DUrole{n}{input\_price}}, \emph{\DUrole{n}{output\_raw}}, \emph{\DUrole{n}{output}}, \emph{\DUrole{n}{cost}}, \emph{\DUrole{n}{parameter}}}{}
\pysigstopsignatures
\sphinxAtStartPar
Metrics function.
\begin{quote}\begin{description}
\sphinxlineitem{Parameters}\begin{itemize}
\item {} 
\sphinxAtStartPar
\sphinxstyleliteralstrong{\sphinxupquote{scale}} (\sphinxstyleliteralemphasis{\sphinxupquote{float}}) – The scale of operation.

\item {} 
\sphinxAtStartPar
\sphinxstyleliteralstrong{\sphinxupquote{capital}} (\sphinxstyleliteralemphasis{\sphinxupquote{array}}) – Capital costs.

\item {} 
\sphinxAtStartPar
\sphinxstyleliteralstrong{\sphinxupquote{lifetime}} (\sphinxstyleliteralemphasis{\sphinxupquote{float}}) – Technology lifetime.

\item {} 
\sphinxAtStartPar
\sphinxstyleliteralstrong{\sphinxupquote{fixed}} (\sphinxstyleliteralemphasis{\sphinxupquote{array}}) – Fixed costs.

\item {} 
\sphinxAtStartPar
\sphinxstyleliteralstrong{\sphinxupquote{input\_raw}} (\sphinxstyleliteralemphasis{\sphinxupquote{array}}) – Raw input quantities (before losses).

\item {} 
\sphinxAtStartPar
\sphinxstyleliteralstrong{\sphinxupquote{input}} (\sphinxstyleliteralemphasis{\sphinxupquote{array}}) – Input quantities.

\item {} 
\sphinxAtStartPar
\sphinxstyleliteralstrong{\sphinxupquote{output\_raw}} (\sphinxstyleliteralemphasis{\sphinxupquote{array}}) – Raw output quantities (before losses).

\item {} 
\sphinxAtStartPar
\sphinxstyleliteralstrong{\sphinxupquote{output}} (\sphinxstyleliteralemphasis{\sphinxupquote{array}}) – Output quantities.

\item {} 
\sphinxAtStartPar
\sphinxstyleliteralstrong{\sphinxupquote{cost}} (\sphinxstyleliteralemphasis{\sphinxupquote{array}}) – Costs.

\item {} 
\sphinxAtStartPar
\sphinxstyleliteralstrong{\sphinxupquote{parameter}} (\sphinxstyleliteralemphasis{\sphinxupquote{array}}) – The technological parameterization.

\end{itemize}

\end{description}\end{quote}

\end{fulllineitems}

\index{production() (in module technology.utility\_pv)@\spxentry{production()}\spxextra{in module technology.utility\_pv}}

\begin{fulllineitems}
\phantomsection\label{\detokenize{technology:technology.utility_pv.production}}
\pysigstartsignatures
\pysiglinewithargsret{\sphinxcode{\sphinxupquote{technology.utility\_pv.}}\sphinxbfcode{\sphinxupquote{production}}}{\emph{\DUrole{n}{scale}}, \emph{\DUrole{n}{capital}}, \emph{\DUrole{n}{lifetime}}, \emph{\DUrole{n}{fixed}}, \emph{\DUrole{n}{input}}, \emph{\DUrole{n}{parameter}}}{}
\pysigstopsignatures
\sphinxAtStartPar
Production function.
\begin{quote}\begin{description}
\sphinxlineitem{Parameters}\begin{itemize}
\item {} 
\sphinxAtStartPar
\sphinxstyleliteralstrong{\sphinxupquote{scale}} (\sphinxstyleliteralemphasis{\sphinxupquote{float}}) – The scale of operation.

\item {} 
\sphinxAtStartPar
\sphinxstyleliteralstrong{\sphinxupquote{capital}} (\sphinxstyleliteralemphasis{\sphinxupquote{array}}) – Capital costs.

\item {} 
\sphinxAtStartPar
\sphinxstyleliteralstrong{\sphinxupquote{lifetime}} (\sphinxstyleliteralemphasis{\sphinxupquote{float}}) – Technology lifetime.

\item {} 
\sphinxAtStartPar
\sphinxstyleliteralstrong{\sphinxupquote{fixed}} (\sphinxstyleliteralemphasis{\sphinxupquote{array}}) – Fixed costs.

\item {} 
\sphinxAtStartPar
\sphinxstyleliteralstrong{\sphinxupquote{input}} (\sphinxstyleliteralemphasis{\sphinxupquote{array}}) – Input quantities.

\item {} 
\sphinxAtStartPar
\sphinxstyleliteralstrong{\sphinxupquote{parameter}} (\sphinxstyleliteralemphasis{\sphinxupquote{array}}) – The technological parameterization.

\end{itemize}

\end{description}\end{quote}

\end{fulllineitems}



\subsubsection{Transportation}
\label{\detokenize{technology:module-technology.transport_model}}\label{\detokenize{technology:transportation}}\index{module@\spxentry{module}!technology.transport\_model@\spxentry{technology.transport\_model}}\index{technology.transport\_model@\spxentry{technology.transport\_model}!module@\spxentry{module}}
\sphinxAtStartPar
Phase\sphinxhyphen{}1 model to estimate the cost, energy, and emissions associated with a
particular vehicle/transport technology.
\index{capital\_cost() (in module technology.transport\_model)@\spxentry{capital\_cost()}\spxextra{in module technology.transport\_model}}

\begin{fulllineitems}
\phantomsection\label{\detokenize{technology:technology.transport_model.capital_cost}}
\pysigstartsignatures
\pysiglinewithargsret{\sphinxcode{\sphinxupquote{technology.transport\_model.}}\sphinxbfcode{\sphinxupquote{capital\_cost}}}{\emph{\DUrole{n}{scale}}, \emph{\DUrole{n}{parameter}}}{}
\pysigstopsignatures
\sphinxAtStartPar
Capital cost function.
\begin{quote}\begin{description}
\sphinxlineitem{Parameters}\begin{itemize}
\item {} 
\sphinxAtStartPar
\sphinxstyleliteralstrong{\sphinxupquote{scale}} (\sphinxstyleliteralemphasis{\sphinxupquote{float}}) – The scale of operation.

\item {} 
\sphinxAtStartPar
\sphinxstyleliteralstrong{\sphinxupquote{parameter}} (\sphinxstyleliteralemphasis{\sphinxupquote{array}}) – The technological parameterization.

\end{itemize}

\end{description}\end{quote}

\end{fulllineitems}

\index{fixed\_cost() (in module technology.transport\_model)@\spxentry{fixed\_cost()}\spxextra{in module technology.transport\_model}}

\begin{fulllineitems}
\phantomsection\label{\detokenize{technology:technology.transport_model.fixed_cost}}
\pysigstartsignatures
\pysiglinewithargsret{\sphinxcode{\sphinxupquote{technology.transport\_model.}}\sphinxbfcode{\sphinxupquote{fixed\_cost}}}{\emph{\DUrole{n}{scale}}, \emph{\DUrole{n}{parameter}}}{}
\pysigstopsignatures
\sphinxAtStartPar
Capital cost function.
\begin{quote}\begin{description}
\sphinxlineitem{Parameters}\begin{itemize}
\item {} 
\sphinxAtStartPar
\sphinxstyleliteralstrong{\sphinxupquote{scale}} (\sphinxstyleliteralemphasis{\sphinxupquote{float}}) – The scale of operation.

\item {} 
\sphinxAtStartPar
\sphinxstyleliteralstrong{\sphinxupquote{parameter}} (\sphinxstyleliteralemphasis{\sphinxupquote{array}}) – The technological parameterization.

\end{itemize}

\end{description}\end{quote}

\end{fulllineitems}

\index{metrics() (in module technology.transport\_model)@\spxentry{metrics()}\spxextra{in module technology.transport\_model}}

\begin{fulllineitems}
\phantomsection\label{\detokenize{technology:technology.transport_model.metrics}}
\pysigstartsignatures
\pysiglinewithargsret{\sphinxcode{\sphinxupquote{technology.transport\_model.}}\sphinxbfcode{\sphinxupquote{metrics}}}{\emph{\DUrole{n}{scale}}, \emph{\DUrole{n}{capital}}, \emph{\DUrole{n}{lifetime}}, \emph{\DUrole{n}{fixed}}, \emph{\DUrole{n}{input\_raw}}, \emph{\DUrole{n}{input}}, \emph{\DUrole{n}{input\_price}}, \emph{\DUrole{n}{output\_raw}}, \emph{\DUrole{n}{output}}, \emph{\DUrole{n}{cost}}, \emph{\DUrole{n}{parameter}}}{}
\pysigstopsignatures
\sphinxAtStartPar
Metrics function.
\begin{quote}\begin{description}
\sphinxlineitem{Parameters}\begin{itemize}
\item {} 
\sphinxAtStartPar
\sphinxstyleliteralstrong{\sphinxupquote{scale}} (\sphinxstyleliteralemphasis{\sphinxupquote{float}}) – The scale of operation.

\item {} 
\sphinxAtStartPar
\sphinxstyleliteralstrong{\sphinxupquote{capital}} (\sphinxstyleliteralemphasis{\sphinxupquote{array}}) – Capital costs.

\item {} 
\sphinxAtStartPar
\sphinxstyleliteralstrong{\sphinxupquote{lifetime}} (\sphinxstyleliteralemphasis{\sphinxupquote{float}}) – Technology lifetime.

\item {} 
\sphinxAtStartPar
\sphinxstyleliteralstrong{\sphinxupquote{fixed}} (\sphinxstyleliteralemphasis{\sphinxupquote{array}}) – Fixed costs.

\item {} 
\sphinxAtStartPar
\sphinxstyleliteralstrong{\sphinxupquote{input\_raw}} (\sphinxstyleliteralemphasis{\sphinxupquote{array}}) – Raw input quantities (before losses).

\item {} 
\sphinxAtStartPar
\sphinxstyleliteralstrong{\sphinxupquote{input}} (\sphinxstyleliteralemphasis{\sphinxupquote{array}}) – Input quantities.

\item {} 
\sphinxAtStartPar
\sphinxstyleliteralstrong{\sphinxupquote{output\_raw}} (\sphinxstyleliteralemphasis{\sphinxupquote{array}}) – Raw output quantities (before losses).

\item {} 
\sphinxAtStartPar
\sphinxstyleliteralstrong{\sphinxupquote{output}} (\sphinxstyleliteralemphasis{\sphinxupquote{array}}) – Output quantities.

\item {} 
\sphinxAtStartPar
\sphinxstyleliteralstrong{\sphinxupquote{cost}} (\sphinxstyleliteralemphasis{\sphinxupquote{array}}) – Costs.

\item {} 
\sphinxAtStartPar
\sphinxstyleliteralstrong{\sphinxupquote{parameter}} (\sphinxstyleliteralemphasis{\sphinxupquote{array}}) – The technological parameterization.

\end{itemize}

\end{description}\end{quote}

\end{fulllineitems}

\index{production() (in module technology.transport\_model)@\spxentry{production()}\spxextra{in module technology.transport\_model}}

\begin{fulllineitems}
\phantomsection\label{\detokenize{technology:technology.transport_model.production}}
\pysigstartsignatures
\pysiglinewithargsret{\sphinxcode{\sphinxupquote{technology.transport\_model.}}\sphinxbfcode{\sphinxupquote{production}}}{\emph{\DUrole{n}{scale}}, \emph{\DUrole{n}{capital}}, \emph{\DUrole{n}{lifetime}}, \emph{\DUrole{n}{fixed}}, \emph{\DUrole{n}{input}}, \emph{\DUrole{n}{parameter}}}{}
\pysigstopsignatures
\sphinxAtStartPar
Production function.
\begin{quote}\begin{description}
\sphinxlineitem{Parameters}\begin{itemize}
\item {} 
\sphinxAtStartPar
\sphinxstyleliteralstrong{\sphinxupquote{scale}} (\sphinxstyleliteralemphasis{\sphinxupquote{float}}) – The scale of operation.

\item {} 
\sphinxAtStartPar
\sphinxstyleliteralstrong{\sphinxupquote{capital}} (\sphinxstyleliteralemphasis{\sphinxupquote{array}}) – Capital costs.

\item {} 
\sphinxAtStartPar
\sphinxstyleliteralstrong{\sphinxupquote{lifetime}} (\sphinxstyleliteralemphasis{\sphinxupquote{float}}) – Technology lifetime.

\item {} 
\sphinxAtStartPar
\sphinxstyleliteralstrong{\sphinxupquote{fixed}} (\sphinxstyleliteralemphasis{\sphinxupquote{array}}) – Fixed costs.

\item {} 
\sphinxAtStartPar
\sphinxstyleliteralstrong{\sphinxupquote{input}} (\sphinxstyleliteralemphasis{\sphinxupquote{array}}) – Input quantities.

\item {} 
\sphinxAtStartPar
\sphinxstyleliteralstrong{\sphinxupquote{parameter}} (\sphinxstyleliteralemphasis{\sphinxupquote{array}}) – The technological parameterization.

\end{itemize}

\end{description}\end{quote}

\end{fulllineitems}



\subsection{Tutorial Technologies}
\label{\detokenize{technology:tutorial-technologies}}
\sphinxAtStartPar
The technology models in this section are for exploratory and learning purposes only.


\subsubsection{Simple Electrolysis}
\label{\detokenize{technology:module-technology.simple_electrolysis}}\label{\detokenize{technology:simple-electrolysis}}\index{module@\spxentry{module}!technology.simple\_electrolysis@\spxentry{technology.simple\_electrolysis}}\index{technology.simple\_electrolysis@\spxentry{technology.simple\_electrolysis}!module@\spxentry{module}}
\sphinxAtStartPar
Simple electrolysis.
\index{capital\_cost() (in module technology.simple\_electrolysis)@\spxentry{capital\_cost()}\spxextra{in module technology.simple\_electrolysis}}

\begin{fulllineitems}
\phantomsection\label{\detokenize{technology:technology.simple_electrolysis.capital_cost}}
\pysigstartsignatures
\pysiglinewithargsret{\sphinxcode{\sphinxupquote{technology.simple\_electrolysis.}}\sphinxbfcode{\sphinxupquote{capital\_cost}}}{\emph{\DUrole{n}{scale}}, \emph{\DUrole{n}{parameter}}}{}
\pysigstopsignatures
\sphinxAtStartPar
Capital cost function.
\begin{quote}\begin{description}
\sphinxlineitem{Parameters}\begin{itemize}
\item {} 
\sphinxAtStartPar
\sphinxstyleliteralstrong{\sphinxupquote{scale}} (\sphinxstyleliteralemphasis{\sphinxupquote{float}}) – The scale of operation.

\item {} 
\sphinxAtStartPar
\sphinxstyleliteralstrong{\sphinxupquote{parameter}} (\sphinxstyleliteralemphasis{\sphinxupquote{array}}) – The technological parameterization.

\end{itemize}

\end{description}\end{quote}

\end{fulllineitems}

\index{fixed\_cost() (in module technology.simple\_electrolysis)@\spxentry{fixed\_cost()}\spxextra{in module technology.simple\_electrolysis}}

\begin{fulllineitems}
\phantomsection\label{\detokenize{technology:technology.simple_electrolysis.fixed_cost}}
\pysigstartsignatures
\pysiglinewithargsret{\sphinxcode{\sphinxupquote{technology.simple\_electrolysis.}}\sphinxbfcode{\sphinxupquote{fixed\_cost}}}{\emph{\DUrole{n}{scale}}, \emph{\DUrole{n}{parameter}}}{}
\pysigstopsignatures
\sphinxAtStartPar
Fixed cost function.
\begin{quote}\begin{description}
\sphinxlineitem{Parameters}\begin{itemize}
\item {} 
\sphinxAtStartPar
\sphinxstyleliteralstrong{\sphinxupquote{scale}} (\sphinxstyleliteralemphasis{\sphinxupquote{float}}) – The scale of operation.

\item {} 
\sphinxAtStartPar
\sphinxstyleliteralstrong{\sphinxupquote{parameter}} (\sphinxstyleliteralemphasis{\sphinxupquote{array}}) – The technological parameterization.

\end{itemize}

\end{description}\end{quote}

\end{fulllineitems}

\index{metrics() (in module technology.simple\_electrolysis)@\spxentry{metrics()}\spxextra{in module technology.simple\_electrolysis}}

\begin{fulllineitems}
\phantomsection\label{\detokenize{technology:technology.simple_electrolysis.metrics}}
\pysigstartsignatures
\pysiglinewithargsret{\sphinxcode{\sphinxupquote{technology.simple\_electrolysis.}}\sphinxbfcode{\sphinxupquote{metrics}}}{\emph{\DUrole{n}{scale}}, \emph{\DUrole{n}{capital}}, \emph{\DUrole{n}{lifetime}}, \emph{\DUrole{n}{fixed}}, \emph{\DUrole{n}{input\_raw}}, \emph{\DUrole{n}{input}}, \emph{\DUrole{n}{input\_price}}, \emph{\DUrole{n}{output\_raw}}, \emph{\DUrole{n}{output}}, \emph{\DUrole{n}{cost}}, \emph{\DUrole{n}{parameter}}}{}
\pysigstopsignatures
\sphinxAtStartPar
Metrics function.
\begin{quote}\begin{description}
\sphinxlineitem{Parameters}\begin{itemize}
\item {} 
\sphinxAtStartPar
\sphinxstyleliteralstrong{\sphinxupquote{scale}} (\sphinxstyleliteralemphasis{\sphinxupquote{float}}) – The scale of operation.

\item {} 
\sphinxAtStartPar
\sphinxstyleliteralstrong{\sphinxupquote{capital}} (\sphinxstyleliteralemphasis{\sphinxupquote{array}}) – Capital costs.

\item {} 
\sphinxAtStartPar
\sphinxstyleliteralstrong{\sphinxupquote{lifetime}} (\sphinxstyleliteralemphasis{\sphinxupquote{float}}) – Technology lifetime.

\item {} 
\sphinxAtStartPar
\sphinxstyleliteralstrong{\sphinxupquote{fixed}} (\sphinxstyleliteralemphasis{\sphinxupquote{array}}) – Fixed costs.

\item {} 
\sphinxAtStartPar
\sphinxstyleliteralstrong{\sphinxupquote{input\_raw}} (\sphinxstyleliteralemphasis{\sphinxupquote{array}}) – Raw input quantities (before losses).

\item {} 
\sphinxAtStartPar
\sphinxstyleliteralstrong{\sphinxupquote{input}} (\sphinxstyleliteralemphasis{\sphinxupquote{array}}) – Input quantities.

\item {} 
\sphinxAtStartPar
\sphinxstyleliteralstrong{\sphinxupquote{output\_raw}} (\sphinxstyleliteralemphasis{\sphinxupquote{array}}) – Raw output quantities (before losses).

\item {} 
\sphinxAtStartPar
\sphinxstyleliteralstrong{\sphinxupquote{output}} (\sphinxstyleliteralemphasis{\sphinxupquote{array}}) – Output quantities.

\item {} 
\sphinxAtStartPar
\sphinxstyleliteralstrong{\sphinxupquote{cost}} (\sphinxstyleliteralemphasis{\sphinxupquote{array}}) – Costs.

\item {} 
\sphinxAtStartPar
\sphinxstyleliteralstrong{\sphinxupquote{parameter}} (\sphinxstyleliteralemphasis{\sphinxupquote{array}}) – The technological parameterization.

\end{itemize}

\end{description}\end{quote}

\end{fulllineitems}

\index{production() (in module technology.simple\_electrolysis)@\spxentry{production()}\spxextra{in module technology.simple\_electrolysis}}

\begin{fulllineitems}
\phantomsection\label{\detokenize{technology:technology.simple_electrolysis.production}}
\pysigstartsignatures
\pysiglinewithargsret{\sphinxcode{\sphinxupquote{technology.simple\_electrolysis.}}\sphinxbfcode{\sphinxupquote{production}}}{\emph{\DUrole{n}{scale}}, \emph{\DUrole{n}{capital}}, \emph{\DUrole{n}{lifetime}}, \emph{\DUrole{n}{fixed}}, \emph{\DUrole{n}{input}}, \emph{\DUrole{n}{parameter}}}{}
\pysigstopsignatures
\sphinxAtStartPar
Production function.
\begin{quote}\begin{description}
\sphinxlineitem{Parameters}\begin{itemize}
\item {} 
\sphinxAtStartPar
\sphinxstyleliteralstrong{\sphinxupquote{scale}} (\sphinxstyleliteralemphasis{\sphinxupquote{float}}) – The scale of operation.

\item {} 
\sphinxAtStartPar
\sphinxstyleliteralstrong{\sphinxupquote{capital}} (\sphinxstyleliteralemphasis{\sphinxupquote{array}}) – Capital costs.

\item {} 
\sphinxAtStartPar
\sphinxstyleliteralstrong{\sphinxupquote{lifetime}} (\sphinxstyleliteralemphasis{\sphinxupquote{float}}) – Technology lifetime.

\item {} 
\sphinxAtStartPar
\sphinxstyleliteralstrong{\sphinxupquote{fixed}} (\sphinxstyleliteralemphasis{\sphinxupquote{array}}) – Fixed costs.

\item {} 
\sphinxAtStartPar
\sphinxstyleliteralstrong{\sphinxupquote{input}} (\sphinxstyleliteralemphasis{\sphinxupquote{array}}) – Input quantities.

\item {} 
\sphinxAtStartPar
\sphinxstyleliteralstrong{\sphinxupquote{parameter}} (\sphinxstyleliteralemphasis{\sphinxupquote{array}}) – The technological parameterization.

\end{itemize}

\end{description}\end{quote}

\end{fulllineitems}



\subsubsection{Toy Biorefinery}
\label{\detokenize{technology:module-technology.tutorial_biorefinery}}\label{\detokenize{technology:toy-biorefinery}}\index{module@\spxentry{module}!technology.tutorial\_biorefinery@\spxentry{technology.tutorial\_biorefinery}}\index{technology.tutorial\_biorefinery@\spxentry{technology.tutorial\_biorefinery}!module@\spxentry{module}}
\sphinxAtStartPar
Biorefinery model with four processing steps.
\index{capital\_cost() (in module technology.tutorial\_biorefinery)@\spxentry{capital\_cost()}\spxextra{in module technology.tutorial\_biorefinery}}

\begin{fulllineitems}
\phantomsection\label{\detokenize{technology:technology.tutorial_biorefinery.capital_cost}}
\pysigstartsignatures
\pysiglinewithargsret{\sphinxcode{\sphinxupquote{technology.tutorial\_biorefinery.}}\sphinxbfcode{\sphinxupquote{capital\_cost}}}{\emph{\DUrole{n}{scale}}, \emph{\DUrole{n}{parameter}}}{}
\pysigstopsignatures
\sphinxAtStartPar
Capital cost function.
\begin{quote}\begin{description}
\sphinxlineitem{Parameters}\begin{itemize}
\item {} 
\sphinxAtStartPar
\sphinxstyleliteralstrong{\sphinxupquote{scale}} (\sphinxstyleliteralemphasis{\sphinxupquote{float}}) – The scale of operation.

\item {} 
\sphinxAtStartPar
\sphinxstyleliteralstrong{\sphinxupquote{parameter}} (\sphinxstyleliteralemphasis{\sphinxupquote{array}}) – The technological parameterization.

\end{itemize}

\sphinxlineitem{Return type}
\sphinxAtStartPar
Total capital cost for one biorefinery (USD/biorefinery)

\end{description}\end{quote}

\end{fulllineitems}

\index{fixed\_cost() (in module technology.tutorial\_biorefinery)@\spxentry{fixed\_cost()}\spxextra{in module technology.tutorial\_biorefinery}}

\begin{fulllineitems}
\phantomsection\label{\detokenize{technology:technology.tutorial_biorefinery.fixed_cost}}
\pysigstartsignatures
\pysiglinewithargsret{\sphinxcode{\sphinxupquote{technology.tutorial\_biorefinery.}}\sphinxbfcode{\sphinxupquote{fixed\_cost}}}{\emph{\DUrole{n}{scale}}, \emph{\DUrole{n}{parameter}}}{}
\pysigstopsignatures
\sphinxAtStartPar
Fixed cost function.
\begin{quote}\begin{description}
\sphinxlineitem{Parameters}\begin{itemize}
\item {} 
\sphinxAtStartPar
\sphinxstyleliteralstrong{\sphinxupquote{scale}} (\sphinxstyleliteralemphasis{\sphinxupquote{float}}\sphinxstyleliteralemphasis{\sphinxupquote{ {[}}}\sphinxstyleliteralemphasis{\sphinxupquote{Unused}}\sphinxstyleliteralemphasis{\sphinxupquote{{]}}}) – The scale of operation.

\item {} 
\sphinxAtStartPar
\sphinxstyleliteralstrong{\sphinxupquote{parameter}} (\sphinxstyleliteralemphasis{\sphinxupquote{array}}) – The technological parameterization.

\end{itemize}

\sphinxlineitem{Return type}
\sphinxAtStartPar
total fixed costs for one biorefinery (USD/year)

\end{description}\end{quote}

\end{fulllineitems}

\index{metrics() (in module technology.tutorial\_biorefinery)@\spxentry{metrics()}\spxextra{in module technology.tutorial\_biorefinery}}

\begin{fulllineitems}
\phantomsection\label{\detokenize{technology:technology.tutorial_biorefinery.metrics}}
\pysigstartsignatures
\pysiglinewithargsret{\sphinxcode{\sphinxupquote{technology.tutorial\_biorefinery.}}\sphinxbfcode{\sphinxupquote{metrics}}}{\emph{\DUrole{n}{scale}}, \emph{\DUrole{n}{capital}}, \emph{\DUrole{n}{lifetime}}, \emph{\DUrole{n}{fixed}}, \emph{\DUrole{n}{input\_raw}}, \emph{\DUrole{n}{input}}, \emph{\DUrole{n}{input\_price}}, \emph{\DUrole{n}{output\_raw}}, \emph{\DUrole{n}{output}}, \emph{\DUrole{n}{cost}}, \emph{\DUrole{n}{parameter}}}{}
\pysigstopsignatures
\sphinxAtStartPar
Metrics function.
\begin{quote}\begin{description}
\sphinxlineitem{Parameters}\begin{itemize}
\item {} 
\sphinxAtStartPar
\sphinxstyleliteralstrong{\sphinxupquote{scale}} (\sphinxstyleliteralemphasis{\sphinxupquote{float}}) – The scale of operation. Unitless

\item {} 
\sphinxAtStartPar
\sphinxstyleliteralstrong{\sphinxupquote{capital}} (\sphinxstyleliteralemphasis{\sphinxupquote{array}}) – Capital costs. Units: USD/biorefinery

\item {} 
\sphinxAtStartPar
\sphinxstyleliteralstrong{\sphinxupquote{lifetime}} (\sphinxstyleliteralemphasis{\sphinxupquote{float}}) – Technology lifetime. Units: year

\item {} 
\sphinxAtStartPar
\sphinxstyleliteralstrong{\sphinxupquote{fixed}} (\sphinxstyleliteralemphasis{\sphinxupquote{array}}) – Fixed costs. Units: USD/year

\item {} 
\sphinxAtStartPar
\sphinxstyleliteralstrong{\sphinxupquote{input\_raw}} (\sphinxstyleliteralemphasis{\sphinxupquote{array}}) – Raw input quantities (before losses). Units: metric ton feedstock/year

\item {} 
\sphinxAtStartPar
\sphinxstyleliteralstrong{\sphinxupquote{input}} (\sphinxstyleliteralemphasis{\sphinxupquote{array}}) – Input quantities. Units: metric ton feedstock/year

\item {} 
\sphinxAtStartPar
\sphinxstyleliteralstrong{\sphinxupquote{input\_price}} (\sphinxstyleliteralemphasis{\sphinxupquote{array`}}) – Array of input prices. Various units.

\item {} 
\sphinxAtStartPar
\sphinxstyleliteralstrong{\sphinxupquote{output\_raw}} (\sphinxstyleliteralemphasis{\sphinxupquote{array}}) – Raw output quantities (before losses). Units: gal biofuel/year

\item {} 
\sphinxAtStartPar
\sphinxstyleliteralstrong{\sphinxupquote{output}} (\sphinxstyleliteralemphasis{\sphinxupquote{array}}) – Output quantities. Units: gal biofuel/year

\item {} 
\sphinxAtStartPar
\sphinxstyleliteralstrong{\sphinxupquote{cost}} (\sphinxstyleliteralemphasis{\sphinxupquote{array}}) – Costs.

\item {} 
\sphinxAtStartPar
\sphinxstyleliteralstrong{\sphinxupquote{parameter}} (\sphinxstyleliteralemphasis{\sphinxupquote{array}}) – The technological parameterization. Units vary; given in comments below

\end{itemize}

\end{description}\end{quote}

\end{fulllineitems}

\index{production() (in module technology.tutorial\_biorefinery)@\spxentry{production()}\spxextra{in module technology.tutorial\_biorefinery}}

\begin{fulllineitems}
\phantomsection\label{\detokenize{technology:technology.tutorial_biorefinery.production}}
\pysigstartsignatures
\pysiglinewithargsret{\sphinxcode{\sphinxupquote{technology.tutorial\_biorefinery.}}\sphinxbfcode{\sphinxupquote{production}}}{\emph{\DUrole{n}{scale}}, \emph{\DUrole{n}{capital}}, \emph{\DUrole{n}{lifetime}}, \emph{\DUrole{n}{fixed}}, \emph{\DUrole{n}{input}}, \emph{\DUrole{n}{parameter}}}{}
\pysigstopsignatures
\sphinxAtStartPar
Production function.
\begin{quote}\begin{description}
\sphinxlineitem{Parameters}\begin{itemize}
\item {} 
\sphinxAtStartPar
\sphinxstyleliteralstrong{\sphinxupquote{scale}} (\sphinxstyleliteralemphasis{\sphinxupquote{float}}) – The scale of operation.

\item {} 
\sphinxAtStartPar
\sphinxstyleliteralstrong{\sphinxupquote{capital}} (\sphinxstyleliteralemphasis{\sphinxupquote{array}}) – Capital costs.

\item {} 
\sphinxAtStartPar
\sphinxstyleliteralstrong{\sphinxupquote{lifetime}} (\sphinxstyleliteralemphasis{\sphinxupquote{float}}) – Technology lifetime.

\item {} 
\sphinxAtStartPar
\sphinxstyleliteralstrong{\sphinxupquote{fixed}} (\sphinxstyleliteralemphasis{\sphinxupquote{array}}) – Fixed costs.

\item {} 
\sphinxAtStartPar
\sphinxstyleliteralstrong{\sphinxupquote{input}} (\sphinxstyleliteralemphasis{\sphinxupquote{array}}) – Input quantities.

\item {} 
\sphinxAtStartPar
\sphinxstyleliteralstrong{\sphinxupquote{parameter}} (\sphinxstyleliteralemphasis{\sphinxupquote{array}}) – The technological parameterization.

\end{itemize}

\sphinxlineitem{Returns}
\sphinxAtStartPar
Ideal/theoretical production of each technology output: biofuel at
gals/year

\sphinxlineitem{Return type}
\sphinxAtStartPar
output\_raw

\end{description}\end{quote}

\end{fulllineitems}



\subsubsection{Onshore Wind Turbines}
\label{\detokenize{technology:module-technology.tutorial_basic}}\label{\detokenize{technology:onshore-wind-turbines}}\index{module@\spxentry{module}!technology.tutorial\_basic@\spxentry{technology.tutorial\_basic}}\index{technology.tutorial\_basic@\spxentry{technology.tutorial\_basic}!module@\spxentry{module}}
\sphinxAtStartPar
Template for technology functions.
\index{capital\_cost() (in module technology.tutorial\_basic)@\spxentry{capital\_cost()}\spxextra{in module technology.tutorial\_basic}}

\begin{fulllineitems}
\phantomsection\label{\detokenize{technology:technology.tutorial_basic.capital_cost}}
\pysigstartsignatures
\pysiglinewithargsret{\sphinxcode{\sphinxupquote{technology.tutorial\_basic.}}\sphinxbfcode{\sphinxupquote{capital\_cost}}}{\emph{\DUrole{n}{scale}}, \emph{\DUrole{n}{parameter}}}{}
\pysigstopsignatures
\sphinxAtStartPar
Capital cost function.
\begin{quote}\begin{description}
\sphinxlineitem{Parameters}\begin{itemize}
\item {} 
\sphinxAtStartPar
\sphinxstyleliteralstrong{\sphinxupquote{scale}} (\sphinxstyleliteralemphasis{\sphinxupquote{float}}) – The scale of operation.

\item {} 
\sphinxAtStartPar
\sphinxstyleliteralstrong{\sphinxupquote{parameter}} (\sphinxstyleliteralemphasis{\sphinxupquote{array}}) – The technological parameterization.

\end{itemize}

\end{description}\end{quote}

\end{fulllineitems}

\index{fixed\_cost() (in module technology.tutorial\_basic)@\spxentry{fixed\_cost()}\spxextra{in module technology.tutorial\_basic}}

\begin{fulllineitems}
\phantomsection\label{\detokenize{technology:technology.tutorial_basic.fixed_cost}}
\pysigstartsignatures
\pysiglinewithargsret{\sphinxcode{\sphinxupquote{technology.tutorial\_basic.}}\sphinxbfcode{\sphinxupquote{fixed\_cost}}}{\emph{\DUrole{n}{scale}}, \emph{\DUrole{n}{parameter}}}{}
\pysigstopsignatures
\sphinxAtStartPar
Capital cost function.
\begin{quote}\begin{description}
\sphinxlineitem{Parameters}\begin{itemize}
\item {} 
\sphinxAtStartPar
\sphinxstyleliteralstrong{\sphinxupquote{scale}} (\sphinxstyleliteralemphasis{\sphinxupquote{float}}) – The scale of operation.

\item {} 
\sphinxAtStartPar
\sphinxstyleliteralstrong{\sphinxupquote{parameter}} (\sphinxstyleliteralemphasis{\sphinxupquote{array}}) – The technological parameterization.

\end{itemize}

\end{description}\end{quote}

\end{fulllineitems}

\index{metrics() (in module technology.tutorial\_basic)@\spxentry{metrics()}\spxextra{in module technology.tutorial\_basic}}

\begin{fulllineitems}
\phantomsection\label{\detokenize{technology:technology.tutorial_basic.metrics}}
\pysigstartsignatures
\pysiglinewithargsret{\sphinxcode{\sphinxupquote{technology.tutorial\_basic.}}\sphinxbfcode{\sphinxupquote{metrics}}}{\emph{\DUrole{n}{scale}}, \emph{\DUrole{n}{capital}}, \emph{\DUrole{n}{lifetime}}, \emph{\DUrole{n}{fixed}}, \emph{\DUrole{n}{input\_raw}}, \emph{\DUrole{n}{input}}, \emph{\DUrole{n}{input\_price}}, \emph{\DUrole{n}{output\_raw}}, \emph{\DUrole{n}{output}}, \emph{\DUrole{n}{cost}}, \emph{\DUrole{n}{parameter}}}{}
\pysigstopsignatures
\sphinxAtStartPar
Metrics function.
\begin{quote}\begin{description}
\sphinxlineitem{Parameters}\begin{itemize}
\item {} 
\sphinxAtStartPar
\sphinxstyleliteralstrong{\sphinxupquote{scale}} (\sphinxstyleliteralemphasis{\sphinxupquote{float}}) – The scale of operation.

\item {} 
\sphinxAtStartPar
\sphinxstyleliteralstrong{\sphinxupquote{capital}} (\sphinxstyleliteralemphasis{\sphinxupquote{array}}) – Capital costs.

\item {} 
\sphinxAtStartPar
\sphinxstyleliteralstrong{\sphinxupquote{lifetime}} (\sphinxstyleliteralemphasis{\sphinxupquote{float}}) – Technology lifetime.

\item {} 
\sphinxAtStartPar
\sphinxstyleliteralstrong{\sphinxupquote{fixed}} (\sphinxstyleliteralemphasis{\sphinxupquote{array}}) – Fixed costs.

\item {} 
\sphinxAtStartPar
\sphinxstyleliteralstrong{\sphinxupquote{input\_raw}} (\sphinxstyleliteralemphasis{\sphinxupquote{array}}) – Raw input quantities (before losses).

\item {} 
\sphinxAtStartPar
\sphinxstyleliteralstrong{\sphinxupquote{input}} (\sphinxstyleliteralemphasis{\sphinxupquote{array}}) – Input quantities.

\item {} 
\sphinxAtStartPar
\sphinxstyleliteralstrong{\sphinxupquote{output\_raw}} (\sphinxstyleliteralemphasis{\sphinxupquote{array}}) – Raw output quantities (before losses).

\item {} 
\sphinxAtStartPar
\sphinxstyleliteralstrong{\sphinxupquote{output}} (\sphinxstyleliteralemphasis{\sphinxupquote{array}}) – Output quantities.

\item {} 
\sphinxAtStartPar
\sphinxstyleliteralstrong{\sphinxupquote{cost}} (\sphinxstyleliteralemphasis{\sphinxupquote{array}}) – Costs.

\item {} 
\sphinxAtStartPar
\sphinxstyleliteralstrong{\sphinxupquote{parameter}} (\sphinxstyleliteralemphasis{\sphinxupquote{array}}) – The technological parameterization.

\end{itemize}

\end{description}\end{quote}

\end{fulllineitems}

\index{production() (in module technology.tutorial\_basic)@\spxentry{production()}\spxextra{in module technology.tutorial\_basic}}

\begin{fulllineitems}
\phantomsection\label{\detokenize{technology:technology.tutorial_basic.production}}
\pysigstartsignatures
\pysiglinewithargsret{\sphinxcode{\sphinxupquote{technology.tutorial\_basic.}}\sphinxbfcode{\sphinxupquote{production}}}{\emph{\DUrole{n}{scale}}, \emph{\DUrole{n}{capital}}, \emph{\DUrole{n}{lifetime}}, \emph{\DUrole{n}{fixed}}, \emph{\DUrole{n}{input}}, \emph{\DUrole{n}{parameter}}}{}
\pysigstopsignatures
\sphinxAtStartPar
Production function.
\begin{quote}\begin{description}
\sphinxlineitem{Parameters}\begin{itemize}
\item {} 
\sphinxAtStartPar
\sphinxstyleliteralstrong{\sphinxupquote{scale}} (\sphinxstyleliteralemphasis{\sphinxupquote{float}}) – The scale of operation.

\item {} 
\sphinxAtStartPar
\sphinxstyleliteralstrong{\sphinxupquote{capital}} (\sphinxstyleliteralemphasis{\sphinxupquote{array}}) – Capital costs.

\item {} 
\sphinxAtStartPar
\sphinxstyleliteralstrong{\sphinxupquote{lifetime}} (\sphinxstyleliteralemphasis{\sphinxupquote{float}}) – Technology lifetime.

\item {} 
\sphinxAtStartPar
\sphinxstyleliteralstrong{\sphinxupquote{fixed}} (\sphinxstyleliteralemphasis{\sphinxupquote{array}}) – Fixed costs.

\item {} 
\sphinxAtStartPar
\sphinxstyleliteralstrong{\sphinxupquote{input}} (\sphinxstyleliteralemphasis{\sphinxupquote{array}}) – Input quantities.

\item {} 
\sphinxAtStartPar
\sphinxstyleliteralstrong{\sphinxupquote{parameter}} (\sphinxstyleliteralemphasis{\sphinxupquote{array}}) – The technological parameterization.

\end{itemize}

\end{description}\end{quote}

\end{fulllineitems}



\subsection{Module contents}
\label{\detokenize{technology:module-technology}}\label{\detokenize{technology:module-contents}}\index{module@\spxentry{module}!technology@\spxentry{technology}}\index{technology@\spxentry{technology}!module@\spxentry{module}}

\renewcommand{\indexname}{Python Module Index}
\begin{sphinxtheindex}
\let\bigletter\sphinxstyleindexlettergroup
\bigletter{t}
\item\relax\sphinxstyleindexentry{technology}\sphinxstyleindexpageref{technology:\detokenize{module-technology}}
\item\relax\sphinxstyleindexentry{technology.pv\_residential\_large}\sphinxstyleindexpageref{technology:\detokenize{module-technology.pv_residential_large}}
\item\relax\sphinxstyleindexentry{technology.pv\_residential\_simple}\sphinxstyleindexpageref{technology:\detokenize{module-technology.pv_residential_simple}}
\item\relax\sphinxstyleindexentry{technology.simple\_electrolysis}\sphinxstyleindexpageref{technology:\detokenize{module-technology.simple_electrolysis}}
\item\relax\sphinxstyleindexentry{technology.transport\_model}\sphinxstyleindexpageref{technology:\detokenize{module-technology.transport_model}}
\item\relax\sphinxstyleindexentry{technology.tutorial\_basic}\sphinxstyleindexpageref{technology:\detokenize{module-technology.tutorial_basic}}
\item\relax\sphinxstyleindexentry{technology.tutorial\_biorefinery}\sphinxstyleindexpageref{technology:\detokenize{module-technology.tutorial_biorefinery}}
\item\relax\sphinxstyleindexentry{technology.utility\_pv}\sphinxstyleindexpageref{technology:\detokenize{module-technology.utility_pv}}
\item\relax\sphinxstyleindexentry{tyche}\sphinxstyleindexpageref{tyche:\detokenize{module-tyche}}
\item\relax\sphinxstyleindexentry{tyche.DecisionGUI}\sphinxstyleindexpageref{tyche:\detokenize{module-tyche.DecisionGUI}}
\item\relax\sphinxstyleindexentry{tyche.Designs}\sphinxstyleindexpageref{tyche:\detokenize{module-tyche.Designs}}
\item\relax\sphinxstyleindexentry{tyche.Distributions}\sphinxstyleindexpageref{tyche:\detokenize{module-tyche.Distributions}}
\item\relax\sphinxstyleindexentry{tyche.EpsilonConstraints}\sphinxstyleindexpageref{tyche:\detokenize{module-tyche.EpsilonConstraints}}
\item\relax\sphinxstyleindexentry{tyche.Evaluator}\sphinxstyleindexpageref{tyche:\detokenize{module-tyche.Evaluator}}
\item\relax\sphinxstyleindexentry{tyche.Investments}\sphinxstyleindexpageref{tyche:\detokenize{module-tyche.Investments}}
\item\relax\sphinxstyleindexentry{tyche.IO}\sphinxstyleindexpageref{tyche:\detokenize{module-tyche.IO}}
\item\relax\sphinxstyleindexentry{tyche.Types}\sphinxstyleindexpageref{tyche:\detokenize{module-tyche.Types}}
\end{sphinxtheindex}

\renewcommand{\indexname}{Index}
\printindex
\end{document}